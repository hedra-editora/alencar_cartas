\chapter[Introdução, por Tâmis Parron]{Introdução}
\hedramarkboth{Introdução}{Tâmis Parron}

\noindent\textsc{Em 1867,} José de Alencar começou a publicar \textit{Ao imperador: novas
cartas políticas de Erasmo}, provavelmente a mais controversa de suas
dezenas de obras. Nela não compôs romance nem peça de teatro, não
teorizou a nacionalidade brasileira nem a estética literária, áreas que
o erigiram em mestre do vernáculo para os contemporâneos e gerações
futuras. As \textit{Novas cartas políticas} tiveram por objeto
principal um assunto bem menos nobre, ao menos para o leitor de hoje: a
defesa da escravidão negra no Brasil. Dirigidas ao imperador D.~Pedro
\textsc{ii} em tom contundente e pedagógico, elas davam continuidade à primeira
série epistolar, \textit{Ao imperador: cartas} (1865), em que o
romancista tinha apostrofado o monarca para tratar de outros problemas
políticos, como a conflituosa relação entre a Coroa, o Executivo e o
Parlamento.\footnote{ Vide José de Alencar. \textit{Ao Imperador:
cartas}. Rio de Janeiro: Typ.~de Mello, 1865. No ano seguinte, essa
primeira sequência ganhou o título \textit{Ao Imperador: cartas
políticas de Erasmo}. 3\ai ed.~Rio de Janeiro: Typ.~de Pinheiro e Cia, 1866.} 

Os dois conjuntos de cartas tiveram, porém, trajetória editorial e
fortuna crítica distintas, por conta de seus respectivos assuntos.
Enquanto a série mais antiga, relativamente citada na historiografia,
acabou inserida na \textit{Obra completa} que a Editora José Aguilar
lançou em 1959, a segunda foi excluída do \textit{corpus} de textos do
autor, na provável tentativa de expurgar sua memória artística de uma
posição moralmente insustentável para os padrões culturais hegemônicos
desde o final do século \textsc{xix}. É justamente essa lacuna que a presente
edição propõe reparar, para enriquecer o repertório discursivo do
Império do Brasil com uma composição que possa interessar aos
estudiosos da política, da escravidão e da literatura, bem como ao
público em geral que se sinta contemplado nas discussões recentes sobre
raça e mobilidade social no Brasil.\footnote{ A relação de obras que não mencionam 
nem analisam as \textit{Novas cartas políticas} é extensa. Entre as omissões mais relevantes, 
pode"-se citar José de Alencar. \textit{Obra completa}. Rio de Janeiro: José Aguilar, 1960, v. \textsc{iv}; 
o vastíssimo \textit{Catálogo da Exposição de História do Brasil}. 1\ai ed., 1881. Brasília: Editora 
da Universidade de Brasília, 3 v., 1981; e a biografia escrita por Luiz Viana Filho. 
\textit{A vida de José de Alencar}. São Paulo: Ed. Unesp/Salvador: Edufba, 2008. Cf. tb. Silvio Romero. 
\textit{História da literatura brasileira}. Org. por Nelson Romero. Rio de Janeiro: José Olympio, 1960, 
v. \textsc{v}, pp. 1464 e ss., onde se fala genericamente das ``cartas de Erasmo''. Exceção deve ser feita 
ao trabalho de Sílvia Cristina Martins de Souza, ``Um panfletista pouco conhecido do império''. 
In \textit{Cadernos de História Social}, Campinas, n. 3 (abril/1996), pp. 69--95, que republicou a segunda, 
a terceira e a quarta missivas. Note"-se apenas que, num equívoco frequente entre especialistas, 
Souza diz ter havido seis \textit{Novas cartas políticas}. De fato, a série consiste em sete epístolas.}

Como se sabe, José de Alencar (1829--1877) era escritor fecundo. São de
sua lavra mais de vinte romances, quase uma dezena de peças de teatro,
numerosas crônicas de jornal, textos de crítica literária, estudos
jurídicos, ensaios teóricos sobre a Constituição e sobre o regime
representativo, arguições parlamentares editadas em opúsculos e,
finalmente, as sequências das \textit{Cartas políticas}, publicadas sob
o pseudônimo de Erasmo, em alusão ao grande humanista neerlandês Erasmo
de Roterdã (c.~1469--1536). Desse repertório se pode dizer que os
escritos sobre política e direito constituem produção tardia em relação
ao romance e ao teatro, com datação posterior à  entrada do já
consagrado escritor no circuito dos altos cargos públicos do Império.
No caso particular das \textit{Cartas políticas}, Alencar as escreveu
após seu primeiro mandato na Câmara dos Deputados (1861--1863), obtido
sob os auspícios do Partido Conservador, que o apoiaria em toda a
carreira política, desde a reeleição como deputado"-geral
(1869--1872) e a nomeação como ministro da Justiça (1868), até a
disputa sem êxito por uma vaga na instituição mais importante do
Império, o Senado. Como foi em nome dessa agremiação que o escritor
concebeu as missivas, talvez seja necessário retomar, brevemente, suas
linhas ideológicas fundamentais.\footnote{ Entre as biografias do escritor, ver as clássicas de Raimundo de Menezes. 
\textit{José de Alencar: literato e político}. São Paulo: 
Martins Editora, 1965; e de Raimundo Magalhães Jr. \textit{José de Alencar e sua época}. São Paulo: Lisa, 1971. 
Cf. também a já citada de Luiz Viana Filho. \textit{A vida de José de Alencar}; e a recente de Lira Neto. 
\textit{O inimigo do rei: uma biografia de José de Alencar ou a mirabolante aventura de um romancista 
que colecionava desafetos, azucrinava D.~Pedro} \textsc{ii} \textit{e acabou inventando o Brasil}. São Paulo: Globo, 2006.} 

\section{política da escravidão}
 Na esteira da primeira reforma constitucional brasileira --- o Ato
Adicional, de 1834, que instituiu as Assembleias Provinciais ---,
políticos ligados a grandes proprietários rurais do Rio de Janeiro,
Minas Gerais, Pernambuco e Bahia se uniram para propor uma radical
reorganização do Estado. Sua agenda consistia em centralizar o
Judiciário do Império nas mãos do Executivo e repassar parte do
controle das eleições a funcionários nomeados direta ou indiretamente
pelo governo central. Em 1841, após anos de contendas parlamentares, o
grupo, então chamado de Regresso, realizou esse programa com uma
superlei que reformou o Código do Processo Criminal. Seus ideólogos
diziam que a ``autoridade'' e a ``ordem'' do poder central deviam
prevalecer sobre o caos da Regência (1831--1840). Nesse período, outro
princípio igualmente poderoso forjou a coesão e disciplina dos líderes
do Regresso: a defesa articulada, no Parlamento, da reabertura do
tráfico negreiro sob a forma de contrabando. Banido em tratado
internacional a partir de 1830 e proibido por uma lei nacional de 1831,
o comércio de escravos declinara nos primeiros e conturbados anos da
Regência. Entretanto, a ação decisiva e coesa dos chefes do Regresso
por meio de artigos de jornal, edição de livros, projetos de lei e
falas parlamentares, tudo acompanhado de representações (petições)
provinciais e municipais de sua base eleitoral, possibilitou a
reabertura do contrabando em larga escala, a despeito dos altos custos
diplomáticos encarnados na violenta oposição da Inglaterra, a maior
potência mundial da época. Ao fim e ao cabo, mais de 650 mil africanos
seriam ilegalmente transplantados para o Brasil entre 1836 e
1850.\footnote{ Sobre os partidos imperiais e suas ideologias, ver
Ilmar Rohloff de Mattos. \textit{Tempo saquarema}. São Paulo:
Hucitec/Instituto Nacional do Livro, 1987; José Murilo de Carvalho.
\textit{A construção da ordem. Teatro de sombras}. 1\ai ed.,
respectivamente, 1980 e 1988. Rio de Janeiro: Civilização Brasileira,
2003; e Jeffrey Needell. \textit{The Party of Order: The Conservatives,
the State and Slavery in the Brazilian Monarchy, 1831--1871}.
Stanford, California: Stanford University Press, 2006. Há ainda duas
referências inescapáveis, escritas no século \textsc{xix}: Joaquim Nabuco.
\textit{Um estadista do Império --- Nabuco de Araújo}. Rio de Janeiro:
Garnier, 3 vols., 1897/1899; e J.~M.~Pereira da Silva. \textit{Memórias
do meu tempo}. Rio de Janeiro: Garnier, 2 vols., 1896. Sobre as
relações entre macro"-política e escravidão, cf.~Tâmis Parron.
\textit{A política da escravidão no Império do Brasil, 1826--1865}.
Fapesp. Relatório parcial de Mestrado. São Paulo: \textsc{dh/fflch/usp}, 2008.} 

\subsection{O grupo saquarema}

Na década de 1840, esse grupo, agora conhecido como saquarema, continuou
a editar obras e propor leis em favor do trabalho forçado no Brasil.
Dessa época datam os famosos discursos do senador Bernardo Pereira de
Vasconcelos, que afirmava que a África tinha civilizado o Brasil com o
tráfico negreiro; e também a publicação da principal resposta
ideológica brasileira ao \textit{Bill Aberdeen} (1845), o livro
\textit{Inglaterra e Brasil --- tráfego de escravos} (1845), de J. M.
Pereira da Silva, um dos brasileiros radicados em Paris que inauguraram
o Romantismo entre nós com a revista \textit{Niterói} (1836). Todos os
chefes do grupo, rebatizado de Partido Conservador no final da década ---
Bernardo Pereira de Vasconcelos, Paulino José Soares de Souza (visconde
do Uruguai), Honório Hermeto Carneiro Leão (marquês de Paraná), Pedro
de Araújo Lima (visconde de Olinda), Joaquim José Rodrigues Torres
(visconde de Itaboraí) e Eusébio de Queirós ---, mantiveram disciplina
militar na defesa política do tráfico negreiro e da escravidão. 

Na década de 1850, mesmo após o desmonte do infame comércio em face de
manobras bélicas da Inglaterra e o início de dissensos entre esses
caciques, nenhum abandonou a linha de conduta escravista.\footnote{ Após 1853, 
tornou"-se comum distinguir conservadores e saquaremas como
dois grupos dentro do mesmo partido. Pode"-se dizer \textit{grosso
modo} que os primeiros constituíam a ala moderada e transigiam nas
propostas de D.~Pedro \textsc{ii}. Os segundos compunham o núcleo histórico e
ideológico do partido e seguiam os votos de Uruguai, Itaboraí e
Eusébio. A distinção está anotada em Joaquim Nabuco. \textit{Um
estadista do Império}, v.~\textsc{i}, pp.~206 e ss e 405; e foi desenvolvida em
J.~Needell. \textit{The Party of Order}, pp.~174--5 e n.~18, p.~385.} 
Quando políticos mais moços pisaram na tribuna sob o patrocínio da
chefatura do partido nos anos cinquenta e sessenta, acabaram por se
engajar também na manutenção política do cativeiro no Brasil. Tal foi o
caso de Antonio Ferreira Viana, Paulino Jr.~(filho de Uruguai) e,
finalmente, José de Alencar --- a essa falange o historiador de hoje
ainda poderia acrescentar escritores de envergadura que, em algum
momento, se mostraram simpatizantes do \textit{status quo} da
escravidão, como o cônego Joaquim Caetano Fernandes Pinheiro, o
historiador Francisco Adolfo de Varnhagen (visconde de Porto Seguro), o
crítico literário José Feliciano de Castilho, futuro inimigo de
Alencar, e o já mencionado J.~M.~Pereira da Silva.\footnote{ Ver, por exemplo, J. C.~Fernandes Pinheiro, 
``Discussão histórica: o que se deve pensar do systema de colonização seguido pelos portuguezes no Brasil. 
Ponto desenvolvido em sessão de 14 de Julho de 1871, pelo sócio effectivo do Instituto Historico e 
Geographico Brasileiro J. C. Fernandes Pinheiro''; F.~Adolfo de Varnhagen. \textit{Os Índios Bravos 
e o Sr.~Lisboa}. Lima: Imprensa Liberal, 1867; e J.~Feliciano de Castilho, ``Carta introdutória''. 
In T. F.~de Almeida, \textit{O Brazil e a Inglaterra ou o tráfico de africanos}. Rio de Janeiro: Typ.~Perseverança, 
1868, pp.~11--32; veja"-se também o opúsculo elogioso de Varnhagen e crítico da emancipação que escreveu 
Frederico A.~Pereira de Moraes. \textit{Diatribe contra a timonice do Jornal de Timon Maranhense acerca da 
História Geral do Brasil do Sr.~Varnhagen}. Lisboa: Typ.~de José da Costa, 1859.}  

\subsection{A oposição}

As \textit{Novas cartas políticas} foram escritas em um momento
desfavorável para os conservadores e crítico para o sistema escravista
brasileiro. Desde 1863, vinha ocupando os ministérios e elegendo a maioria 
para a Câmara dos Deputados a Liga Progressista, uma facção partidária nova, 
composta por liberais moderados e conservadores dissidentes, que D.~Pedro \textsc{ii} 
apoiava explicitamente para contrabalançar o poder dos saquaremas.
Ao mesmo tempo, a Guerra Civil
nos Estados Unidos levara à  supressão instantânea da escravidão no
único país independente das Américas, afora o Brasil, que ainda a
preservava. Potência política, militar e econômica já no século \textsc{xix}, a
república norte"-americana conseguia compelir outras nações a adotar
uma conduta \mbox{diplomática} de neutralidade respeitosa à  existência da
escravidão no país, defendida como pauta de soberania nacional. O
desenlace da Guerra Civil provocou, assim, uma quebra estrutural na
relação de forças internacionais que vinha garantindo uma sobrevivência
relativamente estável do cativeiro no Brasil e na colônia espanhola de
Cuba, numa época vazada em liberalismo e autonomia
individual.\footnote{ Ainda está por ser feito um estudo sistemático da
influência política do Sul dos Estados Unidos sobre a existência da
escravidão na América. Até mesmo Joaquim Nabuco entreviu na Guerra
Civil o mais importante fator isolado para a abolição no Brasil e nas
ilhas hispânicas. ``Ninguém pode dizer quanto duraria a escravidão, se
os Estados meridionais não procedessem como procederam. Com
separar"-se, condenaram"-na à  morte''. Cf.~Joaquim Nabuco, ``O
centenário de Lincoln'' e ``A influência de Lincoln no mundo''. In
\textit{Discursos e conferências}. Rio de Janeiro: Benjamin
Aguila, 1911, pp.~162--3 e 108--9.} 

A partir de 1864, D.~Pedro \textsc{ii} começou a instar políticos da Liga
Progressista a planejar a emancipação gradual dos escravos. Dois anos
depois, o \textit{Comité Français d'Émancipation} enviou um apelo
subscrito por letrados de nomeada como Augustin Cochin e François
Guizot, pedindo ao Império que deixasse de ser a ``última terra cristã
manchada pela servidão''. O monarca redigiu a resposta emprazando o fim
do cativeiro para a próxima ocasião oportuna. Em 1867, induziu o
gabinete progressista a incluir a pauta na \textit{Fala do Trono} ---
espécie de agenda das propostas básicas do Ministério, lida para as
Câmaras no início do ano legislativo.\footnote{ Ver José Murilo de Carvalho. 
\textit{D.~Pedro \textsc{ii} --- ser ou não ser.} São Paulo: Cia. das Letras, 2007, pp. 130--136.} Foi
nesse contexto que \textit{Erasmo} resolveu publicar a nova série de epístolas políticas. 

\section{Erasmo de Roterdã e José de Alencar}

José de Alencar não escolheu por capricho o pseudônimo Erasmo. Como se
sabe, o célebre renascentista tinha ajudado a divulgar o gênero
retórico"-político chamado espelhos de príncipe, ou \textit{specula
principis}, com a obra \textit{A educação de um príncipe cristão}
(1516), escrita especialmente para o cristianíssimo e futuro imperador
Carlos \textsc{v} (1519--1558), mas também oferecida ao rei inglês Henrique
\textsc{viii} (1509--1547). Codificado do século \textsc{xiii} em diante, esse gênero
compreende obras em que filósofos prescreviam normas morais aos
governantes para a realização de uma administração justa. Globalmente,
tais escritos constituíam a celebração de virtudes ligadas à 
\textit{persona} real, ao comando da casa e da família e à  relação do
suserano com os súditos.
Com frequência, essas obras compunham a imagem virtuosa do príncipe em
exercício, mas também podiam esquadrinhar a educação perfeita do moço
destinado a portar o cetro e a coroa.\footnote{ A obra de 
Erasmo não se destaca por uma eventual singularidade. Sua
relevância provém, antes, do fato de ter sistematizado em estilo digno
de imitação as tópicas principais do gênero. Realmente, até a morte do
autor, em 1536, seu manual teve pelo menos dez reedições impressas,
estimulando a reprodução dessa literatura em publicações semelhantes
por toda a Europa.} Nesse sentido, pode"-se dizer que Alencar optou
pelo pseudônimo porque viu nos escritos de Erasmo uma espécie de
metonímia dos \textit{specula principis} e em sua conduta pessoal o
paradigma da relação entre um letrado conselheiro e um governante.

\section{a história como lei natural}
A peça de abertura das \textit{Novas cartas políticas}, datada em 24
de junho de 1867, traz como matéria discursiva a ameaça de abdicação de
D.~Pedro \textsc{ii} em face das dificuldades nascidas da Guerra do Paraguai.
Trata"-se, contudo, apenas de um pretexto para esboçar uma pequena
teoria da história, sugerir um padrão de dinâmica das crises sociais e
perfilar alguns princípios fundamentais de governo, o que, tudo somado,
dará corpo ao desenvolvimento das cartas subsequentes. De início, o
autor afirma que qualquer fenômeno social já traz, em si mesmo, uma
força que o robustece, leva"-o adiante e, então, o aniquila. Para
tanto, usa a metáfora, corrente no século \textsc{xix}, 
%Jorge: se for citar, dizer onde.
%--- que aparece, por exemplo, em Hegel --- 
do ciclo natural da planta. Nesse modelo, a cadeia
dos eventos históricos se assemelha à  sucessão do ``gérmen que rompe a
semente; efeito que elimina a causa'', num processo inteiramente
submetido a uma ``dura lei, mas natural''. O caso particular que permitia
a generalização teórica era a vida de D.~Pedro \textsc{i}. O monarca teria
inscrito na pedra sua abdicação no justo momento em que fundou o
Império do Brasil --- isso porque a nação, efeito necessário do ato de D.
Pedro, entraria em rota de choque com o velho espírito português do
Imperador, à  medida que aflorasse sua inevitável nacionalidade. Criada
pelo herdeiro da casa de Bragança, tratou de suprimi"-lo em seguida.
Foi o broto que abriu à  força a semente. 

Daí decorre outro princípio. Se as manifestações históricas sofrem um
processo natural de maturação, desenvolvimento e descarte --- como o
curso biológico semente"-planta ---, então elas apresentam uma
finalidade, uma missão mesmo, que não pode ser interrompida antes do
tempo. De volta aos casos particulares, o autor nos ensina que a
abdicação de D.~Pedro \textsc{i}, em 1831, foi um ato louvável, porque o
Imperador já tinha criado o Brasil e, renegando a Coroa, poupou
desgaste com a nação. Subentende"-se que, tivesse o príncipe regente
desistido da herança antes da Independência, seu ato seria reprovável.
Nesse quadro conceitual, os eventos históricos não se apresentam como
portadores de sentido absoluto; o que os torna virtuosos ou reprováveis
é o fato de terem ou não cumprido sua missão. O mesmo acontecimento ---
abrir mão da Coroa, por exemplo --- será objeto de elogio ou de
reproche, conforme a realização de sua finalidade originária. Esse, e
apenas esse, é o prisma moral por que os conselheiros e os governantes
podem e devem julgar homens e instituições na história.

O apanhado teórico sobre a dinâmica das crises sociais é mais sutil,
apreensível apenas na análise do campo semântico dos vocábulos que o
autor emprega. A primeira carta repreende D.~Pedro \textsc{ii} por ter aventado
a abdicação no final de 1866, quando julgou inaceitável a proposta de
alguns estadistas para a suspensão da Guerra do Paraguai. Conforme
Alencar, as palavras da Coroa, quase discretas, constituíram uma ``voz
funesta'' que passou, em gradação crescente, de sussurro para verbo e,
depois, para eco perturbador no seio das instituições políticas do
país. Sem hesitar, o autor a timbrou de ``prenhe de calamidades''. Como
se percebe, o vocábulo ``prenhe'' designa processo de gestação, que por
sua vez sugere um decurso que, não visível nem perceptível no início,
trará inevitavelmente algo novo ao mundo em algum ponto do futuro; ele
só para se for abortado. Isso significa que apenas uma ação (o aborto)
remedeia uma incidência suscitada por uma voz, um verbo. O pressuposto
do autor, que irá estruturar a argumentação ulterior sobre a
escravidão, está em que uma crise é facilmente deflagrada (basta
mencioná"-la, ventilá"-la), mas apenas custosamente controlada (é
necessária uma execução firme e decisiva). No particular, essas ideias
se aplicarão ao fato de D.~Pedro \textsc{ii} ter respondido ao \textit{Comité
Français} e ter inserido a questão da emancipação na \textit{Fala do
Trono}. São palavras, mas não meras palavras. Detonam processos
revolucionários dificilmente passíveis de interrupção.  

\section{escravidão e progresso humano} 

Pode"-se dizer que a primeira missiva tem por função teórica fornecer o
quadro conceitual adequado para a análise da escravidão, assunto
central da segunda, terceira e quarta peça. No campo da História,
Alencar diz que a instituição, igual às demais manifestações do
passado, carrega uma finalidade imanente a ser atingida: favorecer o
progresso do homem. Segundo o autor, quando o ser humano é colocado no
mundo pela primeira vez para viver no estado natural, irrompe nele um
sentimento intrínseco de aperfeiçoar"-se, que o impele ao domínio do
mundo físico, ao melhoramento material e à  elevação moral. Essa regra ---
expressa no vocábulo \textit{necessidade} --- se prende à  natureza
humana; logo, é imutável e como que está fora da história. Em qualquer
lugar, em qualquer circunstância, em \mbox{qualquer} idade ou era, o homem,
arrebatado pela \textit{necessidade}, busca o aperfeiçoamento. Inserido
nas contingências da história, porém, ele realiza seu inescapável
destino mediante os instrumentos de que dispõe \textit{ad hoc}. É em
tal ponto dessa pequena teoria do desenvolvimento social que entra a escravidão: 

\begin{hedraquote}
No seio da barbaria, o homem, em luta contra a natureza, sente a
necessidade de multiplicar suas forças. O único instrumento ao alcance
é o próprio homem, seu semelhante; apropria"-se dele ou pelo direito
da geração ou pelo direito da conquista. Aí está o gérmen rude e
informe da família, agregado dos fâmulos, \textit{coetus servorum}
[reunião de servos]. O mais antigo documento histórico, o Gênesis, nos
mostra o homem filiando"-se à  família estranha pelo cativeiro
(p.~\pageref{barbaria}). 
\end{hedraquote}

Isolado nos ermos do estado natural, o homem não teria arriscado o
primeiro passo rumo à  vida comunitária e à  cultura senão pela
escravização. No estágio seguinte, as famílias recém"-formadas se unem
para compor as \textit{gentes}, cuja reprodução no tempo e expansão no
espaço seriam tolhidas por conflitos violentos e guerras contínuas que
fariam umas às outras. Entretanto, explica Alencar, a ação benéfica do
cativeiro poupou milhares de vidas. ``Se a escravidão não fosse
inventada'', escreve o nosso Erasmo, ``a marcha da humanidade seria
impossível.'' Para explicá"-lo, o autor lembra que, nas leis antigas, o
vencedor na guerra adquiria o direito de matar seus adversários, mas,
numa espécie de ato benevolente, ele também podia comutar a pena
capital em trabalho forçado perpétuo: 

\begin{hedraquote}
Em princípio, reduzida a pequenas proporções, tribo apenas, é pelo
cativeiro ainda que a sociedade se desenvolve, absorvendo e assimilando
as tribos mais fracas. [\ldots] 
Desde que o interesse próprio de possuir o vencido não coibisse a fúria
do vencedor, ele havia de imolar a vítima. Significara, portanto, a
vitória na Antiguidade uma hecatombe; a conquista de um país, o
extermínio da população indígena (p.~\pageref{hecatombe}). 
\end{hedraquote}

Como se percebe, Alencar introduz agora o segundo princípio de sua
teoria social: o \textit{interesse próprio}. Se a \textit{necessidade}
deflagrara a marcha do progresso, o \textit{interesse próprio} é que a
racionalizava. Com base nesses dois conceitos --- pertencentes às
linguagens modernas da teoria política e da ciência econômica ---, o
autor passa então a decifrar a causa das desigualdades entre os povos.
Graças à  escravidão, afirma ele, o Oriente (a Ásia, mas também a
África) tinha conservado a vida dos prisioneiros de guerra, acumulando
contingente humano para preencher seus vastos territórios, a despeito
do solo estéril e do clima abrasador. Totalmente distinto era o caso da
América. Em que pese a natureza propícia e acolhedora, era quase
despovoada antes dos descobrimentos, pois os índios, amantes que eram
da liberdade, não sofriam o cativeiro. Imolavam os homens capturados em
combate e consumiam, irracionalmente, seu próprio capital humano. 

Em síntese, a primeira missão histórica do cativeiro se traduziu em
retirar o homem do estado bruto da natureza. ``O cativeiro foi o embrião
da sociedade'', conclui Alencar, ``o embrião da família no direito civil;
o embrião do estado no direito público''. Daí a constatação realmente
forte de que, à  experiência de escravizar e ser escravizado, ``não
escapou ainda uma só família humana''. Nesse modelo, o papel histórico
do cativeiro ainda se perpetuaria por longos séculos na história,
sempre para ``reparar uma solução de continuidade [evolutiva] entre os
povos''. Na Antiguidade, as comunidades recém"-aculturadas que atingiam
o apogeu da civilização se tornavam, imediatamente, um foco de
concentração de escravos: 

\begin{hedraquote}
Desde as origens do mundo, o país centro de uma esplêndida civilização 
é, no seu apogeu, um mercado, na sua decadência, um produtor de
escravos. O Oriente abasteceu de cativos a Grécia. Nessa terra augusta
da liberdade, nas ágoras de Atenas, se proveram desse traste os
orgulhosos patrícios de Roma. Por sua vez, o cidadão rei, o
\textit{civis romanus}, foi escravo dos godos e dos hunos (p.~\pageref{origem}).
\end{hedraquote}

Posteriormente, o cristianismo moderou as relações sociais europeias e,
num gradualíssimo processo que durou mil anos, fez o cativeiro
declinar na Idade Média em favor da servidão. Nos tempos modernos, porém, 
o descobrimento da América exigiu um esforço épico, sobre"-humano, dos 
colonizadores para efetuar o cultivo do solo e o transplante de civilização, 
isto é, reencenar o domínio da natureza e o progresso moral que 
a humanidade enfrentara no início dos tempos. Os únicos recursos humanos 
então disponíveis pareciam residir nos nativos americanos e nos negros, mas apenas 
aos últimos cabia a execução da grandiosa obra. Para justificar a travessia dos africanos,
Alencar emprega o argumento da ``degeneração indígena'', que
viajantes europeus como Alexander von Humboldt consolidaram no início
do século \textsc{xix}. De acordo com esse lugar"-comum, os povos indígenas
tinham sofrido um intenso processo de decadência da vida cultural e
material desde que entraram em contato com os brancos. O próprio Karl
von Martius afirmou em ``Como se deve escrever a História do Brasil'',
publicado numa edição de 1845 da \textit{Revista trimensal do
Instituto Histórico e Geográfico Brasileiro} e muito conhecido dos
contemporâneos, que os nativos viviam em ``dissolução moral e civil'', de
forma que ``neles não reconhecemos senão ruínas de povos''.\footnote{ Cf.~reedição 
de ``Como se deve escrever a História do Brasil'' na
\textit{Revista trimensal de História e Geographia ou Jornal do
Instituto Histórico e Geographico Brasileiro}. Rio de Janeiro: Typ.~de
João Ignacio da Silva, 1865, pp.~389--411. O argumento da degenerescência
indígena após o Descobrimento foi analisado em \textsc{gibson}, Charles.
``Indian societies under Spanish rule''. In \textit{The Cambridge History of
Latin América}, vol.~\textsc{ii}. Cambridge: Press Syndicate of the University of
Cambridge, 1984, pp.~381--422.} Na frase de Alencar, que ecoa a
sentença ``a África civiliza a América'', de Bernardo Pereira de
Vasconcelos, um dos líderes do Partido Conservador, o argumento aparece
intimamente articulado com a defesa do cativeiro negro: 

\begin{hedraquote}
Se a raça americana suportasse a escravidão, o tráfico não passara de
acidente e efêmero. Mas, por uma lei misteriosa, essa grande família
humana estava fatalmente condenada a desaparecer da face da terra, e
não havia para encher esse vácuo senão a raça africana. [\ldots]
Sem a escravidão africana e o tráfico que a realizou, a América seria
ainda hoje um vasto deserto. A maior revolução do universo depois do
dilúvio fora apenas uma descoberta geográfica, sem imediata
importância. Decerto não existiriam as duas grandes potências do novo
mundo, os Estados Unidos e o Brasil. A brilhante civilização americana,
sucessora da velha civilização europeia, estaria por nascer (p.~\pageref{racaamericana}).
\end{hedraquote}

\section{justificativas para a escravidão}

A essa altura, o leitor se perguntará se o cativeiro já não teria
cumprido seu destino no Brasil. A resposta de Alencar é categórica:
``Nego, senhor, e o nego com a consciência do homem justo que venera a
liberdade''. As causas da proposição ocupam, \textit{grosso modo}, a
terceira e a quarta carta e podem ser classificadas em quatro tipos de
argumentos: o cultural, o político"-social, o econômico e o
identitário. 

\subsection{Cultura e raça}

No plano cultural, o autor escreve que nos tempos
modernos, ao contrário da Antiguidade, os povos bárbaros não conquistam
mais os instruídos. Agora, a civilização vindiça corre os braços às
regiões atrasadas e, por meio da escravização, moraliza seus
habitantes. O cativo antigo, sendo antes vetor de sabedoria acumulada
(do Oriente para a Grécia, da Grécia para Roma, de Roma para os
germânicos), se torna aprendiz bárbaro do progresso humano. ``O escravo
deve ser, então, o homem selvagem que se instrui e moraliza pelo
trabalho. Eu o considero nesse período como o neófito da civilização (p.~\pageref{cultura}).''
O problema era que essa escola da moral ainda não tinha tido tempo de
frutificar no espírito dos africanos, como o operara em outros povos.
``A raça africana tem apenas três séculos e meio de cativeiro. Qual foi
a raça europeia que fez nesse prazo curto a sua educação? (p.~\pageref{tresseculos})''

Revela notar que o autor --- à  maneira da maioria dos estadistas
brasileiros do \textsc{xix} --- não emprega a noção de limitações hereditárias
insuperáveis, argumento do racismo científico, para justificar a
escravidão negra no Brasil. O próprio vocábulo raça, que reincide no
texto, se refere imprecisamente a agrupamentos humanos, definidos ora
conforme a nacionalidade e a geografia, ora conforme a cultura, a cor
de pele e as convicções morais. É esse conceito elástico de raça que
permite conceber a escravidão como instituição aplicável a todos os
povos (e não só aos negros) e, nos tempos modernos, como estágio social
propedêutico, em que o povo dominado se prepara para o exercício
competente da liberdade futura. Veja"-se o trecho que melhor o ilustra: 

\begin{hedraquote}
Se houvesse uma raça infeliz, capaz de permanecer eternamente na
escravidão pelo fato de não consentir a outra em emancipá"-la; então
seria um princípio social aquele absurdo outrora sustentado, da
fatalidade da instituição e desigualdade das castas. Não há porém
contestar, todo povo, toda família humana, acaba cedo ou tarde por
conquistar a liberdade, como a ave implume por devassar o espaço 
(p.~\pageref{implume}). 
\end{hedraquote}

Esse juízo leva o autor a censurar os Estados Unidos, onde as taxas de
manumissão eram baixíssimas e a ideologia senhorial previa o cativeiro
perpétuo dos negros, considerados biologicamente incapazes de
autogoverno. Em uma comparação entre o sistema de trabalho forçado
norte"-americano e o brasileiro, procedimento comum em obras
pró"-escravistas, extrai a conclusão de que as ``atrocidades ali
cometidas contra os escravos'' e ``os prejuízos [preconceitos] selvagens
de raça'' distorciam a função histórica do cativeiro. Infelizmente,
%Jorge: um odioso?
continua ele, o exemplo americano produzia um odioso irreparável que
afetava a imagem da escravidão no Brasil, qualitativamente superior.  

\subsection{Questão político"-social}

No próximo passo, o escritor recorre
ao que se pode chamar de \textit{paternalismo liberal}, isto é, a união conceitual
de uma prática socialmente disseminada (alforria) e de direitos 
constitucionalmente garantidos (cidadania). Essa argumentação é 
introduzida quando Alencar procura rebater o lugar"-comum, muito reiterado 
em discursos antiescravistas, de que a inexpressiva reprodução vegetativa 
dos cativos patenteava a violência do sistema contra a vida. Para fazê-lo, 
recupera a surpreendente expansão numérica da escravidão nos Estados Unidos,
tomados agora como exemplo positivo, durante o século \textsc{xix}. “Em 1790, 
a existência era de 693.397”, diz ele. Em 1820, o número saltou para 1.536.127 
e, em 1850, para 3.178.055. “Onde se viu uma espantosa reprodução da espécie 
humana?” A comparação com a república não era arbitrária. Chefes do Partido 
Conservador e deputados correligionários, como Carneiro Leão (marquês de Paraná), 
Araújo Lima (visconde de Olinda) e Raimundo Ferreira, também já tinham 
evocado o modelo norte"-americano para sugerir a longevidade e a força do 
cativeiro na ausência do tráfico.\footnote{ Cf., por exemplo,
\textit{Anais do Senado}. Brasília: Senado Federal, 1979, 
27 de maio de 1851, p.~388; 28 de maio de 1851, p.~404; 31 de julho de 1854, 
p.~723; \textit{Anais do Parlamento Brasileiro --- Câmara dos Srs. Deputados} [coligidos por
Antonio Pereira Pinto]. Rio de Janeiro: Typographia do Imperial
Instituto Artístico, 1874, 13 de agosto de 1856, pp.~157--160.} 

Nas \textit{Novas cartas políticas}, os mesmos números servem para
aduzir a humanidade da escravidão. Quando o autor, contudo, é levado a reconhecer
que aquelas taxas de reprodução não ocorriam no Brasil, acaba por 
atribuí"-lo não à fúria discricionária do sistema, 
senão, mas, à  frequente concessão de alforrias, expressão da benevolência 
senhorial brasileira (\textit{paternalismo}). Vale notar
que a historiografia, mais ou menos herdeira do discurso abolicionista,
sempre lançou o incremento vegetativo insatisfatório dos plantéis
brasileiros à  conta da superexploração do trabalho. Sem negá"-lo,
deve"-se entrever o papel das alforrias nessa dinâmica como fator
realmente decisivo, a despeito do uso ideológico que se podia tirar
disso, como o faz Alencar: 

\begin{hedraquote}
O primeiro direito da pessoa, a propriedade, o escravo brasileiro não
só o tem, como o exerce. Permite"-lhe o senhor a aquisição do pecúlio,
a exploração das pequenas indústrias ao nível da sua capacidade. Com
esse produto de seu trabalho e economia, rime"-se ele do cativeiro:
emancipa"-se e entra na sociedade. Aí, nenhum prejuízo de casta detrai
seu impulso: um espírito franco e liberal o acolhe e estimula.
[\ldots]
Tranquilizem"-se os filantropos; a escravidão no Brasil não esteriliza
a raça nem a dizima. A redução provém desses escoamentos naturais, que
se operam pela generosidade do senhor, pela liberdade do ventre e
também pela remissão (p.~\pageref{filantropo}). 
\end{hedraquote}

A assertiva ``emancipa"-se e entra na sociedade'' adita ao \textit{paternalismo}
(evento social) uma dimensão essencialmente política. Desde os primeiros debates parlamentares (1827) sobre
o tratado anglo"-brasileiro que previa a supressão do tráfico
negreiro até o \textit{Ensaio crítico} (1861), de A. D.~Pascual,
passando"-se por inúmeros discursos do partido a que Alencar pertencia, os
defensores do sistema escravista recuperavam o artigo \textsc{vi} da
Constituição de 1824.\footnote{ Para um estudo de caso da década de
1830, cf.~Rafael Marquese e Tâmis Parron.~``Azeredo Coutinho, Visconde
de Araruama e a \textit{Memória sobre o comércio dos escravos} de
1838''. In \textit{Revista de História}. Universidade de São Paulo, n.~152, (1º/2005), pp. 99--126; ver também
o artigo de Márcia Berbel e Rafael Marquese. “La esclavitud en las experiencias constitucionales ibéricas, 1810--1824”. In Ivana Frasquet (org.). \textit{Bastillas, cetros y blasones. La independencia en Iberoamérica}. Madrid: Fundación Mapfre-Instituto de Cultura, 2006, pp. 347--374.} O dispositivo concedia ao cativo
alforriado que fosse nascido no Brasil o título de cidadão, permitindo
a participação até mesmo no primeiro grau das eleições, que eram
indiretas. O filho do cativo, por sua vez, desfrutava cidadania plena
como qualquer branco, estando sujeito apenas a restrições censitárias,
cujos critérios não distinguiam cor de pele --- daí as palavras do autor, 
"um espírito franco e liberal o acolhe e estimula".\footnote{ Cf. \textit{Constituição política do Império do Brasil}, 
Título 2º, Artigo 6, Parágrafo \textsc{i}. In Adriano Campanhole e Hilton Lobo Campanhole. 
\textit{Constituições do Brasil}. São Paulo: Atlas, 1989, pp. 749 e ss.} 
Assim, entendia"-se que o sistema escravista brasileiro favorecia
direitos inalienáveis do homem (como a propriedade, o casamento e a alforria) por 
meio de costumes sociais (\textit{paternalismo}) e regulava a inscrição paulatina
dos escravos na cidadania por meio da Constituição (donde o qualificativo \textit{liberal}). Não havia
exemplo de sociedade escravista na América, argumentavam eles, mais
integrada do que a imperial. Qualquer medida que precipitasse o
processo de reinserção dos cativos afro"-descendentes na vida civil provocaria grandes abalos sociais. 

\subsection{Economia}

O autor também dá à  escravidão a missão de garantir o equilíbrio
macroeconômico das riquezas nacionais e dos cofres públicos.
Novamente, ecoa as palavras que Bernardo Pereira de Vasconcelos
proferira na década de 1840. De acordo com o grande chefe conservador,
todos os Estados da América que tinham prescindido do trabalho escravo
sofreram um processo de involução econômica, social e política. José de
Alencar passa em revista essas experiências de emancipação para
advertir contra os males, muito superiores aos benefícios, resultantes
de cada uma delas. Em seguida, compara o ``convulso'' destino das
repúblicas da América hispânica com a estabilidade institucional do
Império do Brasil. Graças à  escravidão, apenas este honrava dívidas
contraídas no exterior e assegurava ao futuro ``uma pátria nobre e digna''. 

\subsection{Identidade nacional}

Finalmente, é possível identificar uma espécie de quarto papel histórico
do cativeiro: contribuir para a formação da identidade nacional
brasileira. Como se sabe, letrados e políticos do século \textsc{xix} --- entre os
quais José Bonifácio e Frederico L.~C.~Burlamaque --- repudiavam o
trabalho forçado por considerá"-lo um entrave à  formação de um povo
único sob o ponto de vista étnico e civil, isto é, homogêneo na
aparência e unido na universalidade do direito. Lamentavam que o
tráfico de africanos aportasse elementos exógenos e que a escravidão
obstasse à  absorção dos negros no tecido social. Por outro lado,
estadistas a favor da instituição recorriam ao diploma de 1824 para
mostrar que ela abastecia o Brasil, mediante a alforria, de novos
cidadãos. Num passo adiante da polêmica, Alencar glorifica a densidade
cultural originária da mistura dos povos, atribuindo sua possibilidade
ao comércio negreiro e sua realização gradual à  escravidão: 

\begin{hedraquote}
Cumpre ser justo e considerar este fato [tráfico e cativeiro] como a
consequência de uma lei providencial da humanidade, o cruzamento das
raças, que lhe restitui parte do primitivo vigor. Bem dizia o ilustre
Humboldt, fazendo o inventário das várias línguas ou famílias
transportadas à  América e confundidas com a indígena: ``Aí está inscrito
o futuro do novo mundo!''

Verdade profética! A próxima civilização do universo será americana como
a atual é europeia. Essa transfusão de todas as famílias humanas no
solo virgem deste continente ficara incompleta se faltasse o sangue
africano (p.~\pageref{profetica}).
\end{hedraquote}


\section{literatura, política e escravidão}
É provável que essa seja a única defesa da escravidão, em toda a América
do século \textsc{xix}, baseada na mistura cultural dos povos; e talvez seja
possível compreendê"-la à  luz da própria produção artística do autor.
Anos antes, Alencar já tinha transfigurado o universo dos escravos em
objeto de imitação artística, nas peças teatrais \textit{O demônio}
\textit{familiar} (1857) e \textit{Mãe} (1860). Nelas chegou a empregar
uma série de expedientes estéticos que imitavam, em chave literária, a
fala dos cativos. A personagem Pedro, por exemplo, de \textit{O demônio
familiar}, comete solecismos (``Sr.~moço não vai visitar ela''),
comunica"-se por onomatopeias (``o chicote estalou: tá, tá, tá; cavalo:
toc, toc; carro, trrr!'') e inventa metáforas rústicas (comendadores
usam, não brasões, mas ``pão"-de"-rala no peito''). Não que o escritor
retratasse o perfil linguístico ao natural; de fato, jornais e charges
satíricas já vinham estabelecendo uma espécie de convenção caricata da
prosódia popular. Alencar deve ter aproveitado o material dessa
codificação para compor falas que obedecessem ao preceito da
coloquialidade, tal como o exigia o gênero dramático em que suas peças
se inseriam, a \textit{Comédia Realista}. Trata"-se de uma imitação
artística por excelência.\footnote{ Cf.~José de Alencar. \textit{O
demônio familiar --- comédia em quatro atos}. Org. por João Roberto
Faria. Campinas: Ed. da Unicamp, 2003, pp.~65, Ato \textsc{i}, cenas \textsc{vi}, \textsc{vii}
e \textsc{viii}. Sobre a linguagem popular na imprensa e na literatura do \textsc{xix},
cf.~Tânia Maria Alkmim, ``A variedade linguística de negros e escravos:
um tópico da história do português no Brasil''. In Rosa Virgínia Mattos
e Silva (org.). \textit{Para a história do português brasileiro}. São
Paulo: Humanitas/Fapesp, 2002, t.~\textsc{ii}, pp.~317--335.}  

Na mesma época, o escritor vinha desenvolvendo um projeto literário
ousado e \textit{sui generis}, cuja plenitude alcançou em
\textit{Iracema} (1865), romance máximo do indianismo no Segundo
Reinado. Tanto no corpo da obra como no posfácio de 1865, Alencar
sustentou a ideia de que a cultura indígena não apenas era fonte inexaurível
de temas para poetas e romancistas, como se observara até ali, mas também dava à  língua 
e à  literatura do Brasil modalidades de elocução que as distinguiam do português falado 
e escrito na Europa. Com efeito, o emprego econômico do pronome reflexivo, a preferência por breves 
orações coordenadas, a eufonia bem dosada dos enunciados e, sobretudo, a reinvenção de 
figuras e tropos na perspectiva do nativo, tudo isso se protrai em \textit{Iracema} como índice
de uma nova linguagem literária localmente condicionada.  

Após a Guerra Civil nos \textsc{eua}, porém, à  medida que o assunto da
escravidão penetrou com força a esfera pública, Alencar estendeu às
falas dos negros sua teoria da literatura nacional. No posfácio à 
segunda edição de \textit{Iracema}, de 1870, escreveu: 

\begin{hedraquote}
Em Portugal, o estrangeiro perdido no meio de uma população condensada
pouca influência exerce sobre os costumes do povo; no Brasil, ao
contrário, estrangeiro é um veículo de novas ideias e um elemento da
civilização nacional.

Os operários da transformação de nossas línguas são esses representantes
de tantas raças, desde a saxônia até a africana, que fazem neste solo
exuberante amálgama do sangue, das tradições e das línguas.\footnote{ Cf.~José de Alencar, 
``Pós"-escrito (à  segunda edição)''. In
\textit{Iracema}. Org.~por Tâmis Parron.~São Paulo: Hedra, 2006, p.~181.} 
\end{hedraquote}

No começo de 1871, ano em que se debateu a lei de emancipação do ventre
escravo, Alencar publicou o romance \textit{O tronco do ipê}, elegendo
uma fazenda fluminense como cenário para a fábula. Entre um episódio e
outro, o narrador justificou a imitação literária da fala dos escravos
--- vale notar que o fez num contexto em que a crítica ao cativeiro
mencionava, entre outras coisas, a corrupção do português provocada por
negros: ``A linguagem dos pretos, como das crianças'', diz ele, ``oferece
uma anomalia muito frequente. É a variação constante da pessoa em que
fala o verbo; passam com extrema facilidade do \textit{ele} ao
\textit{tu}. Se corrigíssemos essa irregularidade, apagaríamos um dos
tons mais vivos e originais dessa frase singela''.\footnote{ Cf.~José de
Alencar. \textit{O tronco do ipê}. In \textit{Obra completa},
v.~\textsc{iv}, p.~647. A símile entre o escravo e a criança é lugar"-comum 
de longuíssima data, em sociedades escravistas. No caso específico dos
textos de Alencar, ela reforça a hipótese de que o cativeiro é uma fase
(como a infância) propedêutica e educativa.} Como se vê, aos poucos
se tornava nítida a elevação da fala do negro à  condição de matéria
poética para conferir à  literatura brasileira singularidade nacional.
Esse é o ponto de chegada de dois projetos originalmente estéticos --- a
imitação da fala dos negros a serviço da \textit{Comédia Realista} e a inovadora 
busca de uma literatura com elocução própria --- que,
gradualmente, também tiveram por propósito a defesa política do sistema
escravista. Nesse sentido, a bela e repetida antítese de José Veríssimo
sobre o romancista --- ``revolucionário em letras, conservador em
política'' --- parece antes uma falsa oposição. No problema do cativeiro,
as letras revolucionárias serviram perfeitamente à  tribuna
conservadora.\footnote{ Cf.~José Veríssimo. \textit{História da
literatura brasileira: de Bento Teixeira (1601) a Machado de Assis
(1908)}. Rio de Janeiro: Francisco Alves, 1916, p.~270. Parafraseando o
crítico, José Honório Rodrigues disse que o escritor era ``vanguarda na
literatura e retaguarda na política''. Cf.~J.~H.~Rodrigues. ``A lei do
Ventre Livre --- primeiro centenário''. In \textit{Carta Mensal} da
Confederação Nacional do Comércio, n.~204, 1972, p.~9. Gilberto Freyre
inverteu o segundo termo da definição de Veríssimo, qualificando
Alencar equivocadamente de ``crítico social''. Revela notar que a
interpretação escravista do Brasil feita por Alencar parece ter
informado o modelo sociológico de Freyre. Cf.~Gilberto Freyre, ``José de
Alencar, renovador das letras e crítico social''. In J. de Alencar.
\textit{O tronco do ipê}. Rio de Janeiro: José Olympio, 1955, pp.
11--32. Para uma análise integrada da literatura e da escravidão que
supera as precedentes, vide Hebe C.~da Silva. \textit{Imagens da
escravidão --- uma leitura de escritos políticos e ficcionais de José de
Alencar}. Dissertação de mestrado. Campinas: Depto. de Teoria e
História Literária/\textsc{iel}/Unicamp, 2004. Cf.~tb. Ricardo M. Rizzo.
\textit{Entre deliberação e hierarquia: uma leitura da teoria política
de José de Alencar (1829--1877)}. Dissertação de mestrado. São Paulo:
Depto. de Ciências Políticas/\textsc{fflch/usp}, 2007.} 

Para o autor das \textit{Novas cartas políticas} --- e, de certa forma,
para os saquaremas, núcleo duro do Partido Conservador ---, o sistema
escravista no Brasil tinha a missão histórica de impulsionar a
economia, garantir a estabilidade social, aperfeiçoar a civilização nos
escravos e, finalmente, enriquecer a identidade nacional. Como toda
manifestação do passado, a instituição só se tornaria imoral se fosse
perpetuada para além da realização de suas finalidades intrínsecas. O
modelo histórico implícito na análise de Alencar é o da transição do
Império Romano para a Idade Média, quando os costumes sociais
substituíram gradualmente o sistema escravista por um outro regime de
trabalho, sem o concurso de leis que o extinguissem de um só golpe. Na
última de suas cartas, o autor ainda sugere que a transição se daria
naturalmente em torno de vinte anos, à  medida que a imigração
preenchesse os vazios do território brasileiro. O leitor deve perceber,
entretanto, que o prazo não passa de um expediente retórico para coibir
a abertura do processo legislativo de emancipação. Afinal, sabe"-se
que, mesmo com o auxílio das leis, ele duraria mais que duas intermináveis décadas. 

\section{Um espelho para D.~Pedro \textsc{ii}}

 O leitor poderá perguntar se D.~Pedro~\textsc{ii} obtinha amparo legal para
interferir tão profundamente no Legislativo e no Executivo,
induzindo"-os à  reforma do sistema escravista. A resposta à  questão não
é nada óbvia nem para o historiador de hoje nem para os agentes da
época. Sabe"-se apenas que a Constituição de 1824 dotava o imperante de
muitas faculdades, como convocar ministérios, dissolver a Câmara dos
Deputados, nomear senador um dos três candidatos mais votados de uma
província etc. Era o conhecido Poder Moderador. No entanto, não estava
claro se D.~Pedro~\textsc{ii} podia ser condenado na imprensa e no Parlamento
por alguma decisão tida por equivocada. Todos os estadistas
reconheciam, por exemplo, que um ministro inepto podia ser moral e
legalmente julgado em processos que, teoricamente, podiam alijá-lo do
poder. Mas, e o Imperador? Podia"-se incriminá-lo por via legal? Não, a
Carta proibia isso, o que significaria deposição. Era possível, então,
condená-lo apenas no plano moral? Aí é que não havia consenso. 

Entre 1860 e 1862, sugiram três soluções para o problema. A primeira,
formulada por liberais extremados, preconizava a reforma da
Constituição e a abolição pura e simples do Poder Moderador. Outra,
teorizada pelo líder do Partido Progressista, Zacarias de Góis e
Vasconcelos, sugeria que o Parlamento e a imprensa podiam acusar o
Imperador no plano moral, para fazê-lo ver seu próprio erro e evitar
reincidência no equívoco.\footnote{ Ver Silvana Mota Barbosa,
“‘Panfletos vendidos como canela’: anotações em torno do debate
politico nos anos 1860”. In J. Murilo de Carvalho (org.). \textit{Nação
e cidadania no Brasil --- novos horizontes}. Rio de Janeiro: Civilização
Brasileira, 2007, pp. 155--183.} Em resposta a essas duas proposições, o
Partido Conservador sustentou a absoluta inviolabilidade moral do
monarca em todas as suas atribuições. Os líderes da agremiação
reafirmaram categoricamente nas câmaras, em jornais e em livros que a
responsabilidade política do governo devia recair apenas sobre o
Ministério e sobre o Conselho de Estado, pois ambos podiam dialogar com
o Imperador e dissuadi"-lo de opiniões com que não concordassem.  Num
eventual impasse, ministros e conselheiros tinham toda a liberdade para
pedir demissão, retirando o apoio ao monarca. Caso não o fizessem, era
sinal de que concordavam com D.~Pedro~\textsc{ii} e, dessa maneira, deviam
assumir toda a responsabilidade.\footnote{ Vide Visconde do Uruguay.
\textit{Ensaio sobre o direito administrativo.} Rio de Janeiro: Typ.
Nacional, 1862, 2 vols., vol. \textsc{ii}, pp. 95--102. Vide também falas de
Paulino de Souza em \textsc{acd}, 6 de julho de 1860, pp. 61--68; de Saião
Lobato, em \textsc{acd}, 01 de julho de 1861, p. 11; e de José de Alencar, em
\textsc{acd}, 2 de agosto de 1861, p. 38.} 

Contudo, a abertura da crise mundial da escravidão (1865), o domínio da
Liga Progressista no Parlamento (1863--1868) e a Guerra do Paraguai
provocariam uma profunda inflexão na maneira como os pró-homens
conservadores entendiam o Poder Moderador. Pela primeira vez na
história do Império, a cúpula do partido consentiu oficialmente nas
censuras diretas ao monarca. No universo discursivo, percebe"-se isso em
algumas publicações da época, do que a sequência das cartas de Erasmo é
precioso exemplo. A primeira série, de 1865, se limitou a clamar da
Coroa mais poder para próceres conservadores no Senado, sem acusá-la de
autoritarismo.\footnote{ Cf. José de Alencar, “Cartas de Erasmo”, In
\textit{Obra completa}, pp. 1049--1065.}  Já as \textit{Novas
cartas políticas} irrogaram expressamente a D.~Pedro a intromissão
direta, pessoal, abusiva e inepta nos dois assuntos magnos do Império,
a Guerra do Paraguai e a escravidão.\footnote{ Alguns estudos
interpretam as críticas de Alencar às atitudes do monarca como efeito
direto do ranço pessoal, que teria nascido quando o Imperador o
preteriu na disputa por uma vaga no Senado, em 1870. Ignoram, dessa
forma, a crise mundial da escravidão, a relevância do cativeiro para a
macropolítica imperial e a posição ideológica dos saquaremas. Cf., por
exemplo, Bernardo Ricupero. \textit{O romantismo e a ideia de nação no
Brasil (1830--1870)}. São Paulo: Martins Fontes, 2004, p. 187; e Brito
Broca, “O drama político de Alencar”, in José de Alencar. \textit{Obra
completa}, pp. 1044--1045. O lugar"-comum nasceu na própria época, entre
inimigos políticos, e foi divulgado por Joaquim Nabuco na polêmica que
travou com o autor. Vide Afrânio Coutinho. \textit{A polêmica
Alencar-Nabuco}. Rio de Janeiro: Tempo Brasileiro, 1965, pp. 189 e ss.
}

Os saquaremas, é verdade, jamais chegaram a exigir reforma
constitucional para suprimir as atribuições do Poder Moderador, como o
demandavam publicistas radicais. Restringiram"-se a condenar moralmente
o monarca na imprensa e no Parlamento --- por ironia, tratava"-se
justamente da atitude que recomendara Zacarias, o líder do Partido
Progressista tão achincalhado nas \textit{Novas cartas políticas}.
Nesse sentido, ao usar de um gênero como o dos \textit{espelhos de príncipe}, José de Alencar
compôs um quadro normativo de conduta moral, por assim dizer,
extraconstitucional, capaz de destacar ações tidas por desviantes,
abusivas e autoritárias de D.~Pedro~\textsc{ii}, sem recorrer à  hipótese de
reformar a constituição para controlá-lo. Suas cartas fornecem um
eloquente exemplo de crença na força das instituições liberais
(Parlamento e imprensa). Indicam como a escravidão e sua defesa
política operaram dentro do paradigma do liberalismo político do século
\textsc{xix}. 

Talvez não seja mera coincidência que o primeiro golpe de estado que
provocou a abolição de um regime de governo e a revogação de uma
constituição no Brasil tenha ocorrido somente um ano após a Lei Áurea
(1888). “Não tardará o desengano”, prenunciou o autor das \textit{Novas
cartas políticas}, “libais agora as delícias da celebridade: breve
sentireis o travo da falsa glória”. Infelizmente, a profecia parece
correta. A escravidão estava tão intrinsecamente ligada ao liberalismo
da monarquia brasileira, que era difícil, se não impossível, acabar com
uma sem anular o outro. A partir de então, não só os ex"-escravos como
todos os brasileiros precisaram reinventar"-se num longo percurso que
mistura interesses de desenvolvimento nacional, lutas por direitos
humanos e disputas por poder político. E que está longe, muito longe de
acabar.  
