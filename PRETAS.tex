\SVN $Id: PRETAS.tex 8196 2011-02-02 17:57:32Z oliveira $
\begin{resumopage}
\item[José Martiniano de Alencar] (Mecejana, 1829---Rio de Janeiro, 1877), romancista, dramaturgo e político, 
destacou"-se como um dos mais brilhantes homens de letras do Brasil no século \textsc{xix}. Lembrado hoje, sobretudo, 
como o autor de \textit{O Guarani} (1857), \textit{Iracema} (1865) e \textit{Senhora} (1875), Alencar dedicou quase um terço de sua vida 
intelectualmente produtiva à atividade parlamentar, elegendo"-se quatro vezes deputado geral 
e ocupando por três anos o cargo de ministro da Justiça (1868--1870), quando quase se tornou senador. Na década de 
1870, em que o Romantismo, a monarquia e a escravidão foram submetidos a uma intensa revisão crítica no país, 
passou a sofrer poderosos ataques de inimigos na literatura e na política. Após grave acometimento de tuberculose, 
veio a falecer deixando um rico espólio de mais de vinte romances, quase uma dezena de peças teatrais, textos de 
crítica literária, estudos jurídicos, ensaios políticos e artigos de jornal. 

\item[Cartas a favor da escravidão] reúne uma série de sete textos políticos inscritos 
no gênero epistolar que José de Alencar publicou em franca oposição a D.~Pedro~\textsc{ii} sob o título 
\textit{Ao imperador: novas cartas políticas de Erasmo} (1867--1868). \mbox{O propósito} central da obra era a 
defesa política da escravidão brasileira, que vinha sofrendo intensa pressão internacional e doméstica 
após a abolição nos Estados Unidos (1865). Talvez por ter abordado um tema controverso para os padrões 
contemporâneos, as \textit{Novas cartas políticas} foram excluídas das obras completas do autor, que a Editora José Aguilar 
lançou no final da década de 1950 sob idealização de Mário de Alencar e organização geral de M.~Cavalcanti Proença. 
A presente edição procura suprir a lacuna, fornecendo um precioso texto ao público interessado nos atuais 
debates sobre relações raciais no país, bem como aos especialistas de literatura, história, sociologia e 
ciências políticas que estudam José de Alencar, Império do Brasil, escravidão e sistema representativo.
 
\item[Tâmis Parron,] formado em jornalismo e história pela Universidade de São
Paulo (\textsc{usp}), organizou a edição de \textit{Iracema}, de José de Alencar
(Hedra, 2006). Atualmente, finaliza o mestrado \textit{A política da escravidão
no Império do Brasil}, e escreve em co"-autoria com Rafael de Bivar Marquese e
Márcia Berbel um livro de história comparada sobre a defesa política do sistema
escravista na monarquia brasileira e no império espanhol, no âmbito do Projeto
Temático ``Formação do Estado e da Nação: Brasil, c.~1780--1850'' (Fapesp).

\item[Série Escola da Cidade] é resultado de uma parceria com o Seminário de
Cultura e Realidade Contemporânea, da faculdade de arquitetura e urbanismo
Escola da Cidade, e visa a publicação do pensamento brasileiro sobre arte,
história, arquitetura e política. 

\end{resumopage}

