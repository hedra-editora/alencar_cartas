\part{Cartas a favor da escravidão}

\chapter{Primeira Carta}

\noindent\textit{Senhor}\footnote{ A série \textit{Ao imperador: novas cartas políticas de Erasmo}
(Rio de Janeiro: Typ.~de Pinheiro e Cia.) consiste em sete missivas
dirigidas a D.~Pedro~\textsc{ii} entre junho de 1867 e março de 1868. A segunda,
a terceira e a quarta tratam efetivamente da escravidão, assunto
central da obra, enquanto a primeira, a quinta e a sexta, mais curtas,
analisam as atitudes do Imperador à luz da Guerra do Paraguai. A
presente edição recoloca à disposição do leitor todas as peças, à
exceção da sétima (denominada ``Última carta''), um longo discurso moral sobre a política
do Império. Dessa carta, publicada após um grande intervalo de seis
meses, foi selecionado apenas o epílogo, que anuncia o fim da
campanha epistolar de Erasmo.}\smallskip

Não posso mais conter a veemência do sentimento que me assoberba. 

Uma voz funesta, que abala até as entranhas; voz prenhe de calamidades,
percorre neste momento, não já a cidade, mas o Império. 

E fostes vós, senhor, que a lançastes como um anátema ao país? 

Em princípio, era um sussurro apenas que se esgueirava na sombra.
Agora, já a opinião articulou distintamente esse verbo de revolução; o
eco repercutiu no senado brasileiro. 

 Rompeu"-se o véu. 

 Contudo, vacilo. Apesar da incompreensível coação em que
desgraçadamente vos colocastes, não se concebe este estranho
desfalecimento da majestade. 

 Será real que vossos lábios, selados sempre pela reserva e prudência,
se abriram para soltar a palavra fatal? É possível que súbita
alucinação desvaire a tal ponto um espírito sólido e
reto?\footnote{ Referência à ameaça de abdicação de D.~Pedro~\textsc{ii}, em fins de 1866. Depois da pior
derrota dos exércitos aliados (Brasil, Argentina e Uruguai) na Guerra
do Paraguai, em setembro daquele ano, políticos argentinos e
brasileiros aventaram selar a paz com a república vizinha. Na ocasião,
o Imperador insinuou que preferia renunciar o trono a suspender a
guerra, pois se tratava de ``uma paz que nossa honra não permite''.
Confira Francisco Doratioto. \textit{Maldita Guerra --- nova história da
guerra do Paraguai}. São Paulo: Cia. das Letras, 2002, pp.~248--252.
As \textit{Novas cartas políticas} censuram D.~Pedro~\textsc{ii} por confundir
sua honra pessoal com a honra nacional, que não seria maculada por um armistício.}
 
Não creio, não posso, não devo crer. 

Recebendo a nova incrível, a população ficou atônita. Voz nenhuma
elevou"-se até o trono para exprimir"-lhe o justo e profundo
ressentimento do povo brasileiro: o espanto lhe embargara a fala.
Porém, que magnitude de eloquência nessa privação da palavra!
\textit{Quanta magna est inania
verba},\footnote{ Sentença provavelmente extraída de
 \textit{Cours de littérature française}, do crítico e escritor Abel"-François Villemain, 
que atribuiu ao orador romano a expressão \textit{Quam magna et inania verborum} 
(``tão grande e vazia de palavras''). Assim teria Cícero definido Roma no início dos tempos, 
grande nos feitos, vazia na arte oratória. Cf.~Villemain. \textit{Cours de littérature
française} --- \textit{tableau de la littérature au \textsc{xviii}\textsuperscript{e} siècle}.
Paris: Didier Libraire"-Editeur, 1851, v.~\textsc{iv}, p.~37.} 
exclamou Cícero, observando o tumultuoso estupor do povo romano. 

 Escutai, senhor, o intenso respiro da nação: escutai"-o antes que venha o estertor. 

 Rara vez, e só em circunstâncias muito especiais, pode a abdicação
tornar"-se um ato de civismo admirável. D.~Pedro~\textsc{i}, vosso augusto pai,
logrou um lance destes, que o consagrou herói da paz e da
liberdade.\footnote{ Alusão à abdicação de D.~Pedro~\textsc{i}, em 07 de abril de 1831.}
 

 Sua missão estava concluída, havia fundado a monarquia brasileira e
criado um povo. A Providência, que o suscitara para a realização desse
grande acontecimento, não permitiu que pusesse o remate a sua obra,
educando a nação, filha sua. 

 Era estrangeiro. Esta nacionalidade ardente e impetuosa que exuberava
do nascente império o rechaçou a ele, seu fundador, e mais
vigorosamente que a nenhum outro. Dura lei, mas natural; gérmen que
rompe a semente; efeito que elimina a causa. 

 Quando o ciúme de origem atingiu à sua maior intensidade, D.~Pedro~\textsc{i},
português de nascimento, deixou de ser um monarca, para tornar"-se um
obstáculo, uma anomalia. A mais veemente das paixões populares, o
patriotismo, sublevou"-se contra o princípio estrangeiro encarnado na sua pessoa.

 Reconhecer a fatalidade da revolução, render justiça aos sentimentos
naturais, embora exagerados, de um povo e submeter"-se singela e
nobremente, sem pesar como sem ostentação, aos desígnios da
Providência: são atos de heroísmo e dignidade que a posteridade aplaude. 

Esta situação não é a do Sr.~D.~Pedro~\textsc{ii}, felizmente para o Brasil.
Americano, como seu povo, com ele nascido neste solo abençoado,
cresceram ambos ao influxo das mesmas crenças e das mesmas ideias. Não
existe, pois, neste reinado o gérmen das invencíveis repulsões que
operam em divórcio entre o monarca e a nação. 

Em tais condições, longe de ser um ato meritório e uma sublime virtude,
a abdicação transforma"-se em crime de lesa"-nação. É um grande
perjúrio pelo qual respondem os reis ante Deus no tribunal augusto da posteridade. 

Esta linguagem será nimiamente\footnote{\textit{Nímio}: excessivo.}
severa, e talvez imprópria de um súdito que se dirige ao soberano. Mas,
senhor, quando o monarca chega a falir daquela majestade inviolável de
que o revestiu a vontade nacional, o cidadão, agravado no seu direito,
oprimido em suas crenças, é um remorso vivo, que se ergue perante a régia consciência.

\setcounter{@sectionNumCenter}{1}

\sectionitem 
%\section{II}

Penetremos, senhor, nos seios de vossa alma; não há nela, estou certo,
coisa que se tema de afrontar a publicidade. Meditemos ambos com		
serenidade as ideias que porventura levaram vosso espírito reto a este
desvio incompreensível. 

É acaso a guerra, e seu desfecho incerto, o motivo da vossa deplorável
intenção? 

Figuro uma conjectura. 

O pensamento inicial da política externa que nos arremessou de chofre à
campanha de Montevidéu e logo após, fatalmente, à luta porfiada contra
o Paraguai; o gérmen desta vasta complicação que envolve o país foi
por vós lançado na marcha do governo. 

Não basta. Depois de encetadas as operações militares, quando a guerra
se patenteou às vistas menos entendidas em toda a enormidade do
sacrifício; a vós unicamente se deve a temeridade com que nos
precipitamos, sem refletir, em uma situação irremissível; dilema cruel
entre a ruína e a vergonha. 

Em uma palavra, fostes o princípio e sois a alma da guerra. Vosso
pensamento a inspirou; vossa convicção a alimenta; as forças vivas de
vossa personalidade, todas estão concentradas nessa aspiração grande,
imensa, única, da vitória: e a vitória significa
Humaitá\footnote{ Principal fortaleza do Paraguai, cuja destruição a Tríplice Aliança
(Brasil, Argentina e Uruguai) acertara em protocolo de 1865.}
 arrasado, Lopes\footnote{ Francisco Solano Lopes, presidente da República do Paraguai
(1862--1870). Nos termos do tratado da Tríplice Aliança, de 1º de maio
de 1865, sua deposição era requisito prévio para a paz entre os
beligerantes.} deposto, franca a navegação ribeirinha. 

Admito todas estas suposições, que vos apresentam como inteiramente
identificado com a guerra. Que razão maior resulta, porém, desse
concurso de circunstâncias, para converter o diadema estrelado de que a
nação brasileira cingiu vossa fronte em coroa de espinhos? 

Julgo compreendê"-la.

As reservas da paz e também os recursos ordinários estão há muito
esgotados pelas despesas exorbitantes. A população, não afeita às lides
guerreiras, se esquivará porventura de fornecer novos e maiores
subsídios de sangue; especialmente para uma luta avara das glórias e
nobres entusiasmos, que somente compensam estes sacrifícios cruentos. 

É possível, portanto, que em um momento de cansaço e prostração, o
império exausto, não da seiva, que é opulenta, mas das forças, que se
relaxam; é possível que deseje pôr um termo à luta e assim o ordene. 

Semelhante possibilidade não há brasileiro que a não repila com
veemência, quando entra no seu coração e tempera"-se ao calor de um
santo patriotismo. Mas também raro cidadão cordato alonga os olhos
pelos foscos horizontes desta guerra desastrosa que não sinta
escurecer"-lhe a vista e vacilar o espírito. 

Então, esmorecido por esta vertigem, o mais heroico e brioso sente o
horror do vácuo. Nada espera, nada pode. Sua razão, perturbada pela
imensidade da crise, se recusa ao trabalho da meditação. Ele sente,
enfim, que nenhum homem tem o direito de arrastar sua mãe pátria à
ruína, para vã satisfação de seus brios revoltados.

Vozes já se ouviram neste sentido. São o balbuciar da opinião, infantil
ainda, para exprimir a vontade nacional. Olhos de longo alcance se
dilataram pelo futuro e volveram espavoridos de sua medonha vacuidade.
Daí as manifestações tímidas pela paz, insinuadas a espaços no espírito público. 

Assegura"-se que esta perspectiva de um desfecho à luta, antes de
realizados vossos nobres desígnios, vos sobressalta. Vedes nessa paz
não consagrada pela vitória esplêndida uma falência da honra nacional,
página maculada para a história brasileira. Repelis, portanto, a
solidariedade deste ato; não quereis rubricar com o vosso nome o que
julgais seria o triste documento de nossa vergonha. 

\sectionitem

Estes sentimentos, cuja exaltação não discuto agora, são próprios de um
caráter nobre e generoso. Mas, senhor, esquecestes uma coisa que deve
sempre estar presente e viva na consciência dos reis.\footnote{ Os próximos parágrafos 
desenvolvem algumas tópicas dos tratados prescritivos \textit{specula principis}, com ênfase na 
sujeição da vontade pessoal do monarca aos desígnios impessoais do Estado. 
Vide Introdução.} 

Vós, monarca, cingido pelo esplendor da majestade, vós, o primeiro no
estado, não tendes o direito que reside no ínfimo dos cidadãos, no
mísero proletário, como no vagabundo coberto de andrajos. Não sois uma
pessoa; não tendes uma individualidade; não há sob o manto imperial que
vos cobre o \textit{eu} livre e independente. 

A nação que vos fez inviolável e sagrado vos privou da personalidade. O
coração é para os reis um deus"-lar, que preside a vida doméstica e
ilumina as doces alegrias de família. Desde que o monarca sai deste
santuário, anula"-se o homem nele, e fica somente o representante da
soberania nacional. 

Vossa honra é a da nação como ela a sentir; vossa dignidade a do império
brasileiro. Quando o povo entenda que chegou o momento de acabar a
guerra e exprima seu voto pelos meios constitucionais, haveis de pensar
do mesmo modo, senão como homem, infalivelmente como soberano.

Em vós está encarnado e vivo o grande \textit{eu} nacional. Imagem da
soberania brasileira, todos os sentimentos da nação devem
necessariamente refletir"-se aí. 

Não há nas questões externas do país duas honras a vingar, a honra do
império e a honra do imperador. O que pleiteamos nos campos do Paraguai
não é a vossa glória nem o nome vosso; mas sim o nome e a glória do
Brasil. A ele, pois, a ele somente e a ninguém mais compete resolver em
última instância esta questão da própria dignidade. 

Este que vos fala, obscuro cidadão, pudera, caso o povo brasileiro
aceitasse a paz indecorosa, repelir a cumplicidade do ato, exprobrar à
pátria semelhante fraqueza e até mesmo deserdar"-se dela, se para
tanto não lhe falecesse o ânimo. Mas eu, senhor, na esfera de minha
humildade, sou rei de mim mesmo; e o monarca, no fastígio do poder, é o
súdito de grandes deveres: por isso mesmo que é o depositário de altas prerrogativas.

O pacto fundamental, jurado entre um povo e uma dinastia, vínculo
consagrado pela religião e pela honra, não se rompe assim bruscamente e
a capricho de uma vontade. Nascem deste ato solene direitos e
obrigações mútuas para a nação e o soberano. O trono não é somente um
berço feliz, é um túmulo também.

Se, por qualquer divergência na política, o soberano tivesse o direito
de resignar a coroa, também a nação que elegeu a sua dinastia pudera ao
menor desgosto cassar a delegação da soberania ao seu perpétuo
representante. Tornar"-se"-ia, portanto, o pacto fundamental, a carta
da qual deriva o império da lei, o mais arbitrário e caprichoso dos atos humanos.

Debalde o revestiram de tantas solenidades e o consagraram pelo sufrágio
nacional, se bastasse o capricho de uma vontade para o aniquilar. Pois
o direito que não tem o menor empregado de abandonar o respectivo cargo
sem receber sua escusa, havia de ser tolerado no magistrado supremo da
nação, naquele que faltaria não só a todos os ramos da administração,
mas a todos os poderes e a todos os direitos?

Senhor, sois o primeiro cidadão brasileiro; o primeiro não tanto pela
supremacia, como pela grandeza do sacrifício. A melhor definição desse
título, que herdastes, de imperador, vosso augusto pai a escreveu logo
após na constituição. Jurastes ser o defensor perpétuo do Brasil, não
somente nos tempos felizes, na monção das glórias e prosperidades, mas
sobretudo no dia da desgraça. O maior e mais onerado dos servidores do
estado, para vós não há prazo, nem repouso.

Qualquer que seja o desfecho da guerra, não tendes o direito de separar
vossa dignidade da causa nacional. Um rei que por sua desgraça
praticasse ato semelhante faria a sua pátria a maior afronta,
jogando"-lhe com a coroa às faces. E haverá algum tão isento de pecha,
a ponto de supor"-se maculado pelo fato de continuar no trono do país
que desistisse de uma guerra desastrada? 

Se existira este monarca sempre solícito pela honra nacional, sempre
suscetível pela dignidade do nome brasileiro, esse mesmo não teria em
caso algum o direito de abandonar na humilhação a pátria decaída, que
sua grande alma bastara para reabilitar. Seria falta de generosidade,
embora justificada pelo rigor de uma consciência austera.

\sectionitem

A honra das nações, como a honra dos indivíduos, não está sujeita aos
acidentes da ordem física. Estes podem influir no resultado de uma
empresa, na realização de uma ideia; mas não modificam a intenção. A
honra é um sentimento, um princípio; e não, como pensam muitos, um
sucesso ou mera casualidade. 

Desdoura"-se a nação que sofre impassível as afrontas a sua dignidade,
mas não aquela que se levanta, como o homem de bem, para repelir o
insulto e defender seus brios. Não importa para a consciência a
vitória; ainda sucumbindo, um povo que o amor nacional inflama é uma
coisa respeitável e santa. 

Quando a nação ofendida tem grande superioridade de recursos em relação
ao outro beligerante, deve por certo mortificá"-la em extremo a
dificuldade da vitória. Mas, se ela empregou os maiores esforços em
sobrepujar a resistência; se deu provas de abnegação e heroísmo na
reparação de sua honra ofendida; não fica desonrada curvando"-se ante a impossibilidade. 

Neste caso estaria o Brasil. O que um povo generoso, possuído de nobre
estímulo e cheio de valor pode fazer, o nosso o tem feito, senhor; e
não obstante os agravos recebidos de seu governo. O sentimento da
nacionalidade brasileira manifestou"-se com arrojos de indignação e
heroísmo, que admiraram as nações de Europa e América.

O Brasil se improvisou guerreiro em poucos meses. O rude operário com
uma constância surpreendentemente se fazia soldado no dia do juramento
e veterano no primeiro combate. O governo chegou a assustar"-se dessa
afluência de bravos que, ao reclamo de honra, corriam pressurosos a
vingar a pátria; e estagnou"-lhe o curso, embora depois se arrependesse. 

Não será, pois, um acontecimento qualquer, por mais cruel ao nosso
orgulho nacional, que há de macular o nome deste povo tão suscetível no
ponto de honra, tão impetuoso nos seus brios. Se a Deus aprouvesse
experimentar"-nos com uma terrível provança, deveríamos
resignar"-nos, pois seríamos vencidos por sua mão inexorável, em
castigo de nossos erros. Mas a honra ficaria intacta. 

Longe, pois, de uma persistência obstinada e intolerante para atingir o
resultado que desejais, a prudência aconselha outro procedimento.
Convém declarar de uma vez ao país toda a extensão do sacrifício que a
guerra exige, e ele, que é o único soberano e o único árbitro da
própria dignidade, decidirá conforme a sua consciência de povo nobre e honrado. 

Não receio que ele se degrade. Se deixar"-se abater um momento pelo
terrível concurso de calamidades, que filhos imprudentes concitaram;
tenho fé robusta na reação próxima. O Brasil sabe perdoar as
ingratidões, porém não esquece as afrontas. Neste ponto, confio mais em
nossa pátria do que vós, senhor, que vos lembrastes de a desamparar ao menor desânimo.

Este meio de ir aos poucos arrastando o país além de sua vontade, de
acenar"-lhe agora com um vislumbre de vitória para lhe pedir mais
levas e, logo após, figurar próximo o desfecho, que sempre se remove
para mais longe, essa falácia me parece, além de pouco decente para o
governo, excessivamente perigosa. 

Um dia, pode o país iludido aterrar"-se ante a medonha perspectiva do
futuro e exigir contas severas daqueles que o levaram de olhos vendados
através dos precipícios. E não há nada medonho e funesto como seja a
irritação dos cegos; dos homens como dos povos cegos. O desespero que
gera a impossibilidade de ver a causa de seu mal os impele a desfechar
golpes tremendos. Almas às quais estão cerrados os horizontes se
esforçam por atingir com a fúria o que não podem atingir com a vista; e
é tudo o que os cerca. 

\sectionitem

Na maior expansão do amor que vos consagro, senhor, peço a vossa
meditação neste assunto capital.

À frente de nossas forças estão os mais experimentados e os mais
ilustres dos nossos cabos de terra e mar; a situação estratégica não é
recente, mas bem antiga, para achar"-se convenientemente estudada.
Digam, pois, aqueles generais ao governo, e este comunique ao país a
verdade inteira das previsões relativas à conclusão ou prolongamento da guerra.

Se é impossível a vitória, o que eu recuso acreditar; acabe"-se uma
luta vã de glória e só repleta de misérias e dores. Quanto mais
depressa repararmos as perdas sofridas, mais prontamente arrebataremos
o triunfo que, porventura, nos escape da primeira vez. 

Devemos vencer, porém, como tudo o augura, e faltam apenas os meios
precisos? Abra"-se então o governo francamente com o país; mas com o
país real, aquele cuja seiva alimenta o tesouro e o exército; não com
esse país simulado, do qual são representantes os maiores e acérrimos
inimigos do Brasil.

Esses nada recusam, porque nada lhes custa. Demitiram a pátria, desde
que a transformaram em feira do estrangeiro. É gente que não duvida
vender aos almudes\footnote{ \textit{Almude}: antiga medida de líquidos.}
 o sangue e o suor do povo por alguns
côvados\footnote{ \textit{Côvado}: medida de comprimento.}
 de galão. Nunca o verso do poeta francês teve mais perfeitos originais:
\textit{Pour l'amour du galon prets à toute livrée}.
(Laprade).\footnote{ Variante com lapso tipográfico do verso ``Par amour du galon prêts à
toute livrée'' (por amor ao galão, prontos para qualquer serviço), do
poema ``Pro aris et focis'', escrito pelo poeta e político Victor de
Laprade (1812--1883). Trata"-se de uma obra satírica, que censura a
decadência dos valores morais na sociedade contemporânea francesa,
assim como as \textit{Novas cartas políticas} reprovam a corrupção política e
moral no Segundo Reinado.}

É a combater essa corrupção espantosa que deveis aplicar toda vossa
atividade e dirigir as forças da nação. Não se ilustra pela vitória,
nem pelas conquistas industriais, um povo que a desmoralização
contaminou. A lepra do vício produz no corpo social úlceras hediondas,
que não escondem algumas folhas de louro e uns remendos de púrpura. 

Regenerai a alma da nação; confortai"-a na virtude vacilante. Este sim
é trabalho digno da insistência do soberano; desígnio no qual a
inflexibilidade será, em vez de erro, dever. Recordai, senhor, o que
vos disse outrora nestas palavras já esquecidas:

\begin{quote}
Quando a nação não ouça a paternal admoestação e se aprofunde no vício,
deturpando a virtude, elevando ao redor do trono maus caracteres e
almas prostituídas, então\ldots{} seria a circunstância única em que um rei
teria o direito de abdicar sem fraqueza, abandonando à justiça de Deus
o povo que delinquiu (\textit{Cartas ao Imperador} --- 7\ai). 
\end{quote}

Nada, infelizmente, nada fizestes ainda para arrancar o país ao contágio
funesto da sórdida cobiça e feia imoralidade. Ao contrário, vossa
indiferença a respeito de tudo quanto não concerne a guerra e vossa
obstinação a respeito dela toleram coisas incríveis para quem estima vosso caráter.

Tudo barateais, tudo concedeis; o bom conceito de vosso nome, o pundonor
da pátria, a inviolabilidade da constituição, os princípios vitais da
sociedade; tudo, contanto que venham em troca munições e soldados para
fazer a guerra. Queira Deus que estas levas guerreiras arrancadas do
solo brasileiro por tal meio não reproduzam o exemplo das hostes que o
rei Cadmus\footnote{ Cadmus é o fundador mítico de Tebas. Após matar um dragão consagrado a
Ares, filho de Zeus, plantou o dente da fera em um solo de onde colheu
guerreiros poderosos. Os deuses do Olimpo não lhe perdoaram a falta
contra Ares, provocando sua expulsão de Tebas. Depois, 
transformaram"-no em serpente. A advertência sugere que o procedimento
insólito de expandir o efetivo militar acarretará ao Brasil o mesmo que
a morte do dragão sagrado a Cadmus, independentemente da qualidade dos
soldados.} tirou da terra com os dentes e a torpe
sânie\footnote{ \textit{Sânie}: líquido purulento.} de um dragão.

\sectionitem

Senhor, afogam"-me o coração as efusões do muito que tenho a
dizer"-vos. Não posso de uma vez arrojar essas abundâncias da alma,
acanhada para seu grande patriotismo, fraca para sua dor ante os males da atualidade.

Voltarei a vossa presença. Compelem"-me não só os grandes interesses do
país e do trono, como a valentia dos meus sentimentos.

Para mim, senhor, representais uma fé. É luz que talvez bruxuleia, mas
não se apaga. Velo nesta crença augusta, como no fogo
vestal\footnote{ Fogo ofertado à deusa romana Vesta e
perpetuamente velado por uma virgem consagrada à divindade.} 
de minha religião política. No instante em que se ele extinguir, creio
que ficará na cinza dessa combustão o meu último entusiasmo. E talvez
não haja seiva para reanimá"-lo jamais! 

Não se nutre esta fé na dedicação a vossa pessoa: o que a fortalece é o
zelo pelo grande princípio representado no Sr.~D.~Pedro~\textsc{ii}; o amor à
dinastia, gêmea da pátria, pois nasceram juntas; e, acima de tudo, o
receio de que decepções amargas e sucessivas derramem no país o tédio
pelas melhores instituições.

Sou monarquista, senhor, como sou cristão, com fervor e entusiasmo, do
mais profundo de minha alma.

O tipo do homem livre, do cidadão independente, não é o republicano, que
se apavora com a ideia de uma delegação permanente da soberania.
Visionário político, sonhando um nivelamento repugnante à natureza
tanto moral como física, ele julga"-se humilhado em sua dignidade,
pelo fato de reconhecer um monarca; e não duvida fazer"-se humilde
vassalo da plebe. Entretanto que envergonha"-se de respeitar a
soberania nacional em um indivíduo, a acata na multidão, só porque é multidão.

Dignidade de algarismo que não compreende o homem de convicções. O
monarca vive pela força moral; no povo, reside a força física. Qualquer
destas forças é suscetível de degenerar; em ambas, há o gérmen
pernicioso da tirania, com a diferença, porém, do alcance. Um rei pode
ir até a ferocidade do tigre, não passa além; mas a multidão é uma
voragem, um abismo, um hiato imenso e pavoroso da atrocidade humana. 

Equivale o republicano ao ateu em política. Nega o ente superior com
receio de amesquinhar"-se em face dele.

O verdadeiro cidadão, como eu o compreendo, o homem livre por
excelência, é aquele que se não assombra com o aspecto da majestade. Ao
contrário, regozija"-se vendo uma cabeça no grande corpo social;
tronco degolado se não a tivesse; arlequim, se a tivera postiça. 

A existência de um poder supremo e permanente que porventura abuse da
força e atente contra seus direitos não perturba a serenidade daquela
alma livre; é como o varão justo, que venera a onipotência do Criador,
mas não trepida nunca!

O mais belo exemplo de liberdade na história dos povos é o do cidadão
que acha na rigidez da consciência a força de arrostar com a majestade
e falar ao soberano a linguagem da razão.

Possa minha palavra, ungida pela veneração que vos consagro, calar em
vosso espírito e sufocar aí as injustas prevenções que levanta uma
desconfiança recíproca entre a nação e a coroa. O momento da maior
angústia para a pátria não era a ocasião própria para o soberano fazer
garbo de sua abnegação pelas grandezas; mas sim para que patenteasse
ainda uma vez a abnegação sublime de sua própria pessoa.

Vossos lábios cometeram, pronunciando a palavra, um lapso que a mente
calma de certo já corrigiu. Disseram abdicação, quando a senha do dia
para todos os brasileiros, e para vós primeiro que todos, é dedicação.

\begin{flushright}
\textit{24 de junho\\
Erasmo}
\end{flushright}

\chapter[Segunda Carta]{Segunda Carta\subtitulo{sobre a emancipação}}

\vspace*{-1ex}

\noindent\textit{Senhor}\smallskip

A fama é um oceano para a imaginação do homem.

Às vezes, refrangem na límpida superfície do mar reverberações que
fascinam. Desenha"-se aos olhos deslumbrados um panorama esplêndido.
Nas magnificências da luz, como na pompa das formas, excede as
maravilhas do oriente. 

Mas no foco brilhante dessa reverberação há infalível um espectro.

O espectro solar é a sombra, a treva, a noite, que jaz no âmago da luz,
como o gérmen do mal no seio do bem. O espectro da fama é o luto de uma
virtude que sucumbiu, o fantasma da justiça imolada, a larva do remorso. 

Vosso espírito, senhor, permiti que o diga, foi vítima desta fascinação.
De longe vos sorrio a celebridade. A glória, única ambição legítima e
digna dos reis, aqueceu e inebriou um coração até bem pouco tempo
ainda frio e quase indiferente. 

Correstes após. Mas, deslumbrado pela visão especular, abandonastes a
luz pura, límpida e serena da verdadeira glória, para seguir o falaz
clarão. Proteger, ainda com sacrifício da pátria, os interesses de
outros povos e favonear, mesmo contra o Brasil, as paixões
estrangeiras, tornou"-se desde então a mira única de vossa incansável atividade. \label{incansavel}

São duras de ouvir para um monarca semelhantes palavras; mais cruéis
ainda são de enunciar para um cidadão leal. Vossa alma, porém, carece
destas verdades nuas para se rever nelas como em um espelho que reflita
sua estranha perturbação.

Povo adolescente, se não infante; derramado por um território cuja
vastidão nos oprime; isolados, nestas regiões quase virgens, do centro
da civilização do mundo; qual lustre e fama poderíamos, nós
brasileiros, nós bárbaros, dar a um grande soberano, que o enchesse de
nobre orgulho? 

Nossa gratidão nacional por um reinado justo e sábio, essa, de todas a
oblação\footnote{ \textit{Oblação}: oferta de objeto sagrado a Deus.}
 mais sublime da pátria, comparada com a celebridade europeia, não passa
de mesquinha e insignificante demonstração. Falamos uma língua que o
mundo desdenha, não obstante sua excelência de mais rica e nobre entre
as modernas. Nossa palavra não tem ainda aquele eco formidável do
canhão que repercute longe no coração das nações.

Ouve"-nos apenas, e imperfeitamente, um punhado de dez milhões de almas.

Para a imaginação ávida, a fama estrangeira tem decerto melhor sabor e
outra abundância. O elogio, nalguma dessas línguas que se fizeram
cosmopolitas, contorna o mundo e difunde"-se imediatamente na opinião
universal. Os quatro ventos da imprensa transportam aos confins da
terra o nome em voga, que repetem centos de milhões de indivíduos.
Disputam as artes entre si a primazia de ilustrar a memória do grande
homem e perpetuar as mínimas particularidades de sua vida.

Serão satisfeitos vossos mais caros desejos, senhor, infelizmente para a
pureza de vosso nome.

Já começastes a colher as primícias da celebridade, que tanto cobiçais.
O jornalismo europeu rende neste momento ao imperador do Brasil aquelas
homenagens da admiração pródiga e inexaurível, que saúda a ascensão de
todos os astros da moda. O estrangeiro vos proclama um dos mais sábios
e ilustres dos soberanos. Não há muitos dias leu o país o trecho da
mensagem em que o presidente dos Estados Unidos, aludindo à franquia do
Amazonas, vos considerou entre os primeiros estadistas do
mundo.\footnote{ A franquia da navegação no Amazonas era controversa. 
Decretada em dezembro de 1866, foi bem recebida nos Estados Unidos, 
que a demandavam havia tempos. O trecho, irônico, mostra como os elogios 
internacionais ao Imperador, que supostamente o inebriavam, ocultavam 
interesses de outras potências em desfavor do Brasil.}

Palavras ocas e sonoras, soalhas do pandeiro, que a fama, sedutora
boêmia, tange com requebros lascivos, insultando a castidade do homem
sisudo. Quem pensara que vossa alma sóbria se havia de render à vulgar tentação? 

Não tardará o desengano. Libais agora as delícias da celebridade: breve
sentireis o travo da falsa glória. Há de causar"-vos nojo, então, esta
fútil celebridade que a moda distribui a esmo por quaisquer novidades artísticas. 

Um espírito robusto como o vosso não pode sofrer por muito tempo o jugo
da vaidade. Reconhecereis que um monarca brasileiro, fosse ele o ídolo
de seu povo e o melhor entre todos os reis da terra, havia de viver
como sua pátria no crepúsculo de nossa civilização nascente. 

É a lei providencial de todas as coisas que tem uma aurora e um ocaso.

Há alguns séculos, a origem histórica de França e Inglaterra era coisa
obscura e indiferente: em nossos dias, quem não preza os ilustres
fundadores destas grandes nações! Quando nossa jovem civilização subir
ao apogeu, também projetará sobre o passado, presente agora, um vivo
clarão. É o raio dessa luz que há de iluminar o berço do povo
brasileiro e o reinado dos soberanos virtuosos que o educarem para o bem.

\setcounter{@sectionNumCenter}{1}

\sectionitem

Não existe para vós, senhor, outra fama lícita e pura senão aquela 
póstuma, que é a verdadeira glória.

Já se foi o tempo em que os povos eram instrumento na mão dos reis, que
os empregavam para obter a satisfação de suas paixões e a conquista de
um renome vão. Agora que as nações se fizeram livres e de coisa
maneável se tornaram em vontade soberana, são elas próprias a mais
generosa ambição e a glória excelsa para os monarcas.

Outrora Alexandre, arrojando à Ásia seu pequeno povo e desbaratando"-o
para conquistar um mundo, foi o maior herói da Antiguidade. O rei que
tal coisa empreendesse atualmente de seu próprio impulso perpetraria um
grande crime, sacrificando à sua glória pessoal os destinos de uma nação livre. 

Maior entre os monarcas neste século de liberdade considero eu aquele,
embora modesto e comedido, que possa ler no fundo de sua consciência
íntegra a satisfação de governar um povo feliz. São estes os únicos
heróis de nosso tempo, os grandes conquistadores da paz e da civilização.

Repassai na mente a vossa história, senhor. Durante um reinado de vinte e
sete anos, em sua quase metade bastante agitado, lutando com duas
rebeliões e a efervescência do espírito público;\footnote{ Alusão imprecisa a duas 
das três principais rebeliões políticas suprimidas após a Maioridade (1840), isto é, a 
Farroupilha (1835--1845), as insurreições liberais de São Paulo e Minas Gerais 
(1842) e a Praieira (1848).} 
nunca vosso nome esteve como hoje sujeito à censura e até mesmo à exprobração. Outrora
pululavam alguns torpes escritos que transudavam fel; eram as escórias
de paixões ínfimas. As acusações atualmente se levantam no parlamento e
no alto jornalismo.

Por que razão recrudesce este sintoma justamente quando nos trazem todos
os paquetes as explosões do entusiasmo estrangeiro por vossa pessoa?
Como se explica esse desgosto nacional por aquilo que, ao contrário,
devera orgulhar um povo?

Confrange o espírito público um ressentimento amargo. O país suspeita
que os entusiasmos de além"-mar não são espontâneos e desinteressados;
mas sim obtidos à custa de concessões perigosas. Rasga"-se o manto
auriverde da nacionalidade brasileira, para cobrir com os retalhos a
cobiça do estrangeiro.

São muitos os cortejos que já fez a coroa imperial à opinião europeia e
americana. Reclama sério estudo cada um destes atos, verdadeiros golpes
e bem profundos na integridade da nação brasileira. Um, porém, sobre
todos me provoca neste momento, pelo seu grande alcance no futuro do
país, como pelo grave abalo que produziu na sociedade.

A emancipação é a questão máxima do dia. Vós a descarnastes, senhor,
para arremessá"-la crua e palpitante na tela da discussão, como um
pábulo\footnote{ \textit{Pábulo}: alimento.} às ambições vorazes do poder. Imediatamente o arrebatou essa facção que
se intitula progressista, como os vândalos se diziam emissários celestes: \textit{agi enim se
divino jusso}.\footnote{ A frase alude à Liga Progressista (vide Introdução). 
A oração latina ``Ser conduzido por mandado divino'', \textit{De gubernatio dei}, livro \textsc{vii}, 13, Salviano 
de Marselha (séc.~\textsc{v}), sugere um paralelo entre a Liga e os vândalos: assim como os bárbaros devastaram 
o norte da África dizendo"-se emissários de Deus, o ministério brasileiro provocaria destruição social 
em nome de um princípio superior, a emancipação dos escravos.} 

A propaganda filantrópica, excitando vivas simpatias entre os povos
civilizados, devia ser arma formidável na mão que a soubesse manear com
vigor. Sentindo estiar a aura efêmera e caprichosa que em princípio os
acolhera, os homens da situação conheceram a necessidade de
amparar"-se com a influência estrangeira. Era o meio de
subtraírem"-se à indignação pública, sublevada por seus desatinos.

Não hesitaram, pois; fizeram de uma calamidade ideia política.
Dissecaram uma víscera social para atar a maioria. 

Considerai, senhor, no alcance funesto deste acontecimento, se os
espíritos refletidos vacilassem um instante na resistência, abalados
pelo impulso do coração. Rompidos porventura os diques da opinião, a
revolução se precipitara assolando este mísero país, já tão devastado.
A ninguém é dado prever até onde chegaria a torrente impetuosa. 

Felizmente o espírito são e prudente do povo, arrostando com a
odiosidade dos preconceitos, acudiu pronto em defesa da sociedade
ameaçada por falsa moral. Salutar energia que poupou à nação brasileira
males incalculáveis e ao vosso reinado um epílogo fatal! 

Pesa"-me desvanecer a grata ilusão em que se deleita vossa alma. 

Libertando uma centena de escravos, cujos serviços a nação vos
concedera; distinguindo com um mimo especial o superior de uma ordem
religiosa que emancipou o ventre; estimulando as alforrias por meio de
mercês honoríficas; respondendo às aspirações beneficentes de uma 
sociedade abolicionista de Europa; e finalmente reclamando na fala do
trono o concurso do poder legislativo para essa delicada reforma
social; sem dúvida, julgais ter adquirido os foros de um rei
filantropo.\footnote{ Referência às medidas em favor dos escravos que D.~Pedro~\textsc{ii} 
tomou ou favoreceu após o fim da escravidão nos \textsc{eua} (1865): emancipação dos cativos 
do Estado para servir na Guerra do Paraguai (novembro de 1866); libertação do ventre que a 
Congregação Beneditina promoveu entre suas próprias escravas a partir de 1866; concessão de 
títulos aos senhores que alforriassem escravos aptos para a guerra; resposta positiva ao 
pedido do \textit{Comité Français d'Émancipation} pelo fim do cativeiro no Brasil; 
menção do problema da escravidão na \textit{Fala do Trono} de 1867.} 
 
Grande erro, senhor, prejuízo rasteiro que não devera nunca atingir a 
altura de vosso espírito. Estas doutrinas que vos seduziram, longe de
serem no Brasil e nesta atualidade impulsos generosos de beneficência,
tomam ao revés o caráter de uma conspiração do mal, de uma grande e
terrível impiedade.

A propagação entusiástica de semelhante ideia neste momento lembra a
existência das seitas exterminadoras, que, presas de um cego fanatismo,
buscam o fantasma do bem através do luto e ruína. Quanto pranto e
quantas vidas custa às vezes o título vão por que almejam alguns
indivíduos de benfeitores da humanidade!

Bem o exprimiu o ilustre Chateaubriand na máxima severa com que
estigmatizou essa hipocrisia social: --- ``A filantropia'', disse ele a
propósito do tráfico de africanos, ``é a moeda falsa da
caridade''.\footnote{ Ver F.~R.~de Chateaubriand. \textit{Congrès de Vérone --- 
Guerre d'Espagne. Négociations: colonies espagnoles}. Paris: Delloye, 1838, p.~79. 
Chateaubriand foi agente diplomático francês no Congresso 
de Verona (1822). Defendeu ali interesses coloniais da Espanha, então aliada da França, 
repelindo exigências inglesas para a supressão do tráfico negreiro. Seus escritos 
diplomáticos forneceram precioso aporte ideológico para estadistas brasileiros.}

\sectionitem

Investiguemos, senhor, com a atenção que merece, este problema humanitário.

A escravidão é um fato social, como são ainda o despotismo e a
aristocracia; como já foram a coempção\footnote{ \textit{Coempção}: compra.}
 da mulher, a propriedade do pai sobre os filhos e tantas outras
instituições antigas.

Se o direito, que é a substância do homem e a verdadeira criatura
racional, saísse perfeito e acabado das mãos de Deus, como saiu o ente
animal, não houvera progresso, e o mundo moral fora incompreensível absurdo. 

Não sofre, porém, séria contestação essa verdade comum e
cediça\footnote{ \textit{Cediço}: antigo, notório, sabido de todos.}
 da marcha contínua da lei que dirige a humanidade. 

O direito caminha. Deus, criando"-o sob a forma do homem e pondo a
inteligência ao seu serviço, abandonou"-o à força bruta da matéria. A
luta gigante do espírito contra o poder físico dos elementos, do sopro
divino contra o vigor formidável da natureza irracional, é a
civilização. Cada triunfo que obtém a inteligência importa a solução de
mais um problema social.

Nessa geração contínua das leis, criaturas do direito, a ideia que nasce
tem, como o homem, uma vida sagrada e inviolável. Truncar a existência
do indivíduo animal é um homicídio; suprimir a existência do indivíduo
espiritual é a anarquia. Crime contra a pessoa em um caso; crime contra
a sociedade em outro.

A escravidão caduca, mas ainda não morreu; ainda se prendem a ela graves
interesses de um povo. É quanto basta para merecer o respeito. No tênue
sopro, que de todo não exalou do corpo humano moribundo, persiste a
alma e, portanto, o direito. O mesmo acontece com a instituição:
enquanto a lei não é cadáver, despojo inane de uma ideia morta,
sepultá"-la fora um grande atentado.

A superstição do futuro me parece tão perigosa como a superstição do
passado. Esta junge o homem ao que foi e o deprime; aquela arrebata o
homem ao que é e o precipita. Consiste a verdadeira religião do
progresso na crença do presente, fortalecida pelo respeito às
tradições, desenvolvida pelas aspirações a melhor destino. 

Decorar com o nome pomposo de filantropia o ideal da ciência e lançar o
ódio sobre as instituições vigentes, qualificando seus defensores de
espíritos mesquinhos e retrógrados, é um terrível precedente em matéria
de reforma. Tolerado semelhante fanatismo do progresso, nenhum
princípio social fica isento de ser por ele atacado e mortalmente ferido.

A mesma monarquia, senhor, pode ser varrida para o canto entre o cisco
das ideias estreitas e obsoletas. A liberdade e a propriedade, essas
duas fibras sociais, cairiam desde já em desprezo ante os sonhos do
comunismo. Seria fácil demonstrar que vosso próprio espírito,
filantropo no assunto da escravidão, não passa de rotineiro a respeito de religião.

Choca semelhante arrogância da teoria contra a lei. Ainda mesmo extintas
e derrogadas, as instituições dos povos são coisa santa, digna de toda
veneração. Nenhum utopista, seja ele um gênio, tem o direito de
profaná"-las. A razão social condena uma tal impiedade. 

A escravidão se apresenta hoje, ao nosso espírito, sob um aspecto
repugnante. Esse fato do domínio do homem sobre o homem revolta a
dignidade da criatura racional. Sente"-se ela rebaixada com a
humilhação de seu semelhante. O cativeiro não pesa unicamente sobre um
certo número de indivíduos, mas sobre a humanidade, pois uma porção
dela acha"-se reduzida ao estado de coisa.

Mais bárbaras instituições, porém, do que a escravidão já existiram e
foram respeitadas por nações em virtude não somenos às modernas. Não se
envergonharam elas em tempo algum de terem laborado no progresso do
gênero humano, explorando uma ideia social. Ao contrário, ainda agora
lhes são títulos de glória essas leis enérgicas e robustas que faziam
sua força e serviam de músculo a uma raça pujante. 

Houve jamais tirania comparável ao direito
quirital\footnote{ \textit{Direito quirital}: repertório legal arcaico de Roma. 
Entre suas disposições consideradas tiranas, destacam"-se as que colocavam 
o devedor à discrição absoluta do credor, como a pena capital e o cativeiro perpétuo.} 
dos romanos? Entretanto, foi essa instituição viril que cimentou a
formidável nacionalidade do povo rei e fundou o direito civil moderno. 

Que mais opressivo governo do que o feudalismo? Saiu dele, não obstante,
por uma feliz transformação o modelo da liberdade política, o sistema
representativo. 

É, pois, um sentimento injusto e pouco generoso o gratuito rancor às
instituições que deixaram de existir ou estão expirantes. Toda a lei é
justa, útil, moral, quando realiza um melhoramento na sociedade e
representa uma nova situação, embora imperfeita, da humanidade. 

Neste caso está a escravidão.

É uma forma, rude embora, do direito; uma fase do progresso; um
instrumento da civilização, como foi a conquista, o
mancípio,\footnote{ \textit{Mancípio}: poder quase absoluto do páter"-famílias
sobre agregados e dependentes.} a gleba. Na qualidade de instituição, me parece tão respeitável como a
colonização; porém, muito superior quanto ao serviço que prestou ao
desenvolvimento social.

De feito, na história do progresso representa a escravidão o primeiro
impulso do homem para a vida coletiva, o elo primitivo da comunhão
entre os povos. O cativeiro foi o embrião da sociedade; embrião da
família no direito civil; embrião do estado no direito público.

Hão de parecer"-vos estranhas estas proposições, senhor; talvez que, à
vossa mente prevenida,\footnote{ \textit{Prevenido}: o mesmo que preconceituoso.} 
se apresentem como a glorificação da tirania doméstica. 

Percorrei comigo de um lanço a história da humanidade. 

\sectionitem

No seio da barbaria, o homem, em luta contra a natureza, sente a
necessidade de multiplicar suas forças. O único instrumento ao alcance
é o próprio homem, seu semelhante; apropria"-se dele, ou pelo direito	\label{barbaria}
da geração ou pelo direito da conquista. Aí está o gérmen rude e
informe da família, agregado dos fâmulos, \textit{coetus
servorum}.\footnote{ \textit{Coetus servorum}: reunião dos escravos.} 
O mais antigo documento histórico, o Gênesis, nos mostra o homem
filiando"-se à família estranha pelo cativeiro. 

Mais tarde, a aglomeração das famílias constitui a nação, \textit{gens},
formada dos homens livres, senhores de si mesmos. Em princípio reduzida
a pequenas proporções, tribo apenas, é pelo cativeiro ainda que a
sociedade se desenvolve, absorvendo e assimilando as tribos mais fracas.

Se a escravidão não fosse inventada, a marcha da humanidade seria
impossível, a menos que a necessidade não suprisse esse vínculo por
outro igualmente poderoso. Desde que o interesse próprio de possuir o
vencido não coibisse a fúria do vencedor, ele havia de imolar a vítima.
Significara, portanto, a vitória na Antiguidade uma hecatombe; a \label{hecatombe}
conquista de um país, o extermínio da população indígena. 

As raças americanas, cheias de tamanho vigor, opulentas de seiva,
haurindo a exuberância de uma natureza virgem, estavam não obstante a
extinguir"-se ao tempo da descoberta. Entretanto, no Oriente, num
clima enervador, sob a ação funesta da decadência física e moral, uma
raça caquética e embrutecida pululava com espantosa rapidez. 

Ignoram os filantropos a razão?

A América desconhecia a escravidão. O vencido era um troféu para o
sacrifício. No selvagem amor da liberdade, o americano não impunha, e
menos suportava, o cativeiro. No Oriente, ao contrário, a escravidão se
achava na sua pátria. A guerra era uma indústria; uma aquisição de
braços. O primeiro capital do homem foi o próprio homem.

Todas as vezes que houve necessidade de reparar uma solução de
continuidade entre os povos, a escravidão se desenvolveu novamente,
a fim de preencher sua missão eminentemente social. 

Primitivamente os povos caminharam pela conquista. Hordas bárbaras
rompiam das florestas para o foco da civilização. O homem culto,
vencido fisicamente pelo selvagem, mas reagindo moralmente pela
superioridade do espírito; eis o escravo antigo, mestre, sábio, filósofo. 

Assim, desde as origens do mundo, o país centro de uma esplêndida \label{origem}
civilização é, no seu apogeu, um mercado, na sua decadência, um
produtor de escravos. O Oriente abasteceu de cativos a Grécia. Nessa
terra augusta da liberdade, nas ágoras de Atenas, se proveram desse
traste os orgulhosos patrícios de Roma. Por sua vez, o cidadão rei, o
\textit{civis romanus},\footnote{ \textit{Civis romanus}: cidadão romano.}
 foi escravo dos godos e hunos. 

Modernamente, os povos caminham pela indústria. São os transbordamentos
das grandes nações civilizadas que se escoam para as regiões incultas,
imersas na primitiva ignorância. O escravo deve ser, então, o homem 	\label{cultura}
selvagem que se instrui e moraliza pelo trabalho. Eu o considero nesse
período como o neófito da civilização.

 A salutar influência do cristianismo adoçou a escravidão; e a
organização da sociedade foi operando nela uma transformação lenta que
terminou entre o nono e o décimo século. Entrou aquela antiquíssima
instituição em outra fase, a servidão, que só foi completamente extinta
com a revolução de 1789. 

O escravo deixou de ser coisa, na frase de Catão, ou animal, segundo a
palavra de Varrão; tornou"-se homem, como exigia
Sêneca;\footnote{ Referências a Marcus Cato (234--149 a.C.), Marcus 
Varro (116--27 a.C.), autores de obras sobre agricultura, e
Lucius Seneca (4 a.C.--65 d.C.), filósofo estoico que igualou conceitualmente escravos 
e homens livres, julgando"-os sujeitos aos mesmos caprichos das paixões e da fortuna.} 
mas o homem propriedade, o homem lígio,\footnote{ \textit{Lígio}: na Idade Média, indivíduo 
ligado ao superior, sem autonomia de decisão.}
 adstrito ao solo ou à pessoa do senhor feudal. Metade livre e metade
cativo: uma propriedade vinculada a uma liberdade; eis a imagem perfeita do servo. 

Havia quinhentos anos que se extinguira na Europa a escravidão, quando
no século \textsc{xv} ressurge ela de repente e no seio da civilização. 

Por que razão? 

Os filantropos abolicionistas, enlevados pela utopia, não sabem explicar
este acontecimento. Vendo a escravidão por um prisma odioso,
recusando"-lhe uma ação benéfica no desenvolvimento humano,
obstinam"-se em atribuir exclusivamente às más paixões humanas, à
cobiça e indolência, o efeito de uma causa superior. 

Ressurge a escravidão no século \textsc{xv} suscitada pela mesma indeclinável
necessidade que a tinha criado em princípio e mantido por tantos milênios. 

Na cabeça da Europa, como lhe chama o grande épico
lusitano,\footnote{Luís de Camões, em \textit{Os Lusíadas} (\textsc{iii}, 20).}
 então cérebro do mundo civilizado, gerava"-se o maior acontecimento da
idade moderna, o que lhe serve de data, a descoberta da América. A essa
raça ibérica, semi"-africana, estava reservada a glória de lançar
primeira a mão ao novo mundo e pô"-lo ao alcance do antigo. 

Pois aí, no seio dessa raça, devia renascer a escravidão europeia.
Depois da expulsão dos mouros em 1440, efetuou"-se o resgate de
prisioneiros brancos por negros. Este foi o estímulo e o princípio do
tráfico de africanos, que só devia terminar em nossos dias. 

Não se podia melhor ostentar a lógica da civilização humana.

Àqueles povos, futuros senhores de um mundo, obrigados a roteá"-lo,
eram indispensáveis massas de homens para devassar a imensidade dos
desertos americanos e arrostar a pujança de uma natureza vigorosa.
Estas massas, não as tinham em seu próprio seio, careciam de
buscá"-las: a raça africana era então a mais disponível e apta. 

Se a raça americana suportasse a escravidão, o tráfico não passara de \label{racaamericana}
acidente, e efêmero. Mas, por uma lei misteriosa, essa grande família
humana estava fatalmente condenada a desaparecer da face da terra, e
não havia para encher esse vácuo senão a raça africana. Ao continente
selvagem, o homem selvagem. Se este veio embrutecido pela barbaria, em
compensação trouxe a energia para lutar com uma natureza gigante. 

Também não havia outro meio de transportar aquela raça à América senão o
tráfico. Por conta da consciência individual, correm as atrocidades
cometidas. Não carrega a ideia com a responsabilidade de semelhantes
atos, como não se imputam à religião católica, a sublime religião da
caridade, as carnificinas da inquisição. O tráfico, na sua essência,
era o comércio do homem; a
\textit{mancipatio}\footnote{ \textit{Mancipatio}: na Roma Antiga, contrato verbal 
reservado para a venda de terras, gados e escravos. O paralelo com o tráfico negreiro 
tem por objetivo qualificá-lo de forma rude e antiga de direito, porém legítima.} dos romanos. 

Sem a escravidão africana e o tráfico que a realizou, a América seria
ainda hoje um vasto deserto. A maior revolução do universo, depois do
dilúvio, fora apenas uma descoberta geográfica, sem imediata
importância. Decerto não existiriam as duas grandes potências do novo
mundo, os Estados Unidos e o Brasil. A brilhante civilização americana,
sucessora da velha civilização europeia, estaria por nascer. 

\sectionitem

Não é, senhor, um paradoxo esta minha convicção da influência decisiva
da escravidão africana sobre o progresso da América.

Os fatos a traduzem com uma lucidez admirável.

Renascida a moderna escravidão na península ibérica, pode"-se afirmar
que não medrou sobre o continente europeu. Ao contrário, foi de si
mesma, pela influência dos costumes, como pela natural repulsão das
duas raças, se extinguindo. Não houve necessidade de derrogar a
instituição; ainda a lei permanecia, que já o fato desaparecera completamente. 

Nas possessões ultramarinas, porém, e especialmente na América, o
tráfico de africanos se desenvolveu em vasta e crescente escala. Não só
Espanha e Portugal, já acostumadas com os escravos mouros, como as
outras potências marítimas, Inglaterra, França e Holanda, se foram
prover, no grande mercado da Nigrícia, dos braços necessários às suas colônias. 

Como se explica essa anomalia de povos, repelindo na metrópole uma
instituição que adotam e protegem no regime colonial? Não era natural
que a mesma salutar influência dos costumes e antipatia de origem
atuassem nesses países, a não interpor"-se uma causa poderosa? 

Essa causa era a necessidade, a suprema lei diante da qual cedem todas
as outras; a necessidade, força impulsora do gênero humano.

Na metrópole, os europeus não sofriam a falta do escravo, facilmente
substituído, e com vantagem, na cidade pelo proletário, na agricultura
pelo servo. Para as possessões americanas, porém, o escravo era um
instrumento indispensável. Tentaram supri"-lo com o índio; este
preferiu o extermínio. Quiseram substituir"-lhe o
galé;\footnote{ \textit{Galé}: criminoso condenado a trabalhos forçados.} 
mas, já civilizado, o facínora emancipava"-se da pena no deserto e
fazia"-se aventureiro em vez de lavrador.

Não houve remédio senão vencer a repugnância do contato com a raça bruta
e decaída. Um escritor notável,
Cochin,\footnote{ Agustin Cochin (1823--1872), autor de \textit{L'Abolition de l'Esclavage}, 
que recebeu o principal prêmio da Academia Francesa em 1862. O livro teve grande 
impacto na Europa, nos Estados Unidos e no Império do Brasil.} 
estrênuo abolicionista, não pôde, apesar de suas tendências
filantrópicas, esquivar"-se à verdade da história. Deu testemunho da
missão civilizadora da escravidão moderna, em sua obra recente, quando
escreveu estas palavras: ``Foi ela, foi a raça africana que
realmente colonizou a América.'' (Abolição da escravidão, v.~2, p.~74).

Erram aqueles que atribuem o desenvolvimento do tráfico a simples
condições climatéricas. Se as admiráveis explorações dos descobridores
não bastam para desvanecer esse prejuízo, diariamente se acumulam os
argumentos contra ele. Quem já não observou a impassibilidade com que o
trabalhador português arrosta o sol ardente dos trópicos, no mais rude labor?

Não. Esta família latina, que desdenha a ridícula abusão\footnote{ \textit{Abusão}:
engano, ilusão.} dos materialistas, tinha tanto como a família saxônia força e energia
de sobra para rotear o solo americano. Outras foram as causas da
insuficiência da raça branca em relação à primitiva colonização do novo mundo.

A população da Europa, longe de transbordar, como agora, era pouco
intensa naquele tempo: seu território, embora pequeno, sobejava"-lhe.
Minguados subsídios, portanto, devia prestar às novas descobertas; e
estes mesmos estorvados pela dificuldade e risco das comunicações. Eram
raras as viagens então; a emigração, nula.

Foi esta uma causa; outra, a degradação do trabalho agrícola em toda a
sociedade mal organizada, que vive dos despojos do inimigo ou dos
recursos naturais do solo. A colônia era uma aglomeração de
aventureiros à busca de minas e tesouros. Sonhando riquezas fabulosas,
qualquer europeu, ainda mesmo o degradado, repelia o cabo do
alvião\footnote{ \textit{Alvião}: instrumento de trabalho em que uma das pontas se assemelha 
à enxada e a outra à picareta.} como um instrumento aviltante. 
A lavoura na América parecia uma nova gleba ao homem livre. 

Eis a necessidade implacável que suscitou neste continente o tráfico
africano. Vinha muito a propósito parodiar a palavra célebre de
Aristóteles: ``Se a enxada se movesse por si mesma, era possível
dispensar o escravo.''\footnote{ Paráfrase que adapta às condições brasileiras o seguinte trecho da
\textit{Política} (\textsc{i}, 13), de Aristóteles: ``Ele [o escravo] é em si uma
ferramenta para manejar ferramentas. Pois se cada instrumento pudesse
realizar seu trabalho obedecendo ou antecipando a vontade de outros [\ldots{}]; 
se, da mesma maneira, a lançadeira do tear tecesse sozinha e a
palheta tocasse a lira, os manufatureiros não precisariam de
trabalhadores, nem os senhores precisariam de escravos''.}

Três séculos durante, a África despejou sobre a América a exuberância de
sua população vigorosa. Calcula"-se em cerca de quarenta milhões o
algarismo dessa vasta importação. Nesse mesmo período, a Europa
concorria para a povoação do novo mundo com um décimo apenas da raça negra.

Não vêm de origem suspeita estes dados; são colhidos na obra citada de
um ardente abolicionista. É certo que ele jogou com aqueles algarismos
para demonstrar o
desperecimento\footnote{ \textit{Desperecimento}: esgotamento gradual.}
 da raça africana na América; mas escapou"-lhe a razão lógica e natural
do número reduzido da população negra, apresentado pelas estatísticas
modernas. Em três e meio séculos, o amálgama das raças se havia de
operar em larga proporção, fazendo preponderar a cor branca. Três ou
quatro gerações bastam às vezes, no Brasil, para uma transformação completa. 

É, pois, uma grande inexatidão avançar que a raça africana nem ao menos
prestou para povoar a América. Quem abriu o curso à emigração europeia,
quem fundou a agricultura nestas regiões, senão aquela casta humilde e
laboriosa, que se prestava com docilidade ao serviço como aos prazeres
a\footnote{ Na edição original, lê"-se ``prazeres da ralé''. Trata"-se, naturalmente, de um
erro tipográfico, pois elimina a dualidade conceitual que se procura
estabelecer entre o africano, laborioso, e o aventureiro, que evita o
trabalho.} ralé, vomitada pelos cárceres e
alcouces\footnote{ \textit{Alcouce}: prostíbulo.} das metrópoles? 

Longe de enxergar a diminuição da gente africana pelo odioso prisma de
um precoce desaparecimento, cumpre ser justo e considerar este fato
como a consequência de uma lei providencial da humanidade, o cruzamento
das raças, que lhe restitui parte do primitivo vigor. Bem dizia o
ilustre Humboldt\footnote{ Alexander von Humboldt (1769--1859), naturalista alemão. 
Após explorações científicas nas Américas, publicou a \textit{Voyage aux régions équinoxiales 
du Nouveau Continent}, obra que o tornou mundialmente famoso.} 
fazendo o inventário das varas línguas ou famílias transportadas à
América e confundidas com a indígena: ``Aí está inscrito o futuro do
novo mundo!''\footnote{ Frase que consta do \textit{Essai politique sur l'île de Cuba} (1826,
p.~408), de Alexander von Humboldt, mas que também se encontra na já mencionada obra de Augustin
Cochin, \textit{L'abolition de l'esclavage} (1861, p.~74), de onde foi provavelmente extraída.}

Verdade profética! A próxima civilização do universo será americana como \label{profetica}
a atual é europeia. Essa transfusão de todas as famílias humanas no
solo virgem deste continente ficara incompleta se faltasse o sangue
africano, que no século \textsc{viii} afervorou o progresso da
Europa.\footnote{ Referência à tomada da Península Ibérica pelos islâmicos, efetuada a
partir de 711.} 

Chego à questão da sua atualidade.

Esse elemento importante da civilização americana, que serviu para
criá"-la e a nutriu durante três séculos, já consumou sua obra? É a
escravidão um princípio exausto que produziu todos os seus bons efeitos
e tornou"-se, portanto, um abuso, um luxo de iniquidade e opressão?

Nego, senhor, e o nego com a consciência do homem justo, que venera a
liberdade; com a caridade do cristão, que ama seu semelhante e sofre na
pessoa dele. Afirmo que o bem de ambas, da que domina como da que
serve, e desta principalmente, clama pela manutenção de um princípio
que não representa somente a ordem social e o patrimônio da nação; mas
sobretudo encerra a mais sã doutrina do evangelho.

Espero em outra carta levar esta convicção ao vosso espírito; não
obstante a fatal abstração que o retira da miséria nacional, para
engolfá"-lo nas auras da celebridade.

\begin{flushright}
\textit{15 de julho 1867\\
Erasmo}
\end{flushright}

\chapter[Terceira Carta]{Terceira Carta \subtitulo{sobre a emancipação}}

\noindent\textit{Senhor}

\setcounter{@sectionNumCenter}{0}

\sectionitem

A repulsão e o amálgama das raças humanas são duas leis de fisiologia
social tão poderosas como na física os princípios da impenetrabilidade
e coesão.

Integralmente, raças diversas não podem coabitar o mesmo país, como não
podem corpos estranhos ocupar simultâneos o mesmo espaço. Os
indivíduos, porém, que formam as moléculas das diferentes espécies,
aderem mutuamente e se confundem em nova família do gênero humano. 

Ninguém desconhece, todavia, quanto é lenta essa coesão ou amálgama de
raças. Demanda séculos e séculos semelhante operação etnográfica; e
traz graves abalos à sociedade. A tradição e o caráter, que formam a
originalidade de cada grupo da espécie humana, não se diluem sem
aturado e contínuo esforço.

Desde que por uma necessidade suprema e indeclinável a raça africana
entrou neste continente e compôs em larga escala a sua população;
infalivelmente submeteu"-se à ação desse princípio adesivo, ao qual
não escapou ainda uma só família humana.

Eis um dos resultados benéficos do tráfico. Cumpre não esquecer, quando
se trata desta questão importante, que a raça branca, embora reduzisse
o africano à condição de uma mercadoria, nobilitou"-o não só pelo
contato, como pela transfusão do homem civilizado. A futura civilização
da África está aí, nesse fato em embrião.

Mas, senhor, que força maior sufocou a invencível repulsão das duas
espécies humanas mais repugnantes entre si, a ponto de as concentrar no
mesmo solo durante trezentos e cinquenta anos? 

A escravidão; a aliagem artificial, que supre e prepara o amalgama
natural. Sem a pressão enérgica de uma família sobre a outra, era
impossível que a imigração europeia, tão diminuta nos primeiros tempos,
resistisse à importação africana, dez vezes superior. Acabrunhada pela
magnitude da natureza americana, entre dois inimigos, o negro e o
índio, a colônia sucumbira sem remédio.

Situada assim a questão dentro de seus verdadeiros limites na ciência
social, a conclusão decorre logicamente. Resolve"-se a escravidão pela
absorção de uma raça por outra. Cada movimento coesivo das forças
contrárias é um passo mais para o nivelamento das castas e um impulso
em bem da emancipação.

Chegado o termo fatal, produzido o amálgama, a escravidão cai decrépita
e exânime de si mesma, sem arranco nem convulsão, como o ancião
consumido pela longevidade que se despede da existência adormecendo.
Mas, antes do seu prazo, quem fere mortalmente uma lei, derrama sangue,
como se apunhalara um homem.

A história, grande mestra para os que a estudam com o necessário
critério, confirma todos estes corolários da razão. Nas memórias da
escravidão moderna está registrado o sumário crime dos governos que
guilhotinaram essa instituição, para obedecer à fatuidade de uma
utopia. De uma utopia, sim; pois outro nome não tem essa pretensão de
submeter a humanidade, o direito, a uma craveira matemática.

Porque somos livres agora, nós, filhos de uma raça hoje superior,
havemos de impor a todo o indivíduo, até ao bárbaro, este padrão único
do homem que já tem a consciência de sua personalidade! Não nos
recordamos que os povos nossos progenitores foram também escravos e
adquiriram, nesta escola do trabalho e do sofrimento, a têmpera
necessária para conquistar seu direito e usar dele?

Enlevo dos espíritos filantrópicos! O catolicismo da liberdade, como o
catolicismo da fé, é o último verbo do progresso: união da espécie
humana e sua máxima perfeição. Aspiremos a esse esplêndido apogeu dos
nossos destinos; mas não tenha alguém a ridícula pretensão de o escalar
de um salto antes do tempo. 

Dois fatos muito salientes de abolição contrastam na história da
escravidão moderna: o das colônias inglesas em 1833 e o das colônias
francesas em
1848.\footnote{ A abolição da escravidão no império inglês foi aprovada no Parlamento britânico em \label{nota1}
1833. A lei, que entraria em vigor a partir de agosto de 1834,
estipulava uma fase de transição do regime escravista para o livre,
chamada de aprendizagem, em que os trabalhadores, ex"-escravos, se
submetiam a estrita vigilância. Prevista para durar até 7 anos, a
aprendizagem foi abolida em 1838, antes do prazo final. No império
francês, a abolição imediata foi uma das primeiras medidas do governo
revolucionário que instituiu a república no país em 1848. As duas
experiências foram vistas como desastrosas por políticos escravistas
dos Estados Unidos, do Brasil e da Espanha. As \textit{Novas cartas
políticas} procuram atrelar os efeitos de cada abolição ao caráter
étnico e cultural da classe dominante.}
 
O primeiro se realizou com abalo, mas sem grandes catástrofes. Ao atrito
do frio caráter saxônio, a população negra se tinha limado. O homem do
norte é originalmente industrioso; sua mesma pessoa representa uma
indústria, uma elaboração constante das forças humanas contra as causas
naturais de destruição. Ele disputa a vida ao clima e a nutrição ao gelo. 

Esse cunho vigoroso da materialidade o colono inglês imprimira na sua
escravatura. O negro não era já mero instrumento em sua mão; porém, um
operário ao qual só faltava o estímulo do lucro. Quando realizou"-se a
emancipação, os escravos, se não estavam completamente educados para a
liberdade, possuíam pelo menos os rudimentos industriais que deviam
mais tarde desenvolver"-se com o trabalho independente. A essa
madureza deve"-se o estado próspero da população negra depois da abolição.

Houve dor e sangue, porque amputou"-se um membro vivo da sociedade, uma
instituição útil ainda; porém, a cicatriz não se demorou muito e o
organismo se restabeleceu. A passagem do trabalho escravo para o
trabalho livre se efetuou com a divisão das terras e a vigilância da autoridade. 

Nas colônias francesas, muda a cena; a abolição toma um aspecto triste.

A raça latina é sobretudo artística; a indústria, que para o filho do
norte começa com a infância do progresso, para o filho do sul
representa a virilidade. Outros estímulos, que não o cômodo e o útil,
impelem o caráter ardente dessa família do gênero humano: ela aspira
sobretudo ao belo e ao ideal. Com uma gana tão delicada, não podia
certamente a raça latina polir com rapidez a rude crosta do africano:
este permanecia um instrumento bruto na sua mão. 

Por isso a emancipação, além da desordem econômica e das insurreições,
acarretou a desgraça e ruína da população negra. Ainda não educada para
a liberdade, entregou"-se à indolência, à miséria e à rapina. Com
razão se disse que a abolição da escravidão ali importara a abolição do
trabalho. Ainda agora faltam às colônias francesas os braços que
demanda a agricultura.

Onde estão os que, embora cativos, mantinham essa indústria? Aflitiva
interrogação, a que não atende a filantropia, mas a estatística
responde com fúnebre algarismo. 

\sectionitem

Não há exemplo, senhor, de um país que se animasse a emancipar a raça
africana, sem ter sobre ela uma grande superioridade numérica.

Quebrar o vínculo moral, quando não existe a intensidade necessária para
absorver e sufocar o princípio estranho, seria o suicídio. Nenhum dogma
de moral ou preceito de filantropia ordena semelhante atentado de uma
nação contra sua própria existência. A primeira lei da sociedade, como
a do homem, é a da sua conservação. A sentença ímpia que se ouviu na
Europa ``morram as colônias, mas salve"-se o
princípio''\footnote{ Afirmação atribuída a Maximilien Robespierre (1758--1794), que
proferiu a frase ``morram as colônias, se custam vossa honra, vossa
glória, vossa liberdade'', em um debate de 1791 sobre a extensão da
cidadania aos homens livres negros e mulatos das colônias francesas no
Caribe. Robespierre respondia a um deputado da \textit{Assemblée Constituante} 
que, advertindo contra o perigo da extensão da
cidadania, alegara a possível perda das colônias. Posteriormente, a
frase adquiriu a forma que ocorre nas \textit{Novas cartas políticas} e
o sentido de idealismo utópico socialmente irresponsável.}
 revela que a filantropia tem, como todos os fanatismos, sua ferocidade.
Contudo, a morte da colônia não passava da amputação de um membro.
Haverá no Brasil quem exija para salvar o princípio a morte do império, a sua ruína total?

E será esse brasileiro?\ldots{}

A Inglaterra e a França não emancipariam a população negra de suas
colônias, se não se achassem nas condições de proteger eficazmente ali
a raça branca. A força moral da metrópole e seu poder militar eram
suficientes para prevenir e sufocar a insurreição. Figure"-se qual
fora, depois da abolição, o destino da Jamaica ou da Martinica
abandonada por suas respectivas nações!

Os Estados Unidos, não obstante haverem já estreado de longa data a
emancipação, só a completaram recentemente, quando sua população livre
excedia cerca de oito vezes a escravatura. Segundo o recenseamento de
1860, sobre trinta e um milhões de habitantes, quatro apenas eram
cativos. Nessa proporção o antagonismo de raça se atenua; quando não se
desvaneça pelo respeito natural da pequena minoria, inferior em todo o sentido.

Entretanto, o fato da abolição do trabalho escravo no sul da
confederação, decretado por violenta guerra civil, ainda não se deve
considerar consumado. A miséria e a anarquia apenas começam a
desdobrar"-se naquele país, ontem florescente; ninguém sabe das cenas
de horror que, porventura, serviram de peripécia ao drama sanguinolento. 

O Brasil está muito longe de uma situação favorável como aquela. Sobre
uma população de dez milhões de habitantes, um terço é de cativos,
rezam os cálculos mais restritos. Segundo o relatório da sociedade
abolicionista de Inglaterra, o censo da escravatura no universo, em
1850, dava ao nosso país um algarismo superior ao da União Americana.
Havia ali então 3.178.000 escravos; nós tínhamos 3.250.000. Concedendo
que a população escrava dobre em um período de 50 anos, período longo
para o clima, ela se elevaria hoje a cerca de quatro milhões ou
3.900.000, sem contar a importação dos meses que ainda durou o tráfico
depois de 1850.\footnote{ Ao contrário dos Estados Unidos e de Cuba, o Império do Brasil não
procedeu ao recenseamento regular de sua população, tanto livre como
escrava, até 1872. Estima"-se, entretanto, que havia no país quase
dois milhões de cativos em 1850, ano da supressão definitiva do tráfico
negreiro, e não 3.250.000, cifra que José de Alencar aproveita para
fundamentar o argumento do desequilíbrio demográfico.}
 

É certo que no sul dos Estados Unidos, área da escravatura, esta se
achava em igual proporção; cerca de quatro milhões sobre um total de
dez. Foi por esse motivo que o sul em peso, como um só homem, se
levantou contra a abolição. Foi o norte, com seus treze milhões\footnote{Leia"-se: vinte e um milhões.} de habitantes livres, que exigiu a reforma e a impôs.

Os algarismos são, na frase do escritor especialista, que já referi,
testemunhas impassíveis; relatam a verdade, sem deixar"-se influir da
paixão e interesse. Esse mesmo testemunho imparcial da estatística
invoco eu agora, em favor do império, ameaçado de uma grande calamidade.

Há alguém de boa"-fé que aconselhe a emancipação em um estado cuja
população não tem a capacidade suficiente para
sopitar\footnote{ \textit{Sopitar}: refrear.}
 o elemento subversivo? Não equivalera semelhante desatino à loucura do
homem que lançara fogo ao morteiro, para abafá"-lo com a mão?

Dois indivíduos atentos às suas ocupações, confiados na proteção das
leis, são acaso força bastante para conter a sanha de um inimigo
irritado pela anterior submissão, movido por instintos bárbaros e
exclusivamente preocupado desse desígnio sinistro, que ele supõe seu
direito e considera justa reparação de um agravo? 

Nas dobras desse futuro sombrio, o espírito mais firme se desvaira.
Melhor é distraí"-lo de semelhante perspectiva.

Ainda outro algarismo, que vem depor como testemunha neste processo da
emancipação precoce. Em 1850, a escravatura dos Estados Unidos se
distribuía por 347.525 possuidores. Desse número, apenas 7.929 possuíam
mais de cinquenta escravos; entretanto que os proprietários de um até
dez escravos montavam a 254.268.

No Brasil não se levantou ainda, que eu saiba, qualquer estatística
acerca deste objeto. Pretende"-se legislar sobre o desconhecido,
absurdo semelhante ao de construir no ar, sem base nem apoio. Alguns
fatos, porém, muito salientes, que por si mesmos se manifestam
independentes de investigação, podem fornecer dados para um paralelo,
embora imperfeito.

É incontestável que a máxima parte de nossa escravatura se concentrou,
depois da extinção do tráfico, nas províncias do Rio de Janeiro, São
Paulo, Minas, Rio Grande do Sul, Bahia e Pernambuco. Pode"-se afirmar
que nesta área está atualmente circunscrito esse elemento do trabalho
em nosso país.

A proporção local de um terço deve, pois, dilatar"-se nestas seis
províncias, à medida que se restringe em outras, de onde com o tempo
foi emigrando a escravatura. Com efeito, se em nove províncias,
Amazonas, Pará, Piauí, Ceará, Rio Grande do Norte, Paraíba, S.
Catarina, Goiás e Mato Grosso, cujas informações estatísticas
discriminam a condição, há anos passados um cativo correspondia, termo
médio, a 10 habitantes, atualmente maior deve ser a diferença.
Pode"-se, pois, conjeturar que naquela área onde se condensou o
elemento servil, as duas populações estejam ao menos em justa equação.
A respeito da província de S.~Pedro, já em 1848 a presidência o afirmava.

Estes cálculos assentam nas melhores informações que possuímos sobre a
população do império. Recentemente um trabalho recomendável, publicado
a propósito da exposição universal, elevando a população livre do
Brasil a 11.780.000 habitantes, reduziu a escrava ao mesquinho
algarismo de 1.400.000.\footnote{ Referência ao livro \textit{Imperio do Brazil na Exposição Universal de
1867 em Paris}. Rio de Janeiro: Laemmert, 1867. Segundo o bibliófilo
Inocêncio Francisco Silva, a obra foi redigida por Luiz Pedreira do
Couto Ferraz e José Ildefonso de Souza Ramos. Cf.~Innocencio F. Silva.
\textit{Diccionario Bibliographico Portuguez}. Lisboa: Imprensa
Nacional, 22 vols., v.~\textsc{x}, 1883, pp.~61--62. Vale notar que Couto
Ferraz era pessoalmente próximo do Imperador, enquanto Souza Ramos
integrava o gabinete da Liga Progressista.}
 Só a província do Rio de Janeiro tem aquele número ou quase.

Parece"-me nocivo esse desejo de encobrir a verdade ao estrangeiro.
Podem perguntar"-nos o que fizemos de 3.250.000 escravos que
possuíamos em 1850 e do seu acrescimento natural de um terço ou
1.083.333. Onde estariam os 2.933.333 infelizes, que não alforriamos nem exportamos?

Eis o perigo da simulação; ela apresentaria o cativeiro de nosso país
sob um aspecto bárbaro e deletério; assanharia as iras filantrópicas
dos sábios europeus.

Pesa ainda sobre esta situação grave um fato. A pequena lavoura não se
desenvolveu em nosso país. Circunstâncias peculiares à agricultura
brasileira, exigindo forças consideráveis para o roteio e amanho da
terra, obstaram a exploração do solo por capitais diminutos. Os
principais ramos de nossa produção, aqueles que provêm quase
exclusivamente do braço escravo, saem dos grandes estabelecimentos
rurais, engenhos ou fazendas. 

Nem sequer, portanto, as duas espécies de população se penetram e
intercalam mutuamente, de modo a neutralizar a repulsão instintiva de
cada uma. Na área das seis províncias mencionadas, destacam"-se
aquelas aglomerações de escravos que solvem a continuidade da outra
casta; e formam núcleos poderosos de insurreição, comprimidos
unicamente pelo respeito da instituição.

Rompa"-se este freio, e um sopro bastará para desencadear a guerra
social, de todas as guerras a mais rancorosa e medonha.

Julgais que seja uma glória para vosso reinado, senhor, lançar o império
sobre um vulcão? Ainda quando a Providência, que tem velado sobre os
destinos de nossa pátria, a tirasse incólume de semelhante voragem, nem
por isso fora menos grave a culpa dos promotores da grande calamidade. 

\sectionitem

Determinar os efeitos reais dos atos de abolição que sucederam"-se
desde fins do século passado até estes últimos anos me parece um estudo
importante para a solução do difícil problema da escravidão.

Os primeiros estados que deram o exemplo dessa medida foram Pensilvânia
e Massachusetts em 1780, decretando a emancipação gradual. Mais seis
estados acompanharam aquela iniciativa a pequena distância. Em 1820, o
censo manifestou que a escravidão estava completamente extinta nessa
parte da confederação.

Entretanto, o número dos escravos da União que em 1790, ponto culminante
do período abolicionista, era de 693.397, apresentava naquela data da
extinção, em 1820, o algarismo enorme de 1.536.127. Em trinta anos
tinha a escravatura mais que duplicado, e sob a influência de medidas
repressivas, como a proibição do tráfico e a emancipação. 

O movimento abolicionista estendeu"-se pelas repúblicas americanas:
Buenos Aires em 1816; Colômbia e Chile em 1826; Bolívia em 1826; Peru,
Guatemala e Montevidéu em 1828; México em 1829; Uruguai em 1843;
finalmente a Inglaterra em 1833 e a França em 1848 para suas colônias.
Tantos golpes sucessivos desfechados na escravatura, parece que deviam
reduzi"-la imenso. 

Pois a estatística demonstra o contrário. Nesse período de cinquenta
anos a soma dos cativos foi constantemente crescendo. No princípio
deste século pouco mais havia de dois milhões de escravos em toda a
superfície da América; em 1850 orçava o número por sete milhões e meio,
dos quais o maior quinhão pertencia ao Brasil e aos Estados Unidos! \footnote{ O raciocínio de 
Alencar procede. Na primeira metade do século \textsc{xix}, o número absoluto de escravos negros cresceu na 
América, e isso se deveu apenas a Cuba, ao Brasil e aos Estados Unidos, já que a instituição 
declinara ou desaparecera nos demais países; ressalve"-se, no entanto, que o número real de cativos, em 1850, montava a cerca de 5,7 milhões.}

Explica"-se naturalmente esta aparente anomalia, que tanto acabrunhava
a sociedade abolicionista. A escravidão ainda não estava morta; os
esforços dos diferentes governos para extirpá"-la da América eram
impotentes. Conseguiram apenas deslocar o trabalho servil.

Quando os estados da União decretaram a extinção gradual, a escravidão
refugiou"-se naturalmente nos estados próximos onde era mantida; e ali
se propagou de modo a invadir o território que dela estava isento. Se
o norte da União não tivesse margem por onde escoar"-se a sua
população escrava, talvez que a visse refluir sobre si, como aconteceu
com a Carolina.

O mesmo serviço prestou o sul da União ao México, assim como as colônias
de Espanha, França e Inglaterra às repúblicas vizinhas. Advirta"-se
que nestas, não existindo uma agricultura regular, a escravatura era
insignificante no tempo da abolição. De passagem mencionarei uma
circunstância digna de séria meditação. Todas as repúblicas
abolicionistas foram dilaceradas pela anarquia; enquanto o Brasil se
organizava com uma prudência e circunspecção admirável. 

Não só pela intensidade e volume ganhou a escravidão com aqueles atos de
abolição; mas também no princípio e substância. Moral como
economicamente, a instituição triunfou de seus adversários; sobretudo,
depois dos dois últimos fatos, das colônias inglesas e francesas.

O tráfico anteriormente frouxo, por causa da superabundância de braços,
desenvolveu"-se rapidamente desde 1833; e lançou no Brasil e Cuba
milhões de africanos. Por outro lado, a instituição se enraizou ainda
mais profundamente nos países onde não a atingira o movimento abolicionista.

Nos Estados Unidos não perdurara ela por tantos anos, a despeito da
superioridade industrial e numérica do norte e do fervor da propaganda
abolicionista, se não fosse a preponderância que assumira nos
espíritos, depois das últimas e infelizes tentativas. Também no Brasil
há muito tempo que a obra da emancipação se adiantara, sem a convicção
gerada por aqueles acontecimentos da necessidade indeclinável do braço
africano para a agricultura colonial.

Uma verdade ficou bem averiguada.

Como todas as instituições sociais que têm radicação profunda na
história do mundo e se prendem à natureza humana, a escravidão não se
extingue por ato de poder; e sim pela caducidade moral, pela revolução
lenta e soturna das ideias. É preciso que seque a raiz, para faltar às
ideias a seiva nutritiva.

E de onde principalmente derivava para a escravidão essa linfa e substância?

Bem o sabeis, senhor. Da Europa, e com especialidade de Inglaterra,
França e Alemanha, tão abundantes de filantropos como de consumidores
dos nossos produtos. Não fomos nós, povos americanos, que importamos o
negro da África para derrubar as matas e laborar a terra; mas aqueles
que hoje nos lançam o apodo e o estigma por causa do trabalho escravo.

Sem esse enorme estômago, chamado Europa, que anualmente digere aos
milhões de gêneros coloniais, a escravidão não regurgitaria na América,
nem resistira à repugnância natural dos filhos deste continente. Mas
era preciso alimentar o colosso; e satisfazer o apetite voraz do grande
sibarita.\footnote{ \textit{Sibarita}: indivíduo dado aos prazeres dos sentidos e à indolência.}
 

 O filantropo europeu, entre a fumaça do bom tabaco de Havana e da taça
do excelente café do Brasil, se enleva em suas utopias humanitárias e
arroja contra estes países uma aluvião de injúrias pelo ato de manterem
o trabalho servil. Mas por que não repele o moralista com asco estes
frutos do braço africano?

Em sua teoria, a bebida aromática, a especiaria, o açúcar e o delicioso
tabaco são o sangue e a medula do escravo. Não obstante, ele os
saboreia. Sua filantropia não suporta esse pequeno sacrifício de um
gozo requintado; e, contudo, exige dos países produtores que em
homenagem à utopia arruínem sua indústria e ameacem a sociedade de uma sublevação.

Neles desculpa"-se. É fácil e cômoda a filantropia que se fabrica em
gabinete elegante, longe dos acontecimentos e fora do alcance da
catástrofe porventura suscitada pela imprudente reforma.

Mas não se compreende, senhor, que brasileiros acompanhem a propaganda;
e estejam brandindo o facho em torno da mina.

\sectionitem

A razão social convence os abolicionistas da necessidade de deixar a
instituição da escravatura preencher seu tempo e extinguir"-se
naturalmente pela revolução das ideias.

Mas, refratários à própria consciência, buscam eximir"-se à verdade.
Alegam que, abandonada a si mesma e aos instintos humanos, será eterna;
porque os hábitos de indolência que ela cria na casta dominante e a
ignorância em que vai sepultando a casta servil são novas raízes que a
instituição de dia em dia projeta no solo onde uma vez brotou.

Não se pode caluniar mais cruelmente a humanidade, senhor. Admira que
espíritos possuídos de uma ideia tão degradante da criatura racional se
arroubem em sonhos de um progresso instantâneo. É pedir muito ao ente,
de que se faz tão miserável conceito.

Se houvesse uma raça infeliz, capaz de permanecer eternamente na
escravidão pelo fato de não consentir a outra em emancipá"-la; então
seria um princípio social aquele absurdo outrora sustentado, da
fatalidade da instituição e desigualdade das castas. Não há porém
contestar, todo povo, toda família humana, acaba cedo ou tarde por
conquistar a liberdade, como a ave implume por devassar o espaço. \label{implume}

É a Europa o melhor exemplo dessa verdade a respeito da escravidão
moderna. Não se extinguiu o fato nesse continente antes de ab"-rogada
a lei? Não chegou a instituição ao seu termo fatal, apesar da pretensa
indolência e da ignorância difundida na população?

No Brasil mesmo, a despeito da suprema necessidade que mantém esse mau
regime de trabalho, já penetrou na classe proprietária a convicção da
injustiça absoluta do seu domínio. Um espírito de tolerância e
generosidade, próprio do caráter brasileiro, desde muito que transforma
sensivelmente a instituição. Pode"-se afirmar que não temos já a
verdadeira escravidão, porém um simples usufruto da liberdade, ou
talvez uma locação de serviços contratados implicitamente entre o
senhor e o estado como tutor do incapaz. 

A lei de nosso país considera o escravo como coisa ainda; porém, o
costume, a razão pública, mais poderosa que todas as leis escritas,
pois é a lima que as vai gastando a todas e cinzelando as novas; a
razão pública já elevou o cativo entre nós à condição de homem, embora
interdito e sujeito.

O primeiro direito da pessoa, a propriedade, o escravo brasileiro não só
o tem, como o exerce. Permite"-lhe o senhor a aquisição do pecúlio, a
exploração das pequenas indústrias ao nível de sua capacidade. Com esse
produto de seu trabalho e economia, rime"-se ele do cativeiro:
emancipa"-se e entra na sociedade. Aí, nenhum prejuízo de casta detrai
seu impulso: um espírito franco e liberal o acolhe e estimula.

O mais sagrado dos contratos civis, o matrimônio, também está ao alcance
do escravo em nosso país. Ele forma sua família; o senhor a respeita e
a garante. A moralidade que falta ainda não provém da escravidão, mas
da ignorância peculiar às classes ínfimas. Nesse ponto a lia social,
ingênua ou cativa, se confunde.

Embora todas estas garantias se tenham estabelecido fora da lei, contudo
a opinião, que de dia em dia robustece, as mantém e consolida. Se a
cobiça ou perversidade pesa alguma vez com o rigor do direito sobre um
infeliz, a indignação pública imediatamente corrige o desmando. 

Os atos de caridade praticados frequentemente em nosso país, para
arrancar ao cativeiro vítimas da brutalidade e obstar que se rompa o
vínculo de família por um fracionamento inevitável de propriedade são
brados contra os moralistas, detratores da sociedade brasileira.

Que exprime, que revela essa transformação benéfica da escravidão no
Brasil, especialmente nos últimos quinze anos?

Não estão aí bem patentes os sinais da decrepitude, os indícios do
declínio rápido dessa instituição em nossa pátria? Não lobrigam já nos
longes do horizonte os espíritos de vista larga a alva de uma completa
redenção; luz serena que surge naturalmente e mais propícia do que o
clarão avermelhado de um incêndio? 

A decadência da escravidão é um fato natural, como foi a sua origem e
desenvolvimento. Nenhuma lei a decretou; nenhuma pode derrogá"-la. Se
a abafarem ainda vivaz, bem pode ser que só consigam concentrar"-lhe
as forças para maior reação. 

Não é menos injusta a outra imputação feita à humanidade de que o
cativeiro não lima as raças bárbaras nem lhes infiltra os raios da civilização. 

Uma raça não se educa e instrui como um indivíduo. 

Este é uma partícula destacada que, submetida à ação múltipla de uma
vasta civilização representada pela generalidade dos habitantes,
depressa se lapida. A raça, porém, é massa compacta, que ocupa larga
superfície e opõe ao progresso forte resistência.

Para educar uma raça, são necessárias duas coisas: grande capacidade e
vigor do povo culto para imergir a massa bruta e insinuar"-se por
todos os poros; longo tempo para que se efetue essa operação lenta e difícil.

A raça africana tem apenas três séculos e meio de cativeiro. Qual foi a \label{tresseculos}
raça europeia que fez nesse prazo curto a sua educação? Com idade igual
todas elas jaziam imersas na barbaria; entretanto, para os filhos da
Nigrícia, já raiou a luz, e raiou na terra do cativeiro.

É a verdade. Essa família do gênero humano, em cuja tez combusta a
tradição mais antiga do mundo lê um estigma da maldição divina, e eu
vejo apenas o símbolo da treva moral em que havia de perdurar; essa
família infeliz esteve sempre condenada ao desprezo e ao animalismo,
desde Cam, seu progenitor, até Colombo, que a devia remir descobrindo a
América, sua terra de promissão.\footnote{ O parágrafo se refere ao episódio 
bíblico em que Noé condena um de seus filhos, Cam, ao cativeiro perpétuo: 
``Maldito seja Canaã [um dos filhos de Cam e, por extensão, toda a sua descendência], 
que ele seja o último dos escravos de seus irmãos'' (Gênesis 9). Na Idade Moderna, 
os europeus associaram a descendência de Cam aos povos africanos e sua maldição ao 
cativeiro. No Brasil independente, estadistas evitavam fundar a justificativa da 
escravidão na Bíblia, preferindo alegações seculares como necessidade econômica, paternalismo
senhorial e liberalidade constitucional.}

Haiti, São Domingos, a Libéria são outras tantas balizas dessa nascente
civilização africana bebida no novo mundo, durante a peregrinação. As
colônias europeias, que se fundaram na costa da Nigrícia, não tiveram
outra origem senão o tráfico, umas para o favorecer, outras para o reprimir.

Se algum dia, como é de esperar, a civilização projetar"-se pelo
continente africano adentro, penetrando os povos da raça negra, a
glória desse imenso acontecimento, amargue embora aos filantropos,
caberá exclusivamente à escravidão. Foi ela que preparou os precursores
negros da liberdade africana.

O Brasil, de que mais especialmente devo ocupar"-me, nossa pátria,
senhor, não terá concorrido eficazmente para a civilização da grei
humana que submeteu a seu domínio?

Fora injustiça negá"-lo.

Ainda não éramos um império, mas nascente colônia, e já dávamos ao mundo
exemplos sublimes. Um herói negro inscrevia seu nome glorioso na
história brasileira; seus irmãos o acompanhavam formando esse regimento
invencível que, por mais de século, guardou o nome de Henriques, em
memória do cabo
ilustre.\footnote{ Referência a Henrique Dias (?--1662), colono negro que lutou na 
guerra de resistência dos pernambucanos contra o domínio da República dos Países Baixos 
sobre o norte da América Portuguesa (1630--1654). Recebeu o título de fidalgo e o Hábito 
da Ordem de Cristo, entre outras condecorações. No século \textsc{xix}, além de Alencar, o deputado 
Cunha Matos o mencionou numa sistemática defesa do tráfico negreiro, em 1827; e Joaquim 
Caetano Fernandes Pinheiro, autor de uma justificativa da escravidão de 1871, chegou a 
compor uma pequena biografia encomiástica de Henrique Dias, a quem chamou herói.}
A munificência real e a gratidão pública porfiavam nas honras tributadas a esses bravos.

Desde então, não se enriquecem diariamente as classes mais distintas de
nossa sociedade com os talentos e as virtudes dos homens de cor? Se os
primeiros negros, que em 1440 foram dados em resgate a Portugal,
ficassem nos pátrios areais, não contaria a raça africana entre seus
descendentes cidadãos ilustrados, porém só magotes de brutos, como os
que feiravam os reis de Congo e de Luanda. 

Se nossa população fosse mais compacta; se a imigração tivesse
abundantemente nutrido; se não protelasse tanto o ciúme da metrópole
nosso tirocínio colonial; os resultados da educação pelo cativeiro
seriam ainda mais brilhantes. Teria a raça europeia amplitude bastante
para absorver em seu seio a escravatura, disseminar rareando"-a por
todo o país e, assim, melhor desbastar"-lhe a rudez. 

Mas, senhor, meio século de tempo e dez milhões de habitantes para este
imenso império, o que são?

Um átomo no espaço; um bochecho de água no oceano. 

Nada mais.

Destes exemplos suculentos se nutre a minha profunda convicção da
natural e não remota extinção da escravidão em nosso país.

A época precisa não é dado ao publicista averiguar, e ainda menos ao
legislador decretá"-la. Depende do incremento da população, que é o
princípio regulador da origem, como do termo da instituição.

Povos guerreiros, mas escassos, serviram"-se da escravidão como uma
leva de operários e um aumento artificial de população. À medida que
avultava o número dos habitantes livres, o cativeiro foi decaindo. Em
chegando a absorção, o escravo torna"-se traste de luxo, de
instrumento industrial que era. Nesse período extremo, o odioso envolve
a instituição e a sufoca. O labéu então reverte"-se para o senhor; a
infâmia é para esse título desprezível.

Quando o nível da população livre sobre a escrava se elevar 		
consideravelmente, de modo que esta fique submersa naquela, a
escravidão se extinguirá logicamente no Brasil. Ela entrará naquela
fase de luxo e aversão. Até então, porém, é um elemento essencial do
trabalho neste vasto país.

\begin{flushright}
\textit{20 de julho de 1867\\ 
Erasmo}
\end{flushright}

\chapter[Quarta Carta]{Quarta Carta \subtitulo{sobre a emancipação}}

\noindent\textit{Senhor}

\setcounter{@sectionNumCenter}{0}

\sectionitem

Não estranhareis, senhor, que me alongue em assunto de si tão vasto.

 Livros, não cartas, reclamava seu completo desenvolvimento. Mas, se em
outro tempo faliam"-me as forças para tal empenho, mingua a vontade
agora. Já não tenho espírito para o estudo, pois tudo é presa da
aflição e tristeza nestes dias de tribulação. 

 Permiti"-me, pois, que discorra à discrição da ideia. 

 O nível da população livre sobe pelo aumento desta, como pela redução
da escravatura.

 Esta redução motiva um dos
aleives\footnote{ \textit{Aleive}: calúnia.}
 levantados pelos filantropos contra a instituição. Dizem que a espécie
humana não se multiplica no cativeiro; nobreza que partilham algumas
espécies irracionais. A comparação basta para espancar o sofisma. A
vida selvagem e a poligamia deviam ser para o gênero humano, como para
o animal, o estado mais prolífico.

 Que a escravidão fosse estéril no Oriente, onde se mutilavam os homens
e arrebanhavam as mulheres em serralhos, não se contesta. Mas na
América, onde a raça africana, longe de degenerar, ao contrário, se
temperou sob a influência de um clima suave; negar a sua espantosa
reprodução seria uma cegueira pertinaz. 

 Quem ignora a indústria da criação de escravos que tamanho
desenvolvimento alcançou nos Estados Unidos e abasteceu exclusivamente
desde o princípio deste século o mercado do sul? O tráfico foi ali
insignificante desde 1808; a maior importação, a que se fez depois de
1843 para o Texas, essa mesma não avultou. 

 Entretanto, a marcha ascendente da escravatura americana sobe nesta
escala admirável.

 Em 1790, a existência era de 693.397. Em 1800, de 892.400, crescimento
de 28\%. Em 1810, de 1.190.930, crescimento de 33\%. Em 1820, de
1.536.127, crescimento de 29\%. Em 1830, de 2.007.913, crescimento de
31\%. Em 1840, de 2.486.138, crescimento de 24\%. Em 1850, de
3.178.055, crescimento de 29\%. 

 Onde se viu uma espantosa reprodução da espécie humana?	\label{reproducao}

 O menor período para a duplicação da população europeia é de 34 anos,
em Baden.\footnote{ Antigo Arquiducado de Baden, hoje pertencente à Alemanha.}
 Na mesma União americana, a população do Norte, apesar dos subsídios
importantes da imigração, só dobra por quartéis de século.

 O Brasil não tem estatística, para que os números, inflexíveis
dialéticos, exibam a mesma irrefragável evidência da vasta reprodução
da raça africana. Mas cada um de nós tem a prova no seu lar, que povoam
as crias, não obstante o mau vezo das mães nas grandes cidades. 

 De resto, cumpre advertir em uma circunstância. A licença dos costumes
prepondera na escravatura, como nas classes ínfimas da sociedade, que
vivem com ela confundidas. Na parte livre, porém, os frutos dessa
desordem acrescem; enquanto escapam à parte escrava. O vigor prolífico
do homem cativo não aproveita à sua casta; o da mulher mesmo, em larga
porção, entra na população livre ou pelas casas de expostos e alforrias
no batismo ou pelo resgate frequente do pai ingênuo.

 Tranquilizem"-se os filantropos; a escravidão no Brasil não esteriliza \label{filantropo}
a raça nem a dizima. A redução provém desses escoamentos naturais, que
se operam pela generosidade do senhor, pela liberdade do ventre e
também pela remissão. Diariamente, esses meios se desenvolvem à medida
que sobe o nível da civilização com o aumento da classe livre.

 Dois são os modos de incremento para a população livre, a geração e a
acessão. Limitada ao primeiro unicamente, só tarde conseguira ela
atingir a capacidade necessária para absorver a escravatura ou
preencher o vácuo deixado por esta. É necessário que a coadjuve o
segundo meio, a acessão, ou incorporação de população estranha.

 Essa incorporação pode ser de castas estranhas já existentes no país,
mas separadas por sua barbaria e condição. Neste caso estão as hordas
selvagens dos indígenas que vagam em Amazonas, Mato Grosso, Goiás e
outras províncias; e também a parte emancipada da casta servil, que se
anexa e assimila ao todo da população.

 A maior acessão de habitantes, depois que se desenvolveram as vias de
comunicação e a Europa regurgita de população, é sem dúvida a
emigração. Foi ela que pôs termo à escravidão nos Estados Unidos e há
de operar a mesma revolução no Brasil. Sem esse transbordamento do
mundo antigo; sem essa locomoção das massas que a indústria facilita; o
braço servil teria de laborar por muitos séculos a América.

 A emigração é a grande artéria que despeja novo sangue vigoroso no
organismo do país enervado pelo trabalho escravo. É ela que restabelece
o temperamento da população e lhe restitui a robustez. 

 Notai, senhor, que eu falo da emigração, e não da colonização: tão
fecunda é aquela, quanto estéril esta. A colonização, se escapa de uma
especulação escandalosa, degenera em servidão, opressiva como a
escravidão e mais turbulenta do que ela; já a chamaram, em com justiça,
escravidão branca.\footnote{ Havia pelo menos dois projetos para a chegada de trabalhadores 
livres ao Brasil: imigração espontânea de pequenos proprietários rurais mediante concessão 
de lotes de terra e colonização subsidiada de trabalhadores rurais para as grandes fazendas. 
Antes da crise mundial da escravidão, muitos conservadores (Alencar inclusive) previam o 
convívio entre grandes unidades produtoras escravistas e pequenas propriedades de imigrantes.}

 A propósito da emigração, quero apresentar"-vos, senhor, uma
consideração triste. 

 Filhos da velhice de um povo, educados neste canto do mundo sem ar e
sem luz, sem o ar da liberdade e a luz da civilização; conquistamos
nossa independência em 1823, quinze anos apenas depois que cessou 
a nossa clausura com a franquia dos portos ao estrangeiro.

 Entrando na sociedade das nações, tomamos logo, do primeiro passo,
lugar entre as mais livres. Ainda na fase agitada da organização,
conseguimos não obstante desenvolver nossos recursos e trilhar a senda
do progresso. Enquanto, em torno de nós, as repúblicas de origem
espanhola eram dilaceradas pela anarquia, o império se consolidava pelo trabalho.

 As provas de honestidade que deu o país nascente no instante de sua
emancipação, indenizando Portugal de uma parte de sua dívida, não se
desmentiram. Apesar das perturbações inevitáveis de suas finanças mal
organizadas, o Brasil foi sempre um estado probo, que honrava sua firma
nas praças da Europa. 

 Um espírito liberal a respeito da nacionalidade animava o povo
brasileiro e sua legislação. Oferecemos hospitalidade cordial a todas
as religiões, como a todas as escolas; e isso no tempo em que estas
ideias de liberdade e tolerância não eram aceitas por muitos dos
principais países da Europa. A naturalização dependia de fácil
processo; e a constituição (art.~6, §1º), hoje infelizmente interpretada, 
nacionalizava a prole do residente estrangeiro.\footnote{ Trata"-se do artigo constitucional 
que concedia cidadania aos filhos de libertos e de estrangeiros, quando nascidos no Brasil. 
Na década de 1860, por reclamação de outros governos, o Império permitiu que os filhos de 
estrangeiros não se tornassem imediatamente cidadãos brasileiros. Daí a reclamação de Alencar.}

 Entretanto, senhor, que fazia a Europa enquanto envidávamos esforços
para mostrar"-nos dignos da civilização? Enviava"-nos acaso as sobras
de sua população industriosa, à mingua de recursos, para coadjuvar a
obra de nosso desenvolvimento, fartando"-se na abundância deste solo?

 Oh! que não! Prescindindo de nossos irmãos de origem, os portugueses,
que vinham trazidos por tantas afinidades; só apareciam no Brasil de
outras nações certo número limitado de comerciantes, que estacionavam
na cidade, e alguns viajantes, que retribuíam nossa cordial
hospitalidade com a maledicência. Parva satisfação de ridicularizar uma
sociedade infantil, como se as crianças nascessem falando; e os povos, já civilizados. 

A Germânia, essa grande fábrica de homens, \textit{humani generis
officinam},\footnote{ ``Fábrica do gênero humano'', frase provavelmente extraída de Montesquieu, 
\textit{L'Esprit des lois}, \textsc{xvii}, cap.~5, onde se lê citação de Jornandez, historiador do século \textsc{v} 
que escreveu \textit{De origine actibusque getarum} (Sobre a origem e os feitos dos godos). Cumpre notar que 
as \textit{Novas cartas políticas} usam \textit{L'Esprit des lois} sem admitir uma de suas ideias capitais, isto é, que o clima 
condiciona as formas de trabalho no mundo.} como a chamou Jornandez, arrojava a aluvião de sua 
raça opulenta para a América do Norte. O Brasil, se quis, teve de pagar bem caro alguns centos de colonos, que
não indenizaram com seu trabalho o mal que fizeram a nossa reputação suas queixas injustas.

 Que decepções temos sofrido, senhor. O homem do norte, o puro saxônio,
o atleta da indústria, portento de atividade, em aportando ao Brasil,
parece que perde seu espantoso vigor e cai numa prostração
incompreensível! Para fazer desse indivíduo um trabalhador, é preciso
agasalhá"-lo bem, abrir"-lhe boas estradas para que penetre no
interior e ali preparar"-lhe a casa com todos os aprestos necessários
a uma cômoda existência. 

 Entretanto, o filho da raça latina, o explorador português, nos tempos
coloniais, arrojava"-se destemidamente ao deserto; levava consigo não
somente seu caminho, que ele abria através da floresta; como sua casa,
que levantava com algumas palmeiras no lugar escolhido. Assim foram
criadas as nossas povoações do interior. 

 Dirão que havia na América do Norte muitas atrações para chamar os
europeus: a língua, a índole, a religião, os usos. Não o contestamos. A
emigração é uma corrente entre a Europa e a América. São baldados os
esforços para desviar seu primeiro curso antes do prazo. Quando os
Estados Unidos abarrotarem de população, o Brasil receberá os
transbordamentos. 

 Mas, se não nos arrogamos o direito de pedir contas à Europa do destino
de sua emigração e do vácuo imenso que deixa neste império; se nos
resignamos a caminhar gradualmente com os subsídios do nosso velho
Portugal; parece que devíamos estar isentos dos reproches da
filantropia europeia a respeito da escravidão.

 Com efeito, quem manteve a escravidão no Brasil desde a nossa
independência? Quem desenvolveu o tráfico depois de 1835? Quem
especialmente, depois da extinção daquele comércio ilícito em 1852,
conservou o trabalho escravo em nosso país?

 A Europa, e somente a Europa. É a verdade, senhor; e eu sinto não ter
uma dessas vozes que o gênio faz estrondosa, para repercutir bem longe,
no seio do velho mundo, velho moralista à guisa de
Epicuro.\footnote{ Alusão ao filósofo grego Epicuro (341--270 a.C.), que, no trecho, metaforiza
o modo de vida europeu, profundamente hipócrita e cujo fim é a busca dos prazeres da vida.} 

 Se aquele grande viveiro de gente houvesse nestes últimos quinze anos
enviado ao Brasil um subsídio anual de sessenta mil emigrantes, número
muito inferior à imigração americana, a escravidão teria cessado neste
país. Venha ainda agora esta torrente de população e, em vinte anos
mais ou menos, afirmo que o trabalho escravo estará extinto no império,
sem lei abolicionista, sem comoção nem violência.

 Prevejo o subterfúgio por onde se hão de escapar. Dizem que a
escravatura repele a imigração branca; e citam o exemplo dos estados do
Norte da União Americana em paralelo com os do sul. Erro completo. A
avultada imigração daquela parte da Confederação foi causa e não efeito
da abolição da escravatura. A teoria da repulsão do trabalho livre pelo
escravo é um grande absurdo. Vale o mesmo que a torrente, força ativa e
enérgica, dizer à terra, à resistência inerte --- ``retirai"-vos que eu
quero passar''. A onda cava e abre seu álveo; é o que faz o trabalho
livre em país de escravos. Assim já vai sucedendo no Amazonas, Ceará,
Rio Grande do Norte e outras províncias. 

 Portanto, em vez de consumir seu tempo a caluniar nossas intenções e
deprimir os costumes brasileiros, melhor promovera a filantropia
europeia suas vistas humanitárias, ocupando"-se em desvanecer as
injustas prevenções levantadas contra o império americano. 

Não é ao monarca do Brasil, a vós, senhor, que se devia dirigir a
sociedade abolicionista de França:\footnote{ Em 1866, após o fim da escravidão no sul dos 
Estados Unidos, o \textit{Comité Français d'Émancipation} enviou ao Império do Brasil uma 
carta pedindo ao país providências para erradicar o cativeiro. Sob ordens de D.~Pedro~\textsc{ii}, 
o gabinete da Liga Progressista subscreveu resposta que prometia resolver o problema o mais rápido 
possível. O anúncio provocou grande repercussão no Império.} 
a causa moral e econômica do trabalho livre está ganha há muito tempo
em vosso espírito e coração, como na consciência de vosso povo. A
aplicação é somente o que falta, para a tornar uma realidade neste país.

 Se o Sr.~Laboulaye\footnote{ Edouard Lefebvre de Laboulaye (1811--1883), 
jurista, político e membro do \textit{Comité Français d'Émancipation}. Ocupou a
presidência da organização e foi um dos signatários da petição enviada ao Império 
em 1866, rogando o fim da escravidão no país.} visitasse o Brasil, havia de palpar esta verdade. 

 Não depende de nós, que não fabricamos população, mas dos emigrantes
unicamente, a aplicação do trabalho livre no Brasil. A eles, pois, aos
europeus convença a sociedade abolicionista da necessidade de buscarem
nosso país, a fim de aliviar a humanidade da pecha da escravidão.
Estabeleçam a propaganda neste sentido: mostrem ao interesse individual
o império como ele é e darão ao grande princípio da liberdade um
triunfo generoso e incruento. A escravidão cairá sem arrastar à miséria
e à anarquia uma nação jovem.

\sectionitem

 Há um terror pânico da unanimidade, que assalta os espíritos fracos.

 Essa resistência da unidade contra a multidão os apavora e acabrunha.
Abatem suas convicções à pressão da totalidade; e deixam"-se arrastar
atados à cauda do prejuízo, como da verdade. 

 A causa da emancipação em nosso país fez caminho rápido por este meio,
graças àquele pânico. Muitos espíritos se assustaram seriamente com a
ideia de que o Brasil era atualmente o único país onde a escravidão
existia no seio mesmo da pátria, sem o caráter colonial; e, brevemente,
seria talvez o único onde vivesse uma instituição universalmente
execrada.\footnote{ O autor faz menção ao fim da escravidão nos Estados Unidos (1865), que surtiu
da Guerra de Secessão (1861--1865), e ao início das atividades
abolicionistas na Espanha contra o cativeiro nas ilhas coloniais de
Cuba e de Porto Rico. Com a abolição nos Estados Unidos, a principal
potência escravista no século \textsc{xix}, Cuba, Porto Rico e Brasil passaram a
ser os únicos lugares que abrigavam a instituição. Sem condições de
impor ao cenário internacional a aceitação do cativeiro, como fazia a
república norte"-americana, o império espanhol e o Brasil procederam
ao processo legislativo de emancipação quase ao mesmo tempo, que
resultou, respectivamente, na Lei Moret (1870) e na Lei Rio Branco (1871).}

 Esta ideia, bem ataviada pelos filantropos, devia comover o ânimo
nacional. Nenhum povo brioso consentiria em ficar na última fila das
nações cultas, quase confundido com os estados semibárbaros do Oriente,
objeto de aversão para a humanidade. No desígnio de resguardar"-se de
semelhante humilhação, ninguém, homem ou povo, hesitaria em
sujeitar"-se aos maiores sacrifícios.

 Será verdade, porém, senhor, que a escravidão, reduzida exclusivamente
ao Brasil, o arraste àquela posição aviltante? Daremos nós prova de
barbaria e iniquidade mantendo a instituição apesar de sua completa
abolição no resto do mundo?

 Decididamente, não.

 Antes de qualquer consideração, não se esqueça a natureza da escravidão
em nosso país, tal como a fizeram, acinte da lei, os costumes nacionais
e a boa índole brasileira.

 Os Estados Unidos, nação poderosa, com perto de um século de existência
política, e um desenvolvimento espantoso da indústria, só agora
conseguiram extirpar o trabalho escravo do sul de seus estados. As mais
poderosas nações da Europa, Inglaterra e França, grandes já quando
estávamos no limbo do desconhecido, só neste século, e no segundo
quartel, obtiveram purgar suas colônias do elemento servil.

 Ao Brasil, pois, é que se há de estranhar a demora neste supremo
esforço, quando ainda está ele na infância, contando apenas quarenta e
quatro anos de existência política, depois de três séculos de
isolamento e abandono? 

 Tanto vale escarnecer da criança porque não se tornou homem ainda!

 Não temo, senhor, para nossa pátria, que lhe venha desonra de conservar
a escravidão por algum tempo ainda, depois de geralmente abolida.
Seremos os últimos a emancipar"-nos dessa necessidade; mas há quem
possa atirar"-nos a pedra por esse pecado da civilização?

 Se esse povo existe, de consciência limpa, ele que se levante.

 Será acaso a França?

 Não é possível. A França, que aboliu a escravidão de suas colônias em
fins do século passado, no momento em que fazia ao mundo a pomposa
declaração dos direitos do homem, retratou"-se restabelecendo"-a
poucos anos depois, para só extingui"-la em 1848; a França não tem o
direito de levantar a voz neste assunto. Conservar escravo o homem que
nasceu tal é uma instituição; reduzir à escravidão pessoa livre é um
crime.\footnote{ O autor se remete à primeira abolição da escravidão no império francês, decretada
pela Convenção francesa em fevereiro de 1794. A medida foi tomada na
esteira do maior levante de escravos da história, em Saint Domingue
(atual Haiti), então colônia francesa e o primeiro produtor mundial de
açúcar. Em 1802, Napoleão despachou tropas para a ilha a fim de
restabelecer o cativeiro e, com isso, aumentar a arrecadação do Estado
francês. Daí a acusação ``reduzir à escravidão pessoa livre é um crime''.
Os ex"-escravos resistiram ao exército metropolitano e decretaram a
independência política da ilha em 1804, a segunda de todo o continente
americano, posterior apenas à dos Estados Unidos (1776).}
 
Será acaso a Inglaterra?

Oh! Essa menos que nenhuma outra! À soberba indignação britânica,
permiti"-me opor a palavra sensata de um homem ilustre, que, se foi
mau político, em sentimentos cristãos ninguém o excedeu. Chateaubriand,
defendendo sua pátria contra a filantropia inglesa, como eu agora
defendo a minha contra a filantropia francesa, escreveu o seguinte: 

 ``A Inglaterra tinha medo que o tráfico de africanos, a que ela
renunciara com pesar, caísse nas mãos de outra nação; queria forçar
França, Espanha, Portugal e Holanda a mudar subitamente o regime de
suas colônias, sem indagar se estes estados haviam chegado ao grau de
preparação moral em que se podia dar liberdade aos negros, abandonando,
ao contrário, à graça de Deus a propriedade e a vida dos
brancos.''\footnote{ Frase extraída da já citada obra de Chateaubriand, 
\textit{Congrès de Verone}, p.~78.}
 

 Em seguida, recorda como todos os torys ilustres, Londonderry,
Wellington, Canning, durante trinta anos adversários firmes da moção de
Wilberforce;\footnote{ Referência, respectivamente, a Robert Stewert (visconde de Castlereagh
e marquês de Londonderry), a Arthur Wellesley (duque Wellington) e a
George Canning, todos eles políticos conservadores proeminentes no
início do século \textsc{xix}. William Wilbeforce foi membro do Parlamento
britânico, onde pregou ostensivamente ideias moderadas do movimento
abolicionista desde a década de 1790. Apoiou tanto a supressão do
tráfico negreiro (1807) como a abolição da escravidão (aprovada em 1833).} 
de repente se haviam eletrizado pela liberdade dos africanos; porque
essa liberdade era a ruína completa das colônias e navegação das nações
marítimas, suas competidoras. O egoísmo se embuçara com a filantropia.

 A Inglaterra, que no tempo de Cromwell\footnote{ Oliver Cromwell, 
que governou a Inglaterra revolucionária de 1653 a 1658.}
 tolerou a venda de escravos brancos na América;\footnote{ Afirmação provavelmente extraída de Chateaubriand, 
\textit{Congrès de Verone}, p.~79. O sistema de trabalho forçado de brancos pobres nas
colônias inglesas e francesas (``indenture labour'' ou ``engagement'')
ascendeu na primeira metade do século \textsc{xvii}, mas declinou assim que o
sistema escravista e o tráfico negreiro se consolidaram. O parágrafo
alude à venda milhares de católicos irlandeses à América Inglesa, no
contexto dos conflitos religiosos da Revolução Inglesa.} 
e ainda hoje admite o chicote como instrumento de castigo em sua
marinha, depois de haver proibido no art.~17 do bill abolicionista de
28 de agosto de 1833, a respeito do negro, essa pena ``que degrada a
dignidade humana''; a Inglaterra devia rasgar quanto antes o bill
Aberdeen,\footnote{ O bill Aberdeen (1845) previa prisão unilateral de barcos 
brasileiros envolvidos no tráfico negreiro e seu julgamento em tribunais britânicos. 
De 1849 a 1851, a Inglaterra estendeu sua aplicação às águas territoriais do Império, 
provocando um estado de guerra virtual no que talvez tenha sido o pior impasse diplomático 
da história brasileira. O resultado imediato foi a supressão definitiva do contrabando (1850), 
mas a Inglaterra só revogou o bill Aberdeen em 1869, após a Guerra Civil selar o destino mundial da escravidão.} 
que é antes uma nódoa viva no seu passado do que uma prepotência contra uma nação fraca.

 Se estas duas nações não podem lançar"-nos a pedra, menos qualquer
outra da Europa. O velho mundo tem em seu próprio seio um cancro
hediondo que lhe rói as entranhas: é o pauperismo. O aspecto repugnante
desta miséria em que jaz a última classe da sociedade, a degradação
dessas manadas brutas, apinhadas em
esterquilínios;\footnote{ \textit{Esterquilínio}: local em que se deposita esterco.} 
rebaixa e avilta a humanidade mais do que a antiga
escravidão.\footnote{ A comparação do cativeiro com outras manifestações sociais --- 
operariado, campesinato, pobreza --- constituía argumento dileto de pró-escravistas norte"-americanos. 
Fazê"-la era uma maneira de tornar ecumênica a desigualdade social, da qual a escravidão seria apenas 
um exemplo particular. No Brasil, a estratégia foi empregada em momentos críticos para a instituição, 
como os debates de 1827 sobre a convenção anglo"-brasileira que previa o fim do tráfico, as discussões 
que o bill Aberdeen provocou em 1845, as contendas subsequentes à supressão definitiva do contrabando 
em 1850 e as controvérsias que surgiram após a crise mundial da escravidão (1865).} 

 Valem"-se os filantropos, apanhados em flagrante, da liberdade e
encarecem este dom além da realidade. Se a independência fosse o
destino do homem, o selvagem seria o mais civilizado e próximo da
perfeição. A liberdade é o meio, um direito; o fim é a felicidade, e
desta o escravo brasileiro tem um quinhão, que não é dado sonhar ao
proletário europeu. De que serve ao pária da civilização a liberdade
que a lei consagra por escárnio, quando a sociedade a anula fatalmente
por sua organização criando a opressão da miséria? 

 Se não há na Europa, devorada em suas entranhas, haverá acaso na
América povo que nos lance a pedra?

 Porventura os Estados Unidos orgulhosos da recente abolição? Não creio.
Era preciso esquecerem as atrocidades ali cometidas contra os escravos;
as caçadas de negros a dente de cão; os prejuízos selvagens de raça;
enfim, todo esse cortejo odioso da escravatura americana, da qual por
crassa ignorância dividem com o Brasil a responsabilidade. 

 Os Estados Unidos têm bastante em que se ocupar com o fermento de suas
paixões políticas e a aluvião de uma escravatura recentemente liberta;
para se darem a utopias filantrópicas, enlevo dos espíritos devolutos. 

 Serão as repúblicas da América que nos exprobem a conservação da escravatura? 

 Talvez, porque não podem sofrer a superioridade do império. Abolindo no
momento da emancipação o trabalho servil, esses povos embriagados de
liberdade sufocaram sua pequena indústria, especialmente sua lavoura
rudimentária. A agricultura é um elemento essencialmente conservador;
eliminando"-o, as repúblicas americanas se abandonaram à anarquia. 

 Esses países convulsos, laborados pela guerra civil, consumidos pela
febre revolucionária, talvez reprochem ao Brasil haver seguido outra 
direção. De feito, o império, resistindo às seduções da liberdade,
preservou sua agricultura. Graças a este esforço pode mostrar"-se
probo e sisudo, honrando sua firma na Europa; e assegurando a seus
filhos uma pátria nobre e digna.

 Uma só página da história das repúblicas do centro e sul da América é
bastante para calar a voz que se levante aí contra a escravidão no império. 

 Caminhe, pois, o Brasil desassombrado. Não se deixe tomar de pânico
ante a opinião geral. Em todos os países, ainda os mais civilizados, há
uma última raiz do passado; entre nós é a escravatura, como na Europa é o pauperismo.

\sectionitem

 É o momento de considerar a abolição a respeito da forma e da
oportunidade.

 Contra as considerações que desenvolvi, sem dúvida surgirão em vosso
espírito objeções deduzidas do projeto em via de elaboração. Não
pretende o governo a abolição imediata, porém, sim, depois de finda a
guerra. Nessa mesma ocasião, a medida não será instantânea, porém,
gradual e a longo prazo.

 Assim, previne"-se o risco de um grande abalo na sociedade, e
modera"-se a perturbação econômica. A substituição do trabalho servil
pelo trabalho livre se realiza proporcionalmente; à medida que um se
retrai, o outro se dilata. Meditei todas estas razões e muitas outras
que se podem produzir em favor do sistema.

 Não hesito, porém; eu o condeno.

 Se um governo, desconhecendo a natureza da escravidão, se propõe
extingui"-la por ato legislativo; neste caso sempre desastroso, eu lhe
aconselhara antes o meio pronto, súbito, instantâneo, como uma
calamidade menor. Era uma amputação dolorosa; se o enfermo não
sucumbisse, a chaga iria cicatrizando, e ele ficaria mutilado, porém, tranquilo.

 Mas essa operação lenta, excessivamente dolorosa, torna"-se
insuportável: quanto mais longa, mais perigosa. A sociedade não pode
permanecer dez ou vinte anos em guarda constante contra a insurreição
minaz que uma faísca basta para levantar. A comoção causada por esse
perigo surdo, mas presente a toda hora, perturba a existência de um povo.

 É ilusória a esperança de uma substituição lenta. No momento em que
plainasse sobre o país uma lei de emancipação qualquer; toda a casta
sujeita se colocaria à sombra dela, para deduzir daí seu direito
indisputável. Pouco importavam as condições; tudo se resumia no grande
princípio, no reconhecimento solene de sua liberdade.

 Desvanecido o prestígio da instituição, cada um desses indivíduos seria
um adversário disputando seu direito ao opressor; e coagindo"-o a
consagrá"-lo em sua plenitude. A geração nova, libertada no ventre,
era a primeira a revoltar"-se para arrancar ao cativeiro seus
progenitores. E quem teria o direito de estranhar neles o estímulo
nobre do amor filial?

 Não esqueçam as simulações. Já tivemos o exemplo a respeito do tráfico:
todos os indivíduos novamente importados eram lançados à conta do
tempo em que era lícita essa
aquisição.\footnote{ Em 7 de novembro de 1831, o Império aprovou uma lei que suprimia 
o comércio de escravos e decretava livres os africanos ilegalmente introduzidos no país. 
A origem do Partido Conservador, a que pertencia José de Alencar, se liga às manobras 
políticas e sociais que buscavam suspender a aplicação da lei. Em consequência, o contrabando 
aportou para o país mais de 650 mil escravos ilegais entre 1836 e 1850, a despeito da lei que devia libertá-los.}
 Assim, hão de retroagirem ao cativeiro os nascimentos acontecidos já no
período de liberdade. Mais um elemento para a combustão.

 A Inglaterra adotou a respeito de suas colônias o sistema gradual.
Criou um estado intermédio entre a escravidão e a liberdade, que
designou com o nome de aprendizagem, durando entre quatro e seis
anos.\footnote{ Vide nota 1, p.~\pageref{nota1}.} ``Transição perigosa'', diz Cochin, 
``que expunha as colônias à desordem, a propriedade à ruína, a liberdade 
a uma derrota sanguinolenta e onerosa''. (Vol.~\textsc{i}, p.~377). 

 Com efeito, se não fosse o grande poder da Inglaterra, vigilante e
alerta durante essa operação arriscada, a explosão da liberdade,
imprudentemente agitada, mas não desabafada, houvera exterminado as
colônias. Assim mesmo, sob o sistema de proteção da metrópole, a
convulsão durou anos e tomou algumas vezes aspecto medonho.

 Que será do Brasil, senhor, em uma crise semelhante, não fora da
influência dela, mas no foco mesmo da agitação, atribulado pelo mal
interno, obrigado a atender a todos os perigos, sociais e políticos? Já
lançastes, senhor, vosso espírito a essa terrível conjectura e
sondastes estes refolhos dos acontecimentos?

 Confesso"-vos que essas profundezas do futuro me causam vertigens.

 A única transição possível entre a escravidão e a liberdade é aquela
que se opera nos costumes e na índole da sociedade. Esta produz efeitos
salutares: adoça o cativeiro; vai lentamente transformando"-o em mera
servidão, até que chega a uma espécie de orfandade. O domínio do senhor
se reduz então a uma tutela benéfica.

 Esta transição, fora preciso cegueira para não observá"-la em nosso
país. Viesse ao Brasil algum estrangeiro, desses que devaneiam em
sonhos filantrópicos nas poltronas estufadas dos salões parisienses, e
entrasse no seio de uma família brasileira. Vendo a dona da casa,
senhora de primeira classe, desvelar"-se na cabeceira do escravo
enfermo; ele pensaria que a filantropia já não tinha que fazer onde
morava desde muito a caridade. 

 Estudando, depois, a existência do escravo, a satisfação de sua alma, a
liberdade que lhe concede a benevolência do senhor; se convenceria que
esta revolução dos costumes trabalha mais poderosamente para a extinção
da escravatura do que uma lei porventura votada no parlamento.

 Todas as concessões que a civilização vai obtendo do coração do senhor
limam a escravidão sem a desmoralizar. O escravo não as erige em
direito para revoltar"-se, como sucede com os mínimos favores de uma
lei; ao contrário, tornam"-se para ele benefícios preciosos que o
prendem ainda mais à casa pela gratidão. Esse cativo, se for libertado,
permanecerá em companhia do senhor; e se tornará em criado.

 O liberto por lei é inimigo nato do antigo dono; foge à casa onde
nasceu. O ódio da raça, que se havia de extinguir naturalmente com a
escravidão, assanha"-se ao contrário daí em diante. Tal será a sua
ferocidade, que uma casta se veja forçada pelo instinto da conservação
a exterminar a outra.

 Bem sabeis, senhor, a sorte deplorável dos cativos que por sua morte
Washington\footnote{ George Washington, primeiro presidente dos Estados Unidos
(1789--1797), assim como Thomas Jefferson, também presidente de 1801 a 1809, 
foram proprietários de escravos na Virgínia, estado que conservou o 
cativeiro até o fim da Guerra de Secessão (1861--1865).} deixou libertos. 
Pereceram na miséria. Não ignorais também que
Jefferson, entristecido com estes exemplos, não se animou a realizar de
plano sua ideia da emancipação geral, limitando"-se a prepará"-la
pela reexportação dos africanos, de que procede a atual república da Libéria.

 Não resta dúvida. A abolição gradual é mais nociva do que a abolição
instantânea. Para esta, a nação concentra suas forças durante a
operação e repousa logo do grande choque. Há perigo, e perigo sério, mas rápido e passageiro.

 Entretanto, senhor, se neste assunto confio principalmente na revolução
íntima dos costumes e ideias da sociedade, não descreio, contudo, da
ação da lei sábia, que exerce nos preconceitos uma influência benéfica,
por isso mesmo que é indireta e branda. Como vício constitucional do
império,\footnote{ Era relativamente comum avocar a Constituição para defender 
o cativeiro no Brasil. O Artigo \textsc{vi}, parágrafo \textsc{i}, o reconhecia na concessão de cidadania 
aos escravos que, sendo nascidos no país, obtivessem alforria. No Artigo 179, garantia 
em toda a plenitude a propriedade existente dos súditos brasileiros, aí incluídos, 
implicitamente, os escravos.} não pode a escravidão ceder a remédio; mas convém submetê"-la a um
certo regime, a uma higiene administrativa.

 Carece de grave meditação o complexo de medidas tendentes à preparação
moral e econômica do país para o trabalho livre. Se eu nutrisse
esperança de que minhas ideias a este respeito captariam vossa atenção,
as explanara de certo. Poupo ao meu espírito mais um desengano. 

 De todas estas considerações que apontei e que, bem desenvolvidas,
davam matéria para um livro, a suma é esta:

 Para a casta sujeita, ainda não educada, a emancipação nas
circunstâncias atuais é um edito de miséria pelo abandono do trabalho e
de extermínio por causa da luta que excita entre as duas raças.

 Para a casta dominante, especialmente a agrícola, importa a ruína pela
deserção dos braços e impossibilidade de sua pronta substituição;
importa igualmente o perigo e sobressalto da insurreição iminente. 

 Para o estado, significa a bancarrota inevitável pelo aniquilamento de
sua primeira indústria, fonte da riqueza pública; e, como consequência,
o crédito nacional destruído, a nossa firma desonrada no mercado estrangeiro.

 E chama"-se a isto filantropia? É esta oblação feita da melhor
substância nacional, amassada com lágrimas e sangue de uma população
inteira, que se deseja votar à caridade?

\begin{flushright}
\textit{Rio, 26 de julho 1867\\
Erasmo}
\end{flushright} 

\chapter[Quinta Carta]{Quinta Carta \subtitulo{sobre o donativo imperial}}

\noindent\textit{Senhor}\smallskip

 Resolvestes desde já ceder para as urgências do estado a contar de
março vindouro a quarta parte de vossa
dotação.\footnote{ Verba permanente destinada ao sustento da família imperial, prevista na
Constituição de 1824 (Título \textsc{v}, capítulo \textsc{iii}) e que montava a 800
contos de réis anuais. Durante a dispendiosa campanha no Paraguai, D.~Pedro~\textsc{ii} 
repassou ao Estado um quarto de sua dotação como contribuição para atender aos esforços de guerra.}


 Dirigistes para este fim uma carta ao Sr.~Zacarias,\footnote{ Referência a 
Zacarias de Góis e Vasconcelos (1815--1877), que presidiu
ao ministério da Liga Progressista de agosto de 1866 a julho de 1868.}
 que a leu perante a câmara dos deputados com a devida solenidade.

 Creio que o nobre presidente do conselho figurou aí como simples órgão
da nação, a quem naturalmente se referia vosso pensamento, praticando
esse ato de abnegação. 

 Como cidadão, que ainda me consentem ser deste império, e um dos
contribuintes do orçamento, tenho uma parte, embora tenuíssima, na
vossa generosidade. Não devo, pois, conservar"-me indiferente. 

 Já a imprensa em nome da opinião pública vos retribuiu com bonitos e
merecidos elogios. No parlamento a leitura de tão importante documento
foi saudada com ferventes aplausos. 

 Quero eu também responder"-vos por minha conta própria.

 Não aceito, senhor, o vosso donativo; e até vos contesto o direito de o
fazer. Se tomais por uma exorbitância este meu modo de pensar, lede a
constituição, que vos fez imperador. 

 A dotação, conferida pela nação ao monarca, bem com aos membros
principais da dinastia, não é uma remuneração de serviços, como o
ordenado do funcionário público.

 Pelo trabalho de governar, decerto não vos daria o Brasil oitocentos
contos de réis anuais; e menos ainda os cem contos que recebem as
augustas princesas, sem a mínima ingerência no governo do país. 

 É o decoro do trono e a dignidade da nação, como diz"-nos a lei
fundamental (art.~108), que determina a dotação. Foram estas razões,
inteiramente alheias a vossa pessoa, que elevaram à soma atual o
pequeno apanágio de vosso augusto pai. 

 Assinando a quantia de oitocentos contos de réis para vosso tratamento
anual, arbitrou a assembleia geral o grau de lustre e pompa da coroa
brasileira. Desde, pois, que cedeis uma parte dessa dotação, não
alienais vosso dinheiro ou uma parte de vosso patrimônio; mas sim um
quinhão do decoro do trono e da dignidade nacional, coisas que não
pertencem ao Sr.~D.~Pedro~\textsc{ii}, pois é delas mero depositário. 

 Pode um empregado ceder em benefício do estado uma parte ou mesmo todo
o vencimento, porque dá do seu; oferta à pátria necessitada algumas
bagas de suor, algumas horas de fadiga. Mas vós, senhor, vós, cuja
existência inteira foi dedicada à felicidade deste povo, não tendes o
direito de ser pródigo de semelhantes migalhas.

 É sabedoria e prudência que a nação espera de seu monarca e lhe pede
com ânsias. Quanto às espórtulas\footnote{ \textit{Espórtula}: esmola, gorjeta.}
 pecuniárias, que lhe jogam em paga de sua paciência evangélica, afirmo
que ela as rejeita. 

 O povo brasileiro tem dado provas de nimiamente sofredor. Não se contam
já as humilhações que ele há suportado impassível desde o princípio
desta guerra. Mas, se esquece seus brios, ainda não desceu felizmente à
vileza de os regatear. 

 Estes duzentos contos, que renunciais, são muito para vossa casa
desfalcada e sempre mal gerida: são demais para os infortúnios que
vossa mão beneficente alivia. São nada, porém, para a nação
oberada\footnote{ \textit{Oberado}: endividado.}
 com uma despesa enorme e um desfalque estupendo. 

 Ah, senhor! Se quereis ser generoso para com esta nossa pátria, tão
deserdada do amor de seus filhos e tão órfã de seu monarca, não é
atirando"-lhe aos centos de contos de esmola que lograreis essa
glória. Não! Será pondo um termo a esse esbanjamento desordenado que
tem exaurido todas as reservas do país e vai sorver os últimos recursos
do futuro.

 Não são os vossos duzentos contos de réis que hão de suprir o vácuo
aberto no orçamento por uma administração imprevidente e desasada. 

 Não há de ser a quarta parte de vossa dotação que nutra o manancial de
ouro já estanque, para de novo despejar aos jorros nas repúblicas do
Rio da Prata.\footnote{ Alusão à Guerra do Paraguai (1864--1870), cujo progresso o autor
atribui a um capricho vaidoso do Imperador. Após o revés militar da
Tríplice Aliança em Curupaiti (setembro de 1866), estadistas brasileiros e argentinos 
aventaram a paz com o Paraguai, mas D.~Pedro~\textsc{ii} não a aceitou. Esta
quinta carta contrapõe à pretendida generosidade do monarca os enormes
custos de guerra, que apequenavam sua doação.}
 
Não é o vosso óbolo\footnote{ \textit{Óbolo}: moeda de pouco valor; esmola.}
 que virá garantir o crédito público profundamente abalado e a probidade
do império brasileiro, ameaçado de uma bancarrota infalível. 

 Não chega, enfim, senhor, a vossa espórtula para restituir à família do
operário e do lavrador a finta onerosa ou a vida do chefe imolada, não
à defesa da honra nacional, seria um dever sagrado, mas ao capricho de
alguns indivíduos, o que é uma iniquidade. 

 De que serve, portanto, senhor, privar"-vos de certa decência
indispensável ao trono; ou mesmo da íntima satisfação de enxugar uma
lágrima e mitigar uma dor? 

 Em vossa mão compassiva e boa, demais, esta soma terá melhor destino.
Talvez se transformasse nos orvalhos santos da caridade, a rociar as
aflições que penetram nessa mansão tranquila de S.
Cristóvão.\footnote{ Paço de São Cristóvão, residência imperial.}
 

 A beneficência é uma das pompas da majestade e prima entre as mais
brilhantes; compõe ainda melhor que os esplendores e as galas o decoro
do trono. Quando a realeza se unge nesta virtude, mostra"-se o
legítimo representante da soberania nacional, porque é também o
representante da Providência, que inspira o coração magnânimo dos povos. 

 Fazer da caridade uma espécie de atribuição exclusiva da igreja e de
seus vigários, como já pretenderam no parlamento brasileiro, seria uma
extravagância, se não fosse infelizmente coisa pior; um efeito do
grosseiro materialismo que pervade o país de todos os lados.

 Porventura, uma parte dessa quantia renunciada por vós não tivera
aquele sublime destino, porém, um emprego menos acertado, como o de
nutrir certas cobiças e vaidades parasitas do trono. Todavia, era
apenas uma prodigalidade de vossa parte, uma bondade mal usada. 

 Entretanto, abandonados ao governo, esses duzentos contos vão ser um
foco de imoralidade e corrupção. Carniça atirada ao tempo, que a
podridão logo decompõe, não tarda cobrir"-se de um enxame de vermes à ceva.

 Quanta paixão sórdida não vem acender esse punhado de ouro atirado
sobre o tapete verde do orçamento? Quanto embuste e mentira não custará
ao pudor político, já expirante, a dissipação desta migalha?

 Em nome da dignidade do país e da honestidade do governo, senhor,
retirai o presente funesto!

 Se houvesse necessidade real dessa quantia de duzentos contos de réis,
para desempenhar algum serviço indispensável da administração, ainda
assim não carecera o governo da quarta parte de vossa dotação. 

 Bastava"-lhe uma pequena emissão de títulos ou condecorações para
levantar prontamente soma igual, senão superior. Vinte baronatos ou
cinquenta comendas, eis, senhor, quanto justamente vale o vosso
donativo ao estado. 

 Que mal faria ao país, já tão inçado da praga, mais cinquenta fidalgos
despachados pela graça de seu dinheiro? No tempo em que se tiram galés
de Fernando de Noronha para confiar"-lhes a guarda do pavilhão
nacional, torna"-se com efeito indispensável enobrecer aqueles que não
perpetram roubos nem assassinatos.\footnote{ O trecho, todo irônico, sugere 
que a concessão de baronatos é compensação justa, já que o governo concedera 
a prisioneiros o direito de atuarem na Guerra do Paraguai.}
 
A não ser assim, que diferença houvera entre um facínora e um homem bem procedido? 

 No mesmo instante em que, para dissipar umas baforadas republicanas
sopradas lá do Serro,\footnote{ Cidade natal de Teófilo Otoni (1807--1869), 
conhecido por suas convicções republicanas.}
 esse decantado Acrópole mineiro, o nobre presidente do
conselho\footnote{ Zacarias de Góis e Vasconcelos.}
 usava de vossa carta como de um argumento de algibeira, 
sabeis o que se rumorejava na cidade?

 Falava"-se na quarta missão
extraordinária,\footnote{ Provável referência ao iminente envio de ajuda brasileira à Argentina,
conflagrada em 1867 por insurreições militares contrárias à Guerra do
Paraguai. O auxílio acabaria por não ocorrer.}
 que vosso insigne governo com um gênio admirável acabava de inventar,
para ir a Buenos Aires consumar a nossa vergonha diplomática e
desentranhar mais uma guerra do ventre fecundo desse monstro chamado a política platina. 

 Compreendeis bem, senhor, o alcance e a profundeza desta fatal coincidência? 

 Talvez não, porque uma névoa sinistra de certo tempo a esta parte tolda
vossa mente e lhe empana a reconhecida lucidez. Desde
1863,\footnote{ Ano da dissolução de uma Câmara majoritariamente conservadora.}
 vedes o país através das evaporações maléficas de uma política
desgraçada: a política da vaidade.

 A coincidência de vossa carta com os boatos de nova missão tem, senhor,
esta medonha significação, que gela a medula do país.

 No instante em que uma das augustas mãos estende à pátria aflita o
óbolo de duzentos contos, a outra, obstinada e imprudente, joga na
banca política uma nova cartada de duzentos mil contos, páreo que o
povo brasileiro terá de pagar, suando sangue e dinheiro. 

 Em maio de 1864, uma primeira embaixada se inventou, que partiu com
aparato para o Rio da Prata. Não soube então o país qual era seu fim.
Creio que nem o próprio monarca brasileiro ou seu gabinete o sabiam;
devo crer, senhor, porque a alternativa seria
cruel.\footnote{ Trata"-se da missão especial de José Antônio Saraiva enviada a Montevidéu, 
para exigir reparação de abusos cometidos contra brasileiros durante a guerra civil 
entre blancos e colorados, eclodida em 1863. O contencioso sofreria imprevisível 
escalada, até ensejar, indiretamente, a Guerra do Paraguai.}

 Só hoje conhece o Brasil o custo dessa filigrana diplomática. Duzentos
mil contos já consumidos; e soma igual, senão maior, para continuar a
obra"-prima do progressismo,\footnote{ Remissão à Liga Progressista (vide Introdução).}
 cujo remate, ficai certo, senhor, há de ser um grande opróbrio, como
foi seu princípio um grave crime.

 Segunda missão foi enviada a Montevidéu. Obteve esta com tino superior
aplacar a labareda açulada nas margens do Prata; porém, uma centelha
voara pelos ares, que produziu a explosão no seio do Paraguai. 

 A missão Paranhos foi condenada pelo
governo.\footnote{ Após o Uruguai desconsiderar as reclamações brasileiras, Saraiva 
deixou o país em gesto inamistoso e ordenou o envio de forças imperiais de mar e de terra. 
Depois de algumas batalhas, José Maria da Silva Paranhos, chefe da segunda missão e substituto 
de Saraiva, selou um acordo de paz entre blancos e colorados, sem lograr a inclusão, no tratado, 
das reclamações imperiais. Por essa razão, foi exonerado.}

 A lógica o exigia. Seu chefe, se não tinha alcançado tudo, conseguira o
possível. Não lhe era dado, nem a outro qualquer, suprimir o passado
implacável e evitar o futuro sinistro que já acudia com espantosa velocidade.

 A situação, que em 1863 se gerara no ventre do absurdo, devia, para ser
coerente, punir o importante serviço prestado ao país por aquela
missão.\footnote{ José de Alencar insinua que Paranhos foi exonerado pelo gabinete em exercício,
da Liga Progressista, apenas por pertencer ao Partido Conservador.}
 
 Passemos a esponja sobre isto.

 Seria nada o arreganho de Lopez se o Brasil fosse Brasil naquele
momento, se o Império se possuísse. Mas, infelizmente, desde maio de
1862,\footnote{ Data da substituição de um ministério conservador por outros gabinetes que
instituíram o domínio da Liga Progressista.}
 senhor, que o havíeis reduzido a \textit{anima vilis},\footnote{ Ser
desprezível, sem honra.} à besta destinada para as experiências 
de uma nova e incompreensível política.

 Que estímulos e brios podia ter uma nação rebaixada à condição
miserável de arcabouço ministerial, para a aprendizagem dos impúberes
estadistas? De que exerções de força e atividade era capaz um povo
enervado por governos fracos e completamente alheios à ciência da administração?

 O gabinete de 12 de agosto,\footnote{ \label{gale}Por lapso, José de Alencar fundiu o gabinete 
de 31 de agosto de 1864 com o
de 12 de maio de 1865, escrevendo ``gabinete de 12 de agosto''.
Trata"-se, na verdade, do ministério de 31 de agosto de 1864, que
estava em exercício quando se firmou Tratado da Tríplice Aliança, em 1
de maio de 1865.} que reprovara o ato diplomático de 20 de
fevereiro,\footnote{ Acordo de paz selado em 1865 que findava a guerra civil no Uruguai e
harmonizava as relações do país com o Brasil. Segundo Pereira da Silva,
o tratado de paz foi mal recebido pelo Imperador porque não regulava
reparações uruguaias aos brasileiros perseguidos na guerra civil.
Paranhos foi imediatamente exonerado em favor de Francisco Otaviano.
Confira J.~M.~Pereira da Silva. \textit{Memórias do meu tempo}, \textsc{ii}, pp.~5--31.}
 selou com seu nome o documento mais vergonhoso de toda esta guerra, o
tratado da tríplice aliança. Quando meus olhos perpassam essa página\ldots{}
suja, é o nome; essa página da diplomacia brasileira, sinto torvar"-se
o ânimo. Involuntariamente ocorre"-me a ideia de um homem assalariando
ao preço da dignidade dois espadachins para instrumento de sua vingança!

 Foi este pensamento ominoso que levou a Buenos Aires a terceira missão
extraordinária, pomposamente designada pelo vulgo de
embaixada.\footnote{ Missão de Francisco Otaviano, que assinou o Tratado da Tríplice
Aliança, em 1 de maio de 1865.}
 Não é possível calcular seu preço com exatidão, mas estou convencido
que ela nos custará ainda mais caro que a primeira.

 Em chegando a época da liquidação, quando tivermos de somar os cheques
pagos por conta do crédito aberto a duas repúblicas insolváveis; então
se poderá orçar o verdadeiro importe dessa aliança, consignada ao
Brasil pelo gabinete de 12 de
agosto.\footnote{ Vide nota 19, p.~\pageref{gale}. No início da guerra, o Tesouro brasileiro emprestara quase 600
mil libras esterlinas à Argentina e ao Uruguai. Posteriormente, o barão
de Mauá também abriu linhas de crédito aos aliados.}


 Portanto, senhor, se quereis ser generoso para nossa pátria, em vez de
reduzir vossa dotação, o que a nada monta, impedi essa quarta missão,
que apavora o espírito público, desde os primeiros e vagos anúncios;
obstai à nova importação de calamidades que se há de realizar por meio
dessa embaixada, como se realizou em 1864 e 1865. 

 Se fizerdes isso, não serão duzentos contos, mas duzentos milhões que
ofertais ao estado. Não poupareis ao Brasil vinte barões ou cinquenta
comendadores, que em tanto anda a quarta parte de vossa dotação;
poupareis uma infinidade de vidas e outra miséria maior, se é possível,
sobre esta miséria que nos aflige. 

 Quereis levar mais longe ainda a vossa generosidade e ser magnânimo e
esplêndido como costumavam os antigos imperadores da Ásia? 

 Despedi este ministério, que o país tem pago com tamanha usura. Cada um
dia de sua vida custa mais ao Brasil do que vossos duzentos contos;
porque lhe custa não somente ouro e sangue, a carne e os ossos, mas a
honra, o brio, a dignidade, cuspida a todo o instante pela
bava\footnote{ Erro tipográfico de solução incerta.} da ambição. 

 Praticásseis vós este esforço, que não seriam os aplausos da câmara
encomendada\footnote{ As eleições para a Câmara dos Deputados eram influenciadas pelo gabinete
em exercício, cujos poderes de intervenção derivavam, em grande parte,
da superlei de 3 de dezembro de 1841, concebida pelo próprio partido de
José de Alencar. À medida, porém, que o Imperador usou do Poder
Moderador para alijar os conservadores e eleger deputados favorecidos
por gabinetes concorrentes, os conservadores passaram a criticar o
sistema eleitoral do Império. A crítica mais sistemática é o
\textit{Systema Eleitoral no Brazil; como funciona, como tem
funcionado, como deve ser reformado} (1872), de Francisco Belisário,
deputado que defendeu a escravidão nos debates sobre o projeto do
Ventre Livre e se frustrou com a aprovação da lei, conseguida em grande
parte graças às ameaças de dissolução da Câmara. O epíteto ``Câmara
encomendada'' indica a conduta revisionista dos conservadores já em
meados da década de 1860.} nem as palavras rituais da imprensa a receber essa prova de amor e
abnegação de vossa parte. Seriam as bênçãos sinceras de todo o país, as
efusões de uma população inteira, sentindo que a mão poderosa e
solícita de seu monarca a suspendia às bordas do abismo onde vai desabar. 

 Eu vos suplico, senhor, pelo vosso dever primeiro; por nossa pátria
depois; e pela dinastia finalmente, que vossa pessoa, bem sei, não vos preocupa!

 Eu vos suplico com todas as potências de minha alma; salvai o Brasil e,
com ele, os penhores de sua integridade.

 Não acabaria com meu coração que vos ele pedisse para mim o que quer
que fosse. Esquiva"-se quanto pode de o fazer aos que lhe estão
iguais. Mas, para minha pátria, para este Brasil tão angustiado quanto
desquerido dos filhos que mais lhe devem; para este império, ainda
fraco e tolhido, onde eu tenho um cantinho humilde que não trocara
pelas maiores celebridades e grandezas do mundo; para este solo, que
Deus abençoou e malsinam os homens; não tenho pejo de suplicar"-vos, senhor. 

 Ou vós ou a revolução. Fora daí, nada existe neste imenso vácuo do presente. 

 Muitos increpam semelhante insistência, que, não obstante, se conservam
impassíveis. Estranham que se peça ao monarca a salvação do país, como
se o monarca fosse inventado para outra coisa, senão para representar a
missão de uma providência nacional. Entretanto, eles, que censuram,
nada obram, nada absolutamente.

 Estáticos à margem dos acontecimentos, que se despenham do alto e fogem
com deslumbrante velocidade, assemelham"-se às aves aquáticas,
taciturnas e sombrias, quando se quedam à beira do rio, com os olhos
fitos na correnteza das águas. 

 Às vezes o viajante que devassa estas paragens ouve um pio triste e
lúgubre a reboar no seio da melancólica solidão. É o grito sinistro de
alguns pássaros, que anuncia a borrasca; depois tudo cai e sepulta"-se
no profundo silêncio; e o rio, toldado pela vasa, continua a correr em
demanda do oceano, túmulo insondável de quantas catástrofes!

 Não encontrais em vossa marcha, senhor, a mínima resistência. Ao
sobrecenho imperial curvam"-se as venerandas cabeças dos cidadãos
encanecidos no traquejo dos negócios públicos. O senado brasileiro,
onde outrora se quebraram as ondas revoltas da anarquia, já não opõe
diques à torrente da corrupção. Vosso ministério pode apresentar"-se
ali com os fardões cobertos de sangue brasileiro e estender a mão, que
o conselho dos anciãos lhe abandonará a bolsa do cidadão e os destinos da pátria.

 Raros, dois ou três, se tanto, ficariam imóveis nas
curules,\footnote{ \textit{Curul}: assento reservado aos altos dignatários na Roma antiga.}
 como os padres conscritos, quando César lhes pedia a ditadura. 

 O senado não teme as iras do leão, mas sim a hidra que se enrosca na
sombra. Erro fatal, que teremos de expiar cruelmente. A única maneira
de evitar a revolução da anarquia, que se está cevando com os desatinos
da atualidade, seria a revolução da lei, a resistência constitucional
dos poderes do estado, a quem a nação confiou a grave e suprema
atribuição conservadora. 

 Negar ao governo pão e água, recusar"-lhe abertamente o orçamento e
abrir a luta franca e leal com a coroa, era a atitude do senado neste
momento culminante. Teríeis então de resolver, senhor, se as
instituições do país deviam de ser imoladas ao vosso
gabinete.\footnote{ Gabinete da Liga Progressista. O autor insiste na ideia de falsidade da
representação política deste ministério (e da Câmara eleita por ele),
pois seria forjado pelo Imperador e desprovido de vínculos com a sociedade.}
 

 Neste caso, a nação ficava sabendo com que podia contar. Caiam as
máscaras da comédia constitucional e entrávamos em pleno arbítrio. Ou
receberíeis como Napoleão \textsc{iii} a nova investidura nacional e podíeis
então dispor deste Brasil com direito perfeito, como coisa vossa; ou a
nação, acordados os brios da prisca liberdade, vos faria conhecer a sua
vontade imutável, e havíeis de obedecer"-lhe como seu primeiro cidadão
e seu primeiro súdito. 

 Mas o senado, em quem estavam postos os destinos do país, encadeou a
revolução legal e deixou subir o nível da arbitrariedade e prepotência.
Há de chegar às bordas e extravasar. O que ficará depois da aluvião?\ldots{}

 Deus o sabe. 

 Só vós, senhor, tendes em vossa mão o cravo da roda fatal; porque só
vós existis neste país, como poder, como força, como opinião. É triste
para um cidadão, filho de um povo livre, confessar estas coisas; mas
são verdades que transbordam sem querer da alma, e é preciso que
transbordem para não afogá"-la. 

 Se, por momentos, um homem, uma voz, um eco mesmo, se levanta para
opor"-vos, não de frente --- quem ousara? ---, mas de longe, através do
ministério, uma resistência oficial; é efeméride política de breve
momento. Dura ainda a surpresa de semelhante energia, que já ela todo
se desvaneceu. 

 Rumorejam baixo uns sussurros misteriosos. Aludem a certos colóquios;
citam"-se palavras sibilinas. E toda a população acha natural que o
homem se incline, a voz emudeça e o eco se dissipe. 

 Tendes, senhor, para tudo, daqueles argumentos de que fala D.~Basílio:
--- \textit{certi argumenti a cui no si resiste} ---;\footnote{ Frase da ópera 
\textit{Il Barbiere di Siviglia} (1816), de Gioachino Rossini
(1792--1868). No original, a fala proferida pela personagem D.~Basílio
é ``certi argomenti a cui non si risponde'' (Ato \textsc{ii}, cena \textsc{xiii}), isto é, a
certos argumentos não se responde.}
 o dilema terrível da pistola e da bolsa; da graça e da desgraça.
Nomeais ministros contra a vontade; alcanceis enviar ao Rio da Prata,
como embaixadores, pessoas de perfeito juízo, coisa inverossímil. De um
homem sisudo, de um caráter severo, tirais de repente não sei por que
alquimia, um aventureiro político ou um estadista poltrão. 

 Enfim, senhor, fazeis do preto branco: e até aquele milagre incrível,
que excedia à onipotência do parlamento inglês, de fazer de um homem
mulher e de uma mulher homem, para vós é nonada.

 Mulheres haveis feito de quase todos estes cidadãos que cercam o trono
e, em vez de resistir"-vos para vos salvar contra vossa própria
obstinação, se contentam de chorar contritas no regaço imperial as
misérias da pátria, sentindo"-se consoladas depois deste desabafo.

Não há meses, vimos estadistas ilustres e alguns dos mais famosos
sacerdotes da liberdade empenhados em fazer constitucionalmente um
varão de uma senhora, somente para vos ser
agradável.\footnote{ Referência não identificada.}
 Se não conseguiram de todo, foi porque pairou nos ares uma dúvida a
respeito do contentamento que vos traria esta fineza. 

 O ministro de vossa íntima confiança, o Sr.~Zacarias, com quem estais
em tão perfeita correspondência epistolar, opôs"-se. Então,
suspeitaram que a prudência do rei houvesse derrogado a ternura do pai.


 Estas divagações, próprias de um espírito alvoroçado, me afastam do
assunto. Ainda vos não disse todo meu pensamento a propósito da vossa
carta. Não acrescentou essa generosidade um ponto sequer à vossa
reputação. Bem conhecidas e justamente apreciadas são a singeleza de
costumes e a sobriedade de vida, que distinguem o monarca brasileiro.

 Ao contrário, pelo modo por que o praticastes, semelhante ato vos
prejudicou no ânimo público. Não havia necessidade dessa solene
confissão, feita em pleno parlamento, dos desarranjos da casa imperial.
Se vosso desinteresse não estivesse acima de qualquer suspeita, diriam
que era um pretexto fornecido para a recusa do donativo. 

 Sobretudo, fostes mal inspirado, tornando em galardão a um indivíduo um
ato vosso de patriotismo.

 Napoleão \textsc{iii}, a quem a França se doou pelo sufrágio universal, escreve
cartas lisonjeiras a seus ministros e até lhes envia mimos de
brilhantes. Mas ainda não se animou a fazer da miséria pública um
pedestal à glória equívoca de
Rouher!\ldots{}\footnote{ Eugène Rouher (1814--1884), senador e presidente do Conselho de Estado
de Napoleão \textsc{iii}, apodado de ``vice"-empereur'' por causa da proximidade 
com o Imperador e de sua ascendência política no Parlamento francês. No
início de 1867, Napoleão \textsc{iii} escrevera uma carta pública a Rouher,
advertindo"-o da necessidade de reformas a que o estadista se opunha.
Na passagem acima, Alencar insinua a D.~Pedro~\textsc{ii} que há um limite, se o
bem público o exige, na relação do governante com seus protegidos. O
alvo é Zacarias, em particular, e a Liga Progressista, em geral.}

\begin{flushright}
\textit{20 de setembro\\ 
Erasmo}
\end{flushright} 


\chapter[Sexta Carta]{Sexta Carta \subtitulo{sobre a guerra}}

\noindent\textit{Senhor}\smallskip

A paz é uma grande vergonha\ldots{}

 O coração brasileiro se congela ao som desta palavra cruel. Reflui o
sangue açoitando as faces do cidadão brioso, que se estremece pela
honra nacional. 

 A paz é um ato de miséria\ldots{}

 O Brasil, a segunda nação da América, destinado à primazia do mundo,
abater seu estandarte ante o arreganho de um pequeno déspota, quase
selvagem?

 Não há filho deste império que se não possua de horror ante a
possibilidade de semelhante opróbrio.

 A paz é uma vilania\ldots{}

 Não tem alma um povo de onze milhões de almas que não esmaga a
insignificante republiqueta por falta de um exército de cinquenta, de
cem, de duzentos mil soldados. Povo pusilânime, avaro de seu sangue e
desamparado do sentimento de sua dignidade!

 Eis o que murmura dentro de vossa alma a voz do pundonor, o pátrio
orgulho.

 Mas, senhor, há coisa pior que a paz. Há outra vergonha, outra miséria,
outra vileza superior a essa. É a guerra como a tem feito vosso governo. 

 Não se concebe que o Brasil possa em condição alguma sofrer maiores
humilhações do que tem curtido sob a influência maléfica da política
internacional inaugurada em 1864.\footnote{ Ano dos acordos diplomáticos 
que levaram à Guerra do Paraguai (1864--1870).}
 

 Esta é também, senhor, a convicção do país.

 Entre dois males terríveis, entre a vergonha da paz e a ignomínia da
atualidade, ele prefere o menor. Dói"-lhe muito deixar incólume a
afronta do Paraguai; porém, dói"-lhe mais cruamente ainda servir de
alvo ao insulto de seus aliados e ao menoscabo do mundo.

 A guerra sob a política dominante tornou"-se impossível.
Compenetrai"-vos bem desta verdade, que é implacável, senhor. Curvemos
a cabeça ao peso da fatalidade. Não há resistir"-lhe. 

 Este gabinete não consegue mais do país o exército indispensável para o
nosso triunfo; não alcança um subsídio sequer de dez mil homens para
suprir as falhas de nossos batalhões.

 Ponham em jogo todos os meios, a sedução como a violência; serão baldados.

 Nenhum brasileiro empunhará as armas para submeter"-se às ordens de um
general estrangeiro, que escarnece impunemente de nossa
pátria.\footnote{ Alusão ao general Bartolomé Mitre, presidente da Argentina (1862--1868),
nomeado comandante em chefe das Tropas Aliadas, cargo que exerceu até janeiro de 1868.}

 Nenhum cidadão deixará sua família ao abandono para
estrecer\footnote{ \textit{Estrecer}: perder a força, desvanecer.}
 nos pântanos do Paraguai, testemunha importante de nossa degradação.

 Nenhum homem de brio arriscará a vida inutilmente para receber em
prêmio de seu heroísmo sob a forma de medalha uma ração de opróbrio e desonra.

 Não há mais quem sacrifique uma só gota de sangue para defender a
dignidade de um país, que seu próprio governo é o primeiro a aviltar e prostituir.

 Não há mais quem sinta ferver em sua alma os entusiasmos generosos da
honra nacional, desde que a sepultaram nos arquivos de Buenos Aires em
tratados e notas de perpétuo estigma.

 Não há mais um filho que se estremeça ao grito da pátria ofendida;
porque a pátria já não existe. Puseram no lugar dela um mercado de
condecorações, um prostíbulo da glória nacional.

 Qual nobre estímulo há de levar agora os brasileiros ao Paraguai?

 Tudo se poluiu; tudo se profanou. Ao heroico defensor do pavilhão
nacional, o Brasil desgraçadamente já não tem outro meio de o
distinguir senão alquilando"-lhe o valor e a intrepidez pelo custo de
alguns escravos!\footnote{ A Guerra do Paraguai exigiu do Brasil mais de 
110 mil soldados, número quatro vezes maior do que o país já recrutara até o momento. 
Em vista das dificuldades, o governo liberou fundos para alforriar escravos 
e enviá"-los ao \textit{front} (vide nota 10, p.~\pageref{nota10}).} 

 À mocidade generosa que se arrancasse dos cômodos da abastança e dos
prazeres dessa floração da vida para correr em defesa do emblema
nacional, o lugar nobre que lhe reserva o governo é ao lado do galé,
como seu companheiro de grilhão!\footnote{ A partir de 1866, o governo recorreu ao 
recrutamento de presidiários das colônias penais de Fernando de Noronha, de Mato Grosso 
e de outras regiões, para incrementar o efetivo militar no Paraguai.}

 Deus! A que profundidade já chegou a perversão do senso moral neste
desgraçado tempo? 

 E o partido\footnote{ Alusão à Liga Progressista (vide Introdução).}
 que reduziu o país a tal extremo, que espancou todos os princípios da
probidade política, assim como do santo patriotismo, ousa invocar o
povo brasileiro em nome da dignidade nacional, que ele próprio fria e
calculadamente abateu, fazendo tapete dela à arrogância gaúcha do Rio da Prata?

Mas há de ter do país a resposta que merece; uma gargalhada de mofa!

 A defesa da honra nacional já não está agora nos campos do Paraguai,
não. Transferiu"-se para aqui, para esta cidade, corte do império,
coração atrofiado deste povo infeliz. 

 Daqui partiu todo o mal; o miasma funesto desta guerra; a praga ainda
mais terrível da tríplice aliança; todo este ramo de peste enfim, que
nos tem custado tantas vidas, tanto ouro, e\ldots{} o que é mais duro,
tantas afrontas!

 Daqui vão ainda e irão as ordens para as constantes humilhações que
diariamente chovem sobre o país, como para submeter às provas
evangélicas sua admirável longanimidade.

 E, por fim, senhor, quando esta política fatal tiver esgotado a série
extensa das transações indecorosas, porá a esse tráfico da honra
nacional, mareada pela ambição do poder, um remate digno da obra: a paz!

 Não tenhais dúvida, senhor.

 Eles, que atualmente se entumecem com a ênfase de um fofo patriotismo e
bramam contra a mera possibilidade de pôr um termo digno à interminável
campanha, prescindindo da vitória; eles mesmos seriam os mais
fervorosos a abraçar"-se com a paz, se vissem nela encarnada a sua
ambição. 

 Querem a guerra presentemente, a guerra a todo o transe; porque esta
significa o pleno arbítrio, a onipotência administrativa, a indulgência
magna de todos os erros e de todos os crimes. A esta palavra mágica
nada se opõe; o país entregou"-se manietado ao governo.

 A guerra presta ainda ao ministério de 4 de
agosto\footnote{ Trata"-se do gabinete de 3 (e não 4) de agosto de 1865, presidido por 
Zacarias de Góis e Vasconcelos. Como esse ministério vinha estimulando o debate em torno da 
emancipação, as \textit{Novas cartas políticas} o chamam, com ironia e derrisão, de ``filantrópico''; porque se 
beneficiava da política da Conciliação, instilada por D.~Pedro~\textsc{ii}, foi apodado de ``conciliador''.} 
um serviço relevante. Apavora os estadistas eminentes que poderiam
salvar o país com seu tino e energia. Há alguns que repelem até a
possibilidade de serem chamados ao poder, porque receiam a
responsabilidade tremenda desta situação. 

 Esse monopólio do governo, garantido ao atual gabinete pela repugnância
de uns e incapacidade de outros, é um dos proventos da permanência
deste estado de coisas. Não convém aos progressistas mudar a situação,
resolvendo a questão eterna. 

 Mas, senhor, repita o país amanhã na praça, em alta voz, o que já vai
dizendo em casa, a meia voz, de timão e barrete. Bata o pé ao governo e
exija a paz; que o marcial gabinete de 4 de agosto, de pronto
conciliador e filantrópico, aceitará aquela solução. 

 Virão a lume os princípios da civilização, o amor da humanidade e toda
essa larga provisão de filantropia cristã, que tanto serviu para as
festas da capitulação de
Uruguaiana.\footnote{ Cidade brasileira tomada por tropas paraguaias em agosto de 1865 e
reconquistada pelos aliados em setembro do mesmo ano.}
 Cantar"-se"-ia em todos os plectros a vitória incruenta da diplomacia!

 Não duvidariam apor as armas do Brasil com as armas do Paraguai nalgum
papel com figura de tratado, digno reverso da tríplice aliança!

É possível que haja brasileiros capazes desta enormidade? Mas, senhor,
esses de que vos falo não são brasileiros, são ambiciosos. Sua pátria é
o governo; mantendo"-se aí, dão a maior prova de civismo e abnegação. 

 Eis a que nos conduzirá infalivelmente a insistência do atual gabinete.
No fim de uma campanha vergonhosa, uma paz humilhante. Ao cabo de
tantos sacrifícios de toda a casta, a consagração da afronta por meio
de uma estipulação indecorosa. 

 Neste caso, antes começar pelo fim.

 Poupar"-se"-iam o tempo, o ouro e até mesmo a dignidade tão
longamente enxovalhada. Em vez de formar um livro triste na história
pátria, a questão paraguaia ficaria apenas como um ponto negro, que o
heroísmo brasileiro não tardaria ofuscar com os esplendores de outras
glórias mais puras e dignas.

 Cego e obstinado, o gabinete se recusa à convicção de sua impotência.
Acredita que pode ainda levantar um exército e, com ele, arrebatar por
fim o triunfo. Na efusão de regozijo nacional pela justa reparação,
esperam os ministros obter da magnanimidade do povo a absolvição de
tantos erros.

 Como se a história, implacável e severa, não os aguardasse nos umbrais
da posteridade para precipitá"-los naquele nono círculo do Dante, onde
se convulsam os
patricidas.\footnote{ Na \textit{Divina Comédia} (início do século \textsc{xiv}), Dante reservou o nono e
último círculo do inferno aos traidores de todos os tipos, isto é, do
próprio sangue, da pátria, dos amigos e dos senhores (parte \textsc{i}, cantos
32--34). As \textit{Novas cartas políticas} acusam a Liga Progressista
de traição contra a pátria.}
 
 Esse exército que se exige do Rio da Prata e sem o qual parece
impossível desfechar o golpe decisivo, onde o haverá o governo?

 Do patriotismo?

 Impossível, repito; porque ele não existe mais, senhor. 

 Da violência?

 Grande temeridade; colocada a questão nesse terreno, desde que se calam
os brios nacionais, clama o instinto da conservação individual.

 Do ouro?

 No tempo em que as guerras eram questões dos reis, que as faziam por
sua conta, se toleravam os exércitos mercenários. Combatiam pelo
capitão que lhes pagava; nada mais justo. Neste século, porém,
tornando"-se as guerras questões dos povos, não parece decente que
eles confiem a mãos estranhas a defesa de sua honra. 

 Demais, o ouro escasseia; muito há que o anunciou o termômetro
infalível de Buenos Aires. Sobrasse ele, porém, que não acharia
emprego; carece o país daquelas sobras de população, ou nacional ou
estrangeira, que em falta de outra exploram a indústria da guerra.

 Apontam outra fonte como aquela de onde pode o governo tirar um forte 
exército de vinte ou trinta mil homens.\footnote{ \label{nota10}Em 5 de novembro de 1866, o Conselho de Estado 
deliberou sobre a possibilidade de alforriar escravos para empregá"-los na Guerra do Paraguai. 
Poder"-se"-ia libertar os cativos da nação, os de ordens religiosas e, finalmente, os de senhores 
particulares mediante ressarcimento. As \textit{Nova cartas} censuram a proposta, advertindo contra os gastos 
indenizatórios e o abalo social da medida. Seus argumentos seguem de perto o voto do conselheiro 
visconde de Itaboraí, chefe dos saquaremas. Confira \textsc{rodrigues}, J.~H.~(org.). 
Atas do Conselho de Estado. Brasília: Senado Federal, \textsc{vi}, 1978, pp.~45--54.} 
Asseguram que a medida já foi resolvida em conselho e se realizará
apenas encerrada a sessão. 

 São vinte mil contos de réis pelo menos, para um país que já lançou mão
do papel"-moeda na importância de cinquenta mil, como o único meio de
prevenir a bancarrota. É cerca de um terço mais no presente orçamento,
já onerado com um déficit bem considerável. 

 Mas arrede"-se a questão de dinheiro, que está na superfície;
acha"-se no âmago a questão máxima, incandescente, medonha, a
questão"-cratera, que desde um ano a esta parte está em ebulição no seio do país.

 Quisera, senhor, dirigir uma só pergunta aos vossos conselheiros,
àqueles que vos inspiram semelhantes ideias.

 Se eles pertencessem a uma casta sujeita e, de repente, se achassem
investidos da força pública no país de sua opressão; qual seria o
primeiro irresistível impulso de seu coração?

 Defender a pátria alheia, pretendida sua desde a véspera unicamente; ou
reclamar igualdade para seus irmãos, seus pais e seus filhos ainda sujeitos? 

 É preciso contar com os instintos naturais do coração humano; e não
entregar o gládio da justiça nacional à mão capaz de espedaçá"-lo para
fazer dele um punhal contra o império. 

 E os cidadãos privados de repente de sua propriedade, embora mediante
indenização; as lavouras desertas dos braços que a trabalhavam; os
estabelecimentos rurais alvorotados com a execução da medida; a nova
massa recrutável sôfrega por caber toda no limitado algarismo da
desapropriação; toda essa perturbação social, toda essa efervescência
das fezes vivas; não é coisa que mereça do governo algum desvelo?

 Não é digno do país, sem dúvida, esse pacto de sangue com os deserdados
da liberdade. Dizer"-lhes: ``Se quereis ser homens, arriscai a vida em
defesa daqueles direitos, daquela independência e dignidade, de que por
necessidade vos privamos. Não quereis ser carne para o látego, sede,
pois, carne para o canhão.''

 Os manes\footnote{ \textit{Manes}: espíritos.}
 dos veneráveis autores da constituição devem estremecer vendo o uso que
esta geração pretende fazer daquela sábia e prudente disposição por
eles escrita no código de nossas liberdades. Nunca pensaram, decerto,
que pudesse ela autorizar tamanha
imprudência.\footnote{ A passagem se reporta ao artigo \textsc{vi}, 
parágrafo \textsc{i}, da Constituição de 1824,
que concedia os direitos de cidadão ao escravo brasileiro que
alcançasse alforria. Esse dispositivo era lembrado com relativa
frequência por defensores do cativeiro no Império, como prova de que a
sociedade escravista brasileira aceitava ex"-escravos ou descendentes
negros como membros da sociedade civil e política, sem atentar para a
cor de pele nem empregar critérios raciais.}

 Escravos combateram na independência. Mas como? Por impulso próprio,
por entusiasmo espontâneo, esposando a causa de seus senhores. Assim,
mostraram"-se dignos da liberdade que tão heroicamente defendiam. 

 Réus de política saíram dos cárceres e pelejaram pela causa do Brasil.
Mas por quê? Eram réus da liberdade, vítimas do despotismo; embora
criminosos, sofriam a opressão de leis iníquas e bárbaras, contra as
quais tinham também o direito de combater.

 De resto, se houve alguma coisa de censurável, então evitemos a
reincidência, antes do que alardeá"-la. Não façamos de um erro da
juventude um crime da virilidade.

Suponho que o projetado exército de trinta mil homens se levanta;
marcha para a campanha do Paraguai; e toma de assalto as fortificações
de Curupaiti\footnote{ Forte que servia de defesa avançada da fortaleza de Humaitá. 
Em 1866, após desastrosa tentativa de invadi"-lo, em que morreram milhares de brasileiros 
e argentinos, as críticas à guerra se avolumaram na opinião pública dos países aliados.} 
e Humaitá, aniquilando assim o último reduto de Lopez. 

 Quando voltasse triunfante aquele exército, integralmente composto de
outra raça, não teria ele o direito de dizer"-nos a todos, a vós como
a qualquer outro cidadão: ``Esta pátria vos não pertence, pois que a não
pudestes defender. Somos nós, os filhos da vitória, coroados dos louros
do combate, somos nós os verdadeiros cidadãos do império brasileiro,
que elevamos por feitos heroicos a uma posição respeitável.
Arredai"-vos para que tomemos posse dos destinos deste país, ganho por nosso valor.''

 E que responder a essa formidável apóstrofe? 

 Arcabuzá"-los?\ldots{}

 Impedi, senhor, a realização deste plano funesto. Não querendo o
imperador, nada se faz: o país inteiro sabe disto e consente.
Abandonou"-se completamente ao seu monarca, não pelo sufrágio
universal, como a França, mas pela geral indolência. É uma felicidade
para ele haver quem o dispense da fadiga de pensar, de querer e de obrar. 

 A vitória com semelhante exército é mais degradante do que a derrota.
Antes o Brasil vencido por Lopez, isto é, pelos obstáculos insuperáveis
da natureza aproveitados pela arte, do que vencidos pela nossa
fraqueza, pelo menospreço da própria dignidade.

 Portanto, senhor, se, apesar da desmoralização do atual gabinete e da
impossibilidade de prosseguir na campanha, persistis em sustentá"-lo,
neste caso, em nome do país, eu vos peço a suspensão das hostilidades.

 Mandai que nossas forças recolham às fronteiras. Uma divisão de
encouraçados pode continuar nas águas do Paraná a hostilizar o inimigo.
Tratemos de organizar o exército de Mato Grosso, o que devera ter sido
o nosso primeiro cuidado; e, sem fazer a paz, como quem abandona uma
empresa mal delineada reservando"-se o direito de renová"-la mais
tarde com sucesso, faríamos uma pausa ao menos nas calamidades do
presente. 

 Fora indigno, decerto, celebrar a paz com o Paraguai; nem há brasileiro
que sofra a só ideia de semelhante baixeza. Não é indecoroso, porém,
abandonar esse povo infeliz à tirania de Lopez, na qual persiste; e
reconhecer o império a impossibilidade de penetrar agora no antro do déspota.

 O maior capitão da Antiguidade, Alexandre, não conseguiu abater a
resistência de um povo bárbaro, os
Citas,\footnote{ Conjunto de povos que habitavam a região da Eurásia. Os próximos parágrafos
arrolam inúmeras derrotas militares de civilizações ilustres, como
exemplos históricos particulares do tópico sobre o possível e conveniente. Para
Alencar, reveses bélicos para um povo menor não implicam desonra a uma nação
brilhante; daí a conveniência do armistício.}
e por isso não ficou mareada a sua glória, a que a providência havia
assinado mais altos destinos do que o desbarato de algumas hordas selvagens. 

 Roma, já orgulhosa república, derrotada pelos
Samnitas,\footnote{ Antigo povo da península itálica que derrotou Roma no séc.~\textsc{iv}.}
curtiu a vergonha de ver passarem seus exércitos pelas forças
caudinas.\footnote{ Forças samnitas.}
 Mais tarde, poderoso império, duas vezes tentou invadir a
Pártia\footnote{ Mesmo que Império Arsácida, potência hegemônica do Oriente Médio 
que impôs duras derrotas aos romanos, vistas por Plutarco como entre as piores 
tragédias militares da época (vide ``Vida de Crasso'', in \textit{Vidas paralelas}).}
 e duas vezes foram destroçados seus numerosos exércitos.

 Em 1498, o imperador Maximiano \textsc{i},\footnote{ Maximiliano \textsc{i} (1459--1519), 
imperador do Sacro"-Império Romano Germânico.} 
então o maior soberano da Europa, sentiu quanto o sentimento da
independência fortalece um pequeno povo. Oito vezes batido em oito
meses pela Suíça, foi coagido a desistir da projetada conquista. 

 Inglaterra não penetrou no coração da Índia de um jato. Foi depois de
uma luta porfiada, a preço de muito sangue, que ela fundou sua
dominação asiática. Também a França teve de suportar enormes
sacrifícios e sucessivas derrotas, antes de conquistar sua colônia de
Algéria.\footnote{ Argélia.}


 O poder colossal da Rússia por longo tempo se quebrou ante a coragem
indômita das tribos caucasianas. Desde 1839 até nossos dias, o
intrépido Shamil\footnote{ Imame Shamil (1797--1871), líder político"-religioso de povos
islâmicos na resistência à invasão russa do norte do Cáucaso.}
 zombou dos exércitos aguerridos do autocrata. 

 Ultimamente, França, a Palas armada da Europa, retirou suas forças do
México sem haver conseguido a completa submissão do país. Não foi ao
infeliz Maximiliano, mas a Napoleão \textsc{iii}, que Juarez destronou do sólio
mexicano.\footnote{ Alusão ao apoio que Napoleão \textsc{iii} prestou à oposição mexicana nos conflitos que
resultaram no fim da república e na deposição do presidente Benito
Juárez. Sob a proteção do Imperador francês, o arquiduque austríaco
Maximiliano de Habsburgo assumiu o trono do efêmero Império Mexicano
(1864--1867), mas acabou deposto e fuzilado pela reação republicana.}


 E dirá alguém que Roma, Alemanha, Rússia, Inglaterra e França ficaram
desonradas perante a posteridade, porque recuaram ante a
impossibilidade a fim de recolher as forças e superar de um impulso os
obstáculos naturais?

 Os remoinhos e as barrancas do Paraguai valem sem dúvida os
desfiladeiros de Clúsio, as geleiras da Suíça, o clima deletério da
Índia, as estepes da África e os despenhadeiros do Cáucaso. 

 Há estadistas, senhor, que adejam pelas alturas e se prendem como os
insetos às teias de aranha. A estes parecerá sem dúvida uma coisa
inaudita e espantosa essa suspensão de uma guerra, sem as fórmulas
consagradas pelos estilos, sem o conveniente aparato da diplomacia, tão
funesto ao país. 

 Bem compreendeis, senhor, que não devemos sacrificar a dignidade
nacional por tais filigranas de ouro falso. Ainda quando a Europa,
mesmo nos tempos modernos, não houvesse dado o exemplo de cessação das
relações internacionais entre nações inimigas, podíamos nós
admiti"-lo; nós, que não reconhecemos nenhum equilíbrio americano; e
não consagramos, portanto, o princípio da intervenção. 

 Mas não creio que o Brasil tenha chegado a um tal estado de inanição,
para suspender a guerra e deixar impune o Paraguai; o que se observa é
somente prostração e torpor; é abatimento causado pela obsessão deste
gabinete, que sufoca a nação, como um pesadelo horrível.

 Retire"-se esta opressão, e o país há de recuperar as forças inertes,
os brios abatidos. O império será outra vez o Brasil da independência,
o Brasil de 1851. 

 Um novo gabinete, composto de boas inteligências e, sobretudo, de
corações de lei, é a única salvação possível para a honra nacional
comprometida no Paraguai e para as instituições pátrias, ameaçadas
aqui, no seio mesmo do país. Um novo gabinete, rico de energia, será o
cravo da revolução, o freio da anarquia.

 Apressai"-vos, senhor, a brindar o monstro que avança. Escolhei homem
capaz de o domar; senão, é inevitável a devastação do império.
Iludi"-vos, se pensais que teremos outro 42 ou
48.\footnote{ Alusão às rebeliões liberais de 1842, em São Paulo 
e Minas Gerais, e de 1848, em Pernambuco, deflagradas após a ascensão de ministérios conservadores.}
 Infelizmente, não há de ser o desespero de um partido que prorrompa;
mas o desprezo formidável de uma sociedade inteira.

 O novo gabinete deve ser exclusivo em política, filho de um só partido
e compacto em uma só vontade. O
contubérnio\footnote{ \textit{Contubérnio}: coabitação. Trata"-se de uma censura oblíqua ao preceito,
imposto por D.~Pedro~\textsc{ii} desde a década de 1850, segundo o qual os
gabinetes em exercício deveriam compor ministérios mistos (com liberais
e conservadores). O Imperador via nesse princípio, batizado de política
da Conciliação, a única maneira de provocar a partilha do poder sem a
dissolução da Câmara dos Deputados, já que o gabinete influía nas
eleições gerais, garantindo invariavelmente maioria absoluta e
perpetuando"-se no poder. O núcleo do Partido Conservador, de José de
Alencar, sempre se opôs, às vezes implícita, outras explicitamente, à
Conciliação.}
 de opiniões diversas é uma prostituição como qualquer outra; não será
lastrando mais a corrupção e envolvendo nela os homens ainda puros que
se há de servir à causa nacional. 

 Se os estadistas brasileiros não podem salvar a pátria senão por este
meio, eu respondo por ela, sem receio de ser desmentido: ``Por tal
preço, não queremos a salvação. Venha então o terrível batismo com que
a Providência nos há de purificar da mácula; para que outra vez sejamos
nação, pois agora quase não temos direito a esse título!''

 É preciso que o novo gabinete tenha bastante civismo para arrostar as
dificuldades da guerra, se for necessária a sua continuação; e afrontar
com as odiosidades e prevenções da paz, caso se torne esta
indeclinável. O partido que trepida diante dessa grave responsabilidade
e carece de reparti"-la com outros não é partido, mas um acervo de
ambições, que por bem do país conviria aniquilar.

O partido conservador está designado pela lógica dos fatos como o
depositário da situação. Não tem a cumplicidade desta guerra; não o
tolhem compromissos do passado. Entraria no poder com a imparcialidade do juiz.

Se o partido conservador recusar o sacrifício, serei o primeiro,
senhor, a proclamá"-lo traidor à pátria e a pedir a sua dissolução,
como uma necessidade pública e uma justa punição.

Pese bem o imperador as circunstâncias do país. O atual gabinete criou
uma situação ambígua e indefinível; a guerra, com todas as vergonhas da
paz, porque não vencemos nem mesmo combatemos; a paz, com todos os
encargos da guerra, porque o ouro jorra de contínuo para o sul, de
envolta com o soro do sangue brasileiro. 

\begin{flushright}
\textit{Rio, 23 de setembro\\
Erasmo}
\end{flushright}

\chapter{Última Carta}

\noindent\textit{Senhor}\footnote{ Publicado após longo intervalo de seis meses, esse texto, 
que glosa a decadência política geral do Império, é desconhecido de alguns pesquisadores e biógrafos, 
cujos trabalhos registram apenas seis \textit{Novas cartas políticas}. Dele reproduzimos o epílogo, 
que anuncia a despedida da campanha epistolar de Erasmo.}\smallskip

 Aqui ponho fim à minha missão na imprensa. Esta é a última carta, a
derradeira palavra que vos dirige o escritor desconhecido. 

 Apareceu ele em fins de 1865; e desaparece hoje para sempre da imprensa
brasileira. Se a não honrou com os esplendores do talento, ao menos aí
deixa uma memória estimada pela franqueza e sinceridade. 

 Quem foi Erasmo, estou convencido que o sabeis. O coração do homem de
bem é uma pedra de toque para as pessoas que dele se aproximam. Desde
os primeiros tempos distinguistes dos assomos do despeito e da ambição
a palavra de um cidadão leal, amigo do soberano, porém súdito
principalmente da verdade e da justiça.

 Não lhe conheceis o nome, e para quê? 

 Esse nome não tem serventia no mundo político. Não podem fazer dele nas
circunstâncias atuais nem um escândalo nem um martírio. Seria uma
questão de letras; fútil curiosidade e nada mais. 

 Se, para dirigir"-me à majestade do Sr.~D.~Pedro~\textsc{ii}, envolvi"-me no
mistério, não foi por temor. Ninguém neste país ignora que as audácias
contra a pessoa inviolável não só não têm o menor perigo, como são
títulos à grandeza. A generosidade do imperador sabe vingar"-se!

 Assim, quando alguma vez a pena se embebia de verdades mais austeras,
hesitei. Receava ofender"-vos, a vós inofensivo; não queria que minha
palavra parecesse uma covardia ou um cálculo: duas coisas, cada qual
mais repreensível. 

 Só a força da convicção me obrigava a produzir exteriormente o
pensamento; mas então jurava a mim mesmo aprofundar"-me cada vez mais
na humilde obscuridade para me esquivar a qualquer tênue raio de vossa
magnanimidade, ou a algum erradio vislumbre de popularidade. Creio que
o consegui, e com esta íntima satisfação entro no nada donde saí. 

 Foi a consciência que me aconselhou o mistério. Para falar"-vos com a
franqueza precisa, era necessário ter um nome respeitado, cheio de
prestígio e autoridade. Faltando"-me esse título, só me restava o da
verdade. A ideia é essencialmente democrática; ela nivela o trono com o
povo. 

 Fiz"-me ideia, portanto, para ter o direito de interrogar a majestade.

 Se houvesse ameaça de perigo no empenho que tomei, ou eu não me lançara
a semelhante cometimento, pois me falia a coragem, ou saberia afrontar
a publicidade. Mas o perigo estava justamente na sombra, no isolamento,
onde eu permanecia. 

 Aí, senhor, entregue às forças próprias, sem conselho e sem conforto,
vendo abrir"-se em torno um vácuo imenso para a fé que tinha nos
homens; aí, duvidando muitas vezes de mim, único entusiasta no meio da
geral descrença; lutei, senhor, contra a opinião e contra mim mesmo. 

 Há gente para quem o perigo é somente a ofensa física, ou o golpe que
fere o corpo e a bolsa. Materialismo que prostitui a coragem, como tem
prostituído outros sentimentos do homem. A vida e a propriedade, bens
preciosos quando servem a um fim nobre, tornam"-se coisas vis, se
prestam unicamente para depravar o homem e corromper"-lhe a alma. 

 Arrostar a corrupção é, em tempos como estes, mais generoso e heroico
esforço do que nas épocas revolucionárias afrontar a morte e o exílio.
Inebriados pelo entusiasmo, as vítimas da tirania sobem ao patíbulo
coroadas de flores e entoando a canção patriótica. Mas a vítima da
imoralidade está sujeita a cada instante a falsear diante da sedução,
deixando"-se arrastar às
gemonias\footnote{ \textit{Gemonias}: na Roma antiga, parte do monte Capitolino onde se expunham
corpos supliciados.} da desonra e do opróbrio. 

 Não é difícil a quem tem nobres e legítimas aspirações resistir ao
afagos do poder corruptor quando a solidariedade dos homens de bem lhe
serve de apoio. 

 Mas, se tomada de um pânico invencível, a gente honesta se extraviou e,
por uma complacência censurável, cerca os audazes, então faz"-se
necessária uma grande força e constância para preservar"-se do contágio. 

 Que doloroso espetáculo o da atualidade!

 Aos que tombam e se escorjam no pó, a multidão os cobre de aplausos e
ovações. Atualmente é glorioso cair; quase infame recatar"-se. Cada
caráter que vacila e se abate no circo é um triunfador. As turbas o
levantam e carregam aos ombros em troféu. Os homens sisudos, que têm a
franqueza de servir a popularidade, fazem cauda ao cortejo. 

 Esses triunfadores se atraem uns aos outros, onde quer que se achem. O
instinto da conservação os aproxima e identifica. Eles se personificam
em um só e mesmo eu, que por escárnio chamam gênio e virtude. Não há
nada mais comum neste tempo do que os ambiciosos que se estreitam e
fazem bícepes\footnote{ \textit{Bícepe}: dotado de duas cabeças.}
 e trifauces\footnote{ \textit{Trifauce}: que tem três goelas.}
 para ameaçar a sociedade brasileira. 

 Obscuro cidadão, posso, querendo, me submergir na vida privada ou
refugiar"-me na tranquila mansão das letras, como fez o velho Milton
depois de uma vida gasta em defesa das liberdades
pátrias.\footnote{ Referência a John Milton (1608--1684), célebre escritor inglês e
polemista profícuo na Guerra Civil Inglesa (1640--1660), em que atuou
ao lado dos protestantes. Após a Restauração monárquica, retirou"-se
da vida política para a plena dedicação a sua obra máxima, o poema
épico \textit{The Paradise Lost} (1677).}
 Com o direito de escolher o modo de servir o meu país, não estou
privado de subtrair"-me à influência perniciosa da política.

 Mas vós, senhor!\ldots{} Que terrível suplício! Assistir como testemunha
impassível à decadência deste grande império, que Deus formou para os
mais altos destinos! Contemplar de braços cruzados a degeneração desta
raça predestinada, a quem a Providência primeiro abriu a imensidade do oceano!

 Tântalo"-rei,\footnote{ Rei mítico grego que, por ludibriar os deuses no Olimpo, foi condenado
a penar no Tártaro em um riacho cujas águas escapavam de seus lábios e
a viver perto de árvores cujos frutos fugiam de suas mãos. Assim como
Tântalo era privado dos prazeres elementares da vida, D.~Pedro~\textsc{ii} não
encontraria seu principal alívio, a nação brasileira.}
 encadeado a esse tártaro da política, desejareis uma nação e
encontrareis apenas\ldots{}

 \textit{Natio comeda est}, disse
Juvenal.\footnote{ ``A nação é uma comediante'', do poeta satírico Juvenal 
(séc.~\textsc{i} d.C.), cujo original se lê ``natio comoeda est'' 
(``Sátira \textsc{iii}'', 100).}


 Adeus, senhor. Eu me retiro deixando a vez à sátira, que é a eloquência
do presente. Só tomam ao sério as coisas e os homens desta época os
charlatães que se apascentam nela. O cidadão cordato ou chora ou gargalha.

 O tempo não é para Erasmo; mas para Jeremias ou
Rabelais;\footnote{ Alusão, respectivamente, à personagem bíblica Jeremias, que viveu a
captura do Reino de Judá por Babilônia, e ao escritor satírico François
Rabelais, autor de \textit{Gargântua e Pantagruel} (séc.~\textsc{xvi}).}
 para o treno\footnote{ \textit{Treno}: elegia; lamento fúnebre.}
 ou para o sarcasmo. \textit{Ride si sapis}:\footnote{ ``Ri, se
fores sábio'', frase do poeta satírico Martial (séc.~\textsc{i}).}
 diz, como o poeta, a história contemporânea a todo o observador grave
que se esforça por estudá"-la. 

 Adeus, senhor. Se nos dias da próxima tribulação vos parecer
conveniente que a voz frágil deste escritor se levante em defesa das
instituições e do Sr.~D.~Pedro~\textsc{ii}, sua expressão viva, o achareis entre
os raros amigos da adversidade: entre os que já não esperam nem temem. 

 Nada vos devo. Se por seu trabalho o indivíduo que fui recebeu
outrora a honra de servir oficialmente seu país, não é isto favor. Que
o fosse, vosso governo o apagou embaciando o lustre dessa glória
legítima. As aspirações mortas em flor já pagaram à usura aquela distinção. 

 Aprendi, sim, a venerar"-vos como um homem de bem e um príncipe
virtuoso. Fora preciso testemunhar fatos muito graves, para
despedir"-me de uma crença que me acompanha desde tantos anos. Não sei
mesmo se vossos defeitos de rei não são inerentes às vossas qualidades	
de homem. 

 O homem, porém, é nada em um trono constitucional. A excelência do
sistema representativo está justamente nessa virtude de anular a
individualidade do monarca e neutralizar por conseguinte suas paixões.
Não há, não pode haver mau imperador, sob o domínio da constituição
brasileira. Tibério ou Filipe
\textsc{ii},\footnote{ Remissão ao
imperador romano Tibério (14--37 d.C.) e ao monarca espanhol Filipe \textsc{ii}
(1556--1598), cujos governos se tornaram sinônimo de administrações
despóticas.} submetidos a ela, seriam impotentes para o mal. 

 O imperador constitucional é um princípio e, portanto, representa
sempre o bem. Não pode falir, dizem os ingleses. Só erra quando o povo
é ruim, os ministros péssimos e a opinião nula. Neste caso, eu creio
que o despotismo é mais que uma justiça, é uma fatalidade.

 Há exemplos de povos que reclamam um tirano com veemência, qual nunca
sentiram pela liberdade. Roma, abeberada de anarquia, teve a luxúria da
tirania; atirou"-se desgrenhada e ébria como uma bacante aos braços
dos triúnviros e ditadores: de Mário a Silas, de Silas a Pompeu, de
Pompeu a César, de César a Augusto, até que achou os Neros e Calígulas
para a cevarem de torpezas e crueldades.

 A história nos ensina esta grande verdade, que devia ser profundamente
gravada na consciência de todas as nações, e eu a deixo aqui, na página
final, como um símbolo para os brasileiros:\\\ 

\textsc{a liberdade nos países constitucionais não depende do rei, e só do
povo. mudar o rei não é ato de justiça, mas uma vingança mesquinha e
uma inépcia do povo que não sabe governar"-se.}

\begin{flushright}
\textit{15 de março de 1868\\
Erasmo}
\end{flushright}
 

\noindent\textsc{p.s.} Motivos imperiosos retardaram a publicação desta carta.




